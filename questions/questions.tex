\documentclass[a4paper,8pt]{extarticle}

%%% Работа с русским языком
\usepackage{cmap}					% поиск в PDF
\usepackage{mathtext} 				% русские буквы в формулах
\usepackage[T2A]{fontenc}			% кодировка
\usepackage[utf8]{inputenc}			% кодировка исходного текста
\usepackage[english,russian]{babel}	% локализация и переносы

%%% Дополнительная работа с математикой
\usepackage{amsmath,amsfonts,amssymb,amsthm,mathtools} % AMS
\usepackage{icomma}
\usepackage{physics}
\usepackage{multicol}
\usepackage{bm}
\usepackage{mathrsfs}
\usepackage{verbatim}

%%% Номера формул
%\mathtoolsset{showonlyrefs=true} 
%\usepackage{leqno} 

%%% Свои команды
\DeclareMathOperator{\sgn}{sgn}

\usepackage{csquotes} 
\usepackage[backend=biber,style=authoryear,language=auto]{biblatex}
\addbibresource{source.bib}


%%% Работа с графикой
\usepackage{graphicx}
\graphicspath{{images/}}  
\setlength\fboxsep{3pt} 
\setlength\fboxrule{1pt} 
\usepackage{wrapfig} 
\usepackage{tikz}
\usepackage{pgfplots}
\usepackage{pgfplotstable}
\usepgfplotslibrary{polar}
\pgfplotsset{compat=1.18} 

%%% Работа с таблицами
\usepackage{array,tabularx,tabulary,booktabs}
\usepackage{longtable}  
\usepackage{multirow} 
\usepackage{caption2}[2008/03/29]
\usepackage{soul} 

%%% Теоремы
\theoremstyle{plain} 
\newtheorem{theorem}{Th}[section]
\newtheorem{proposition}[theorem]{Proposition}
 
\theoremstyle{definition} 
\newtheorem{corollary}{Corollary}[theorem]
\newtheorem{problem}{Problem}[section]
\newtheorem{definition}{Def}[section]
 
\theoremstyle{remark} 
\newtheorem*{nonum}{Solution}

%%% Программирование
\usepackage{etoolbox} 

%%% Гиперссылки
\usepackage{hyperref}

\usetikzlibrary{knots}
\usepackage{tcolorbox}

%%% Страница
\usepackage{geometry} 
	\geometry{top=20mm, bottom=20mm, left=15mm, right=20mm}

\usepackage{fancyhdr} 
 	\pagestyle{fancy}
 	\renewcommand{\headrulewidth}{1pt}  
    \rhead{\today}
    \lhead{Новохатний Артем, Цыганкова Екатерина, Мифтахов Эльдар}

\usepackage{setspace} 

\usepackage{lastpage} 


\begin{document}

\section{Полином Джонса}

Я читала по \parencite{prasolov-sossinsky} 
(еще про скобку Кауфмана через статсуммы 
интуитивно написано в \parencite{sossinsky}). 
Это более классический подход, как у самого 
Кауфмана \parencite{kauffman}. Хованов 
\parencite{khovanov} почему-то определяет 
скобку Кауфмана иначе, но суть одна -- перебор всевозможных состояний узла. 

Остается непонятным:

\begin{itemize}
    \item Чем Хованову не угодило оригинальное определение скобки Кауфмана?

    Ответ: определение Хованова дает связь с R-матрицами (пока непонятно, какое)
    
    \item Зачем нужно аксиоматическое определение полинома Джонса по \parencite{prasolov-sossinsky}?
\end{itemize}

\section{Полином Хованова по \parencite{bar-natan}}
Разобрали первые две степени в полиноме Хованова трилистника ($q^{-2}\cdot t^0+0\cdot t^1 + q^{-6}t^2+q^{-8}t^3$) (посчитали когомологии явно). Остается:
\begin{enumerate}
    \item всё же доботат градуированные пространства с квантовыми размерностями, чтобы нормально с ними работать, а не гадать
    \item понять, можно ли (если да, то как) определить размерность линейной оболочки суммы двух векторов разной градуировки?
    \item досчитать трилистник
    \item разобраться с доказательством инвариантности полинома Хованова (по \parencite{bar-natan})
    \item разобраться в доказательстве теоремы Рейдмейстера (можно по \parencite{prasolov-sossinsky} стр. 20)
    \item понять, почему полином Хованова -- это расширенный Джонс
\end{enumerate}

\printbibliography

\end{document}