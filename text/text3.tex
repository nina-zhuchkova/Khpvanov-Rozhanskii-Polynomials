\documentclass[12pt,a4paper]{article}
%%% Работа с русским языком
\usepackage{cmap}					% поиск в PDF
\usepackage{mathtext} 				% русские буквы в формулах
\usepackage[T2A]{fontenc}			% кодировка
\usepackage[utf8]{inputenc}			% кодировка исходного текста
\usepackage[main=english,russian]{babel}	% локализация и переносы
\selectlanguage{english}
\usepackage{caption}

%%% Дополнительная работа с математикой
\usepackage{amsmath,amsfonts,amssymb,amsthm,mathtools} % AMS
\usepackage{icomma}
\usepackage{physics}
\usepackage{multicol}
\usepackage{bm}
\usepackage{mathrsfs}
\usepackage{verbatim}

%%% Номера формул
%\mathtoolsset{showonlyrefs=true} 
%\usepackage{leqno} 

%%% Свои команды
\DeclareMathOperator{\sgn}{sgn}

\usepackage{csquotes} 
\usepackage[backend=biber,style=authoryear]{biblatex}


%%% Работа с графикой
\usepackage{graphicx}
\graphicspath{{images/}}  
\setlength\fboxsep{3pt} 
\setlength\fboxrule{1pt} 
\usepackage{wrapfig} 
\usepackage{tikz}
\usepackage{pgfplots}
\usepackage{pgfplotstable}
\usepgfplotslibrary{polar}
\pgfplotsset{compat=1.18} 

%%% Работа с таблицами
\usepackage{array,tabularx,tabulary,booktabs}
\usepackage{longtable}  
\usepackage{multirow} 
\usepackage{soul} 

%%% Теоремы
\theoremstyle{plain} 
\newtheorem{theorem}{Th}[section]
\newtheorem{proposition}[theorem]{Proposition}
\newtheorem{question}{Question}[section]
 
\theoremstyle{definition} 
\newtheorem{corollary}{Corollary}[theorem]
\newtheorem{problem}{Problem}[section]
\newtheorem{definition}{Def}[section]
 
\theoremstyle{remark} 
\newtheorem*{nonum}{Solution}

%%% Программирование
\usepackage{etoolbox} 

%%% Гиперссылки
\usepackage{hyperref}

\usetikzlibrary{knots}
\usepackage{tcolorbox}

%%% Страница
\usepackage{geometry} 
	\geometry{top=20mm, bottom=20mm, left=15mm, right=20mm}

\usepackage{setspace} 

\usepackage{lastpage} 
\usepackage{amssymb}
\usepackage{xcolor}
\DeclareMathOperator{\Ker}{Ker}
\DeclareMathOperator{\qdim}{qdim}
\usepackage[all]{xy}
\usepackage{ dsfont }

\addbibresource{../source.bib}
\begin{document}

\begin{center}
    \Large \textbf{Jones-family knot invariants from $\mathcal{R}$-matrices and cobordism-based Khovanov homology: efficient methods for bipartite knots} \\[1em]
    \small
    Moscow Institute of Physics and Technology \\[0.5em]
    Laboratory of Mathematical and Theoretical Physics
\end{center}

\small
\begin{flushright}
\begin{tabular}{c c}
\textbf{Authors:} & \textbf{Scientific advisors:} \\[0.5em]
Artem Novokhatnii & Elena Lanina \\
Ekaterina Tsygankova & Radomir Stepanov \\
Eldar Miftakhov & \\
\end{tabular}


\vspace{1em}
\normalsize
\today
\end{flushright}
\normalsize
\vspace{2em}


\tableofcontents
\vspace{2em}

\section{Introduction}

\section{Jones polynomial via the Kauffman bracket}

\subsection{Knot invariants}

\begin{tcolorbox}
\begin{theorem}\label{thm:Reidemeister}
(Reidemeister 1927) 

Two diagrams represent isotopic links if and only if one 
can be transformed into the other by a finite number of 
planar isotopies and transformations of the following three types:
\begin{equation}
\Omega_1:  ~~~~~~ 
  \begin{minipage}{0.06\linewidth}
      \resizebox{\linewidth}{!}{\input{../tikz knots/R1_1}}
  \end{minipage} 
  \leftrightarrow
  \begin{minipage}{0.06\linewidth}
      \resizebox{\linewidth}{!}{\begin{tikzpicture}
\begin{knot}[
consider self intersections,
clip width=5,
]
\strand[line width=3pt, black]
(-1, -1) to[out=60,in=180,looseness=1]
(0, 1) to[out=0,in=120,looseness=1] (1, -1);
\end{knot}
\end{tikzpicture}}
  \end{minipage} 
  \leftrightarrow
  \begin{minipage}{0.06\linewidth}
      \resizebox{\linewidth}{!}{\input{../tikz knots/R1_3}}
  \end{minipage} 
\end{equation}

\begin{equation}
\Omega_2:  ~~~~~~ 
  \begin{minipage}{0.06\linewidth}
      \resizebox{\linewidth}{!}{\begin{tikzpicture}
\begin{knot}[
consider self intersections,
clip width=5,
]
\strand[line width=3pt, black]
(-1, -1) to[out=45,in=-90,looseness=1]
(0.5, 0) to[out=90,in=-45,looseness=1] (-1, 1);
\strand[line width=3pt, black]
(1, -1) to[out=135,in=-90,looseness=1]
(-0.5, 0) to[out=90,in=-135,looseness=1] (1, 1);
\end{knot}
\end{tikzpicture}}
  \end{minipage} 
  \leftrightarrow
  \begin{minipage}{0.06\linewidth}
      \resizebox{\linewidth}{!}{\input{../tikz knots/R2_2}}
  \end{minipage} 
\end{equation}

\begin{equation}
\Omega_3:  ~~~~~~ 
  \begin{minipage}{0.06\linewidth}
      \resizebox{\linewidth}{!}{\input{../tikz knots/R3_1}}
  \end{minipage} 
  \leftrightarrow
  \begin{minipage}{0.06\linewidth}
      \resizebox{\linewidth}{!}{\begin{tikzpicture}
\begin{knot}[
consider self intersections,
clip width=5,
]
\strand[line width=3pt, black]
(-1, 0) to[out=-30,in=180,looseness=1] 
(0, -0.5) to[out=0,in=-150,looseness=1] (1, 0);
\strand[line width=3pt, black]
(1, 1) to[out=-160,in=60,looseness=1] (-1, -1);
\strand[line width=3pt, black]
(-1, 1) to[out=-20,in=120,looseness=1] (1, -1);
\end{knot}
\end{tikzpicture}}
  \end{minipage} 
\end{equation}

\end{theorem}
\end{tcolorbox}

Theorem \ref{thm:Reidemeister} shows us that when constructing
a knot invariant we need to check for it's conservation on
Reidemeister moves $\Omega_1, \Omega_2, \Omega_3$.

For the proof of Theorem \ref{thm:Reidemeister} see e.g. \cite{prasolov-sossinsky}.

\subsection{Kauffman bracket}

\begin{tcolorbox}
\begin{definition}
Kauffman bracket axioms

\parencite{khovanov}:

    \begin{equation}
    \left\langle 
    \begin{minipage}{0.06\linewidth}
    \resizebox{\linewidth}{!}{\input{../tikz knots/cross1}}
    \end{minipage} \right\rangle = 
    \left\langle 
    \begin{minipage}{0.06\linewidth}
    \resizebox{\linewidth}{!}{\input{../tikz knots/R2_2}}
    \end{minipage} \right\rangle - q
    \left\langle 
    \begin{minipage}{0.06\linewidth}
    \resizebox{\linewidth}{!}{\input{../tikz knots/R2_2_rot}}
    \end{minipage} \right\rangle
    \label{eq:skein-kaufman}
    \end{equation}

    \begin{equation}
    \langle L_1 \cup L_2\rangle = \langle L_1\rangle \langle L_2 \rangle
    \end{equation}

    \begin{equation}
    \left\langle 
    \begin{minipage}{0.06\linewidth}
    \resizebox{\linewidth}{!}{\begin{tikzpicture}
\begin{knot}
\strand[line width=3pt, black] 
(0,0) circle[radius=1cm];
\end{knot}
\end{tikzpicture}}
    \end{minipage} \right\rangle = q + q^{-1}
    \end{equation}
\label{def:kauf-brack}
\end{definition}
\end{tcolorbox}

We will understand the negative smoothing to be the first resolution in
\eqref{eq:skein-kaufman}, and the positive smoothing to be the second
 one\footnote{It is important to note that if a crossing is reversed,
  the designations of the smoothings will also swap places}. Each 
  crossing in a knot can be smoothed in one of these two ways. 
  By smoothing all $n$ crossings, we obtain a diagram consisting of
   a certain number $p$ of circles. In total, there are $2^n$ 
   such diagrams, which we will call states $s$, Each state corresponds
    to a particular choice of smoothings, containing $\gamma$ positive smoothings.
Thus, from \ref{def:kauf-brack}, we obtain an expression for the 
Kauffman bracket as a state sum:

 \begin{equation}
  \langle L \rangle=\sum_s (-q)^{\gamma} (q + q^{-1})^p
  \label{eq:statsum}
 \end{equation}


Note that we can associate a two-dimensional graded vector space $V$,
to each circle, whose dimension is $\text{dim} V = q + q^{-1}$.
Then, \eqref{eq:statsum} can be rewritten as:

\begin{equation}
  \langle L \rangle=\sum_s (-q)^{\gamma} (\text{dim} V)^p
  \label{eq:statsum-V}
\end{equation}

This discussion gives us a significant insight on how the
Khovanov polynomial will be constructed. We will proceed with this
discussion later in Sec. \ref{sec:Khovanov}.

\begin{question}
Can Kauffman polynomial be derived from Kauffman bracket?
\end{question}

\subsection{Jones polynomial as a compensation of Kauffman bracket}

Let's check how Kauffman bracket behaves under Reidemeister moves:

\begin{equation}
  \left\langle 
    \begin{minipage}{0.06\linewidth}
    \vspace{0pt}
    \resizebox{\linewidth}{!}{\input{../tikz knots/R1_1}}
    \end{minipage} \right\rangle = -q^{2}
    \left\langle 
    \begin{minipage}{0.06\linewidth}
    \vspace{0pt}
    \resizebox{\linewidth}{!}{\begin{tikzpicture}
\begin{knot}[
consider self intersections,
clip width=5,
]
\strand[line width=3pt, black]
(-1, -1) to[out=60,in=180,looseness=1]
(0, 1) to[out=0,in=120,looseness=1] (1, -1);
\end{knot}
\end{tikzpicture}}
    \end{minipage} \right\rangle \ , \
    \left\langle 
    \begin{minipage}{0.06\linewidth}
    \vspace{0pt}
    \resizebox{\linewidth}{!}{\input{../tikz knots/R1_3}}
    \end{minipage} \right\rangle = q^{-1}
    \left\langle 
    \begin{minipage}{0.06\linewidth}
    \vspace{0pt}
    \resizebox{\linewidth}{!}{\begin{tikzpicture}
\begin{knot}[
consider self intersections,
clip width=5,
]
\strand[line width=3pt, black]
(-1, -1) to[out=60,in=180,looseness=1]
(0, 1) to[out=0,in=120,looseness=1] (1, -1);
\end{knot}
\end{tikzpicture}}
    \end{minipage} \right\rangle
\end{equation}

\begin{equation}
  \left\langle 
    \begin{minipage}{0.06\linewidth}
    \vspace{0pt}
    \resizebox{\linewidth}{!}{\begin{tikzpicture}
\begin{knot}[
consider self intersections,
clip width=5,
]
\strand[line width=3pt, black]
(-1, -1) to[out=45,in=-90,looseness=1]
(0.5, 0) to[out=90,in=-45,looseness=1] (-1, 1);
\strand[line width=3pt, black]
(1, -1) to[out=135,in=-90,looseness=1]
(-0.5, 0) to[out=90,in=-135,looseness=1] (1, 1);
\end{knot}
\end{tikzpicture}}
    \end{minipage} \right\rangle = -q
    \left\langle 
    \begin{minipage}{0.06\linewidth}
    \vspace{0pt}
    \resizebox{\linewidth}{!}{\input{../tikz knots/R2_2}}
    \end{minipage} \right\rangle
\end{equation}

\begin{equation}
  \left\langle 
    \begin{minipage}{0.06\linewidth}
    \vspace{0pt}
    \resizebox{\linewidth}{!}{\input{../tikz knots/R3_1}}
    \end{minipage} \right\rangle = 
    \left\langle 
    \begin{minipage}{0.06\linewidth}
    \vspace{0pt}
    \resizebox{\linewidth}{!}{\begin{tikzpicture}
\begin{knot}[
consider self intersections,
clip width=5,
]
\strand[line width=3pt, black]
(-1, 0) to[out=-30,in=180,looseness=1] 
(0, -0.5) to[out=0,in=-150,looseness=1] (1, 0);
\strand[line width=3pt, black]
(1, 1) to[out=-160,in=60,looseness=1] (-1, -1);
\strand[line width=3pt, black]
(-1, 1) to[out=-20,in=120,looseness=1] (1, -1);
\end{knot}
\end{tikzpicture}}
    \end{minipage} \right\rangle
\end{equation}

Hence, Kauffman bracket is not a knot invariant itself. Thus, we
introduce normalization coefficient that will fix invariance. In
oreder to do so we define positive and negative crossings on
oriented knot diagram:

\begin{equation}
  +: \ \
  \begin{minipage}{0.06\linewidth}
    \vspace{0pt}
    \resizebox{\linewidth}{!}{\input{../tikz knots/cross1+.tex}}
    \end{minipage} \ , \ \ -: \ \
    \begin{minipage}{0.06\linewidth}
    \vspace{0pt}
    \resizebox{\linewidth}{!}{\input{../tikz knots/cross2-.tex}}
    \end{minipage}
\end{equation}

The amount of positive and negative crossings on a knot diagram is $n_+$ и $n_-$ 
respectively. Let us multiply the result of Kauffman bracket by the
factor $(-1)^{a n_+ + b n_-}q^{c n_+ + d n_-}$ in order to achieve
invariance. This approach yields an invariant polynomial:

\begin{tcolorbox}
\begin{equation}
  \displaystyle
  J(q, L)=(-1)^{n_-}q^{n_+ - 2 n_-}\frac{\langle L \rangle}{\langle
    \begin{minipage}{0.03\linewidth}
    \vspace{0pt}
    \resizebox{\linewidth}{!}{\begin{tikzpicture}
\begin{knot}
\strand[line width=3pt, black] 
(0,0) circle[radius=1cm];
\end{knot}
\end{tikzpicture}}
    \end{minipage}
  \rangle}
  \label{eq:jones}
\end{equation}
\end{tcolorbox}

By Jones polynomial we will understand 
\eqref{eq:jones} \footnote{Usually Jones polynomial $\tilde{J}$
is defined as $\tilde{J}(t, L) := 
J(-\sqrt{t}, L)$}.

\begin{question}
  Jones is normalized on the unknot. How to normalize Khovanov homology?
\end{question}

\subsection{Examples}
\label{subsec:Jones_ex}
\begin{multicols}{2}
\textbf{Ex. 1} Oppositely oriented Hopf links:
\begin{equation}
  J\left(
  \begin{minipage}{0.06\linewidth}
    \vspace{0pt}
    \resizebox{\linewidth}{!}{\input{../tikz knots/hopf1.tex}}
    \end{minipage} \right) = q + q^5 
\end{equation}

\begin{equation}
  J\left(
  \begin{minipage}{0.06\linewidth}
    \vspace{0pt}
    \resizebox{\linewidth}{!}{\begin{tikzpicture}
  \begin{knot}[
    clip width=5,
    flip crossing/.list={1}
  ]
    % первая окружность по дугам
    \strand[line width=3pt,->,black] 
      (0,0) arc[start angle=0,end angle=360,radius=1];
    
    % вторая окружность по дугам
    \strand[line width=3pt,<-,black] 
      (-1,0) arc[start angle=0,end angle=360,radius=-1];
  \end{knot}
\end{tikzpicture}}
    \end{minipage} \right) = q^{-1} + q^{-5}
\end{equation}

\textbf{Ex. 2} Positive and negative trefoils:

\begin{equation}
  J\left(
  \begin{minipage}{0.06\linewidth}
    \vspace{0pt}
    \resizebox{\linewidth}{!}{\begin{tikzpicture}
\begin{knot}[
consider self intersections,
flip crossing/.list={1, 3}, %переворот пересечений
clip width=5,
]
\strand[line width=3pt, black]
(0, 1) to[out=180,in=-120,looseness=2]
%декартовы координаты
(-30:1) to[out=60,in=120,looseness=2]
%полярные координаты
(210:1) to[out=-60,in=0,looseness=2] (90:1);
\end{knot}
\end{tikzpicture}}
    \end{minipage} \right) = q^{2} + q^{6} - q^{8}
\end{equation}

\begin{equation}
  J\left(
  \begin{minipage}{0.06\linewidth}
    \vspace{0pt}
    \resizebox{\linewidth}{!}{\begin{tikzpicture}
\begin{knot}[
consider self intersections,
flip crossing=2, %переворот пересечений
clip width=10,
]
\strand[line width=3pt, black]
(0, 2) to[out=180,in=-120,looseness=2]
%декартовы координаты
(-30:2) to[out=60,in=120,looseness=2]
%полярные координаты
(210:2) to[out=-60,in=0,looseness=2] (90:2);
\end{knot}
\end{tikzpicture}}
    \end{minipage} \right) = q^{-2} + q^{-6} - q^{-8}
\end{equation}
\end{multicols}
A pattern can be observed in the answers regarding the powers of $q$.
This fact turns out to be general for the Jones polynomial:

\begin{tcolorbox}
\begin{theorem}[On the Jones polynomial of a mirror knot]
\begin{equation}
J(q, L) = J(q^{-1}, \bar{L}) \text{, where $\bar{L}$ is the mirror knot}
\end{equation}
\label{thm:mirror}
\end{theorem}
\end{tcolorbox}

The proof of \ref{thm:mirror} is quite straightforward, using the Kauffman bracket defined as a statistical sum.

\subsection{Skein-relations}

Using \eqref{eq:skein-kaufman} for the opposite crossings we obtain:

\begin{equation}
\left\{
\begin{array}{l}
    \left\langle 
    \begin{minipage}{0.06\linewidth}
    \resizebox{\linewidth}{!}{\input{../tikz knots/cross1}}
    \end{minipage} \right\rangle = 
    \left\langle 
    \begin{minipage}{0.06\linewidth}
    \resizebox{\linewidth}{!}{\input{../tikz knots/R2_2}}
    \end{minipage} \right\rangle - q
    \left\langle 
    \begin{minipage}{0.06\linewidth}
    \resizebox{\linewidth}{!}{\input{../tikz knots/R2_2_rot}}
    \end{minipage} \right\rangle  ~~ |\cdot q^{-1}\\ \\
    
    \left\langle 
    \begin{minipage}{0.06\linewidth}
    \resizebox{\linewidth}{!}{\begin{tikzpicture}
\begin{knot}[
consider self intersections,
flip crossing = 1,
clip width=5,
]
\strand[line width=3pt, black]
(-1, -1) to[out=45,in=-135,looseness=1] (1, 1);
\strand[line width=3pt, black]
(-1, 1) to[out=-45,in=135,looseness=1] (1, -1);
\end{knot}
\end{tikzpicture}}
    \end{minipage} \right\rangle = 
    \left\langle 
    \begin{minipage}{0.06\linewidth}
    \resizebox{\linewidth}{!}{\input{../tikz knots/R2_2_rot}}
    \end{minipage} \right\rangle - q
    \left\langle 
    \begin{minipage}{0.06\linewidth}
    \resizebox{\linewidth}{!}{\input{../tikz knots/R2_2}}
    \end{minipage} \right\rangle
\end{array}
\right.
\;+\;
\end{equation}

\begin{equation}
  q^{-1} \left\langle 
    \begin{minipage}{0.06\linewidth}
    \resizebox{\linewidth}{!}{\input{../tikz knots/cross1}}
    \end{minipage} \right\rangle +
  \left\langle 
    \begin{minipage}{0.06\linewidth}
    \resizebox{\linewidth}{!}{\begin{tikzpicture}
\begin{knot}[
consider self intersections,
flip crossing = 1,
clip width=5,
]
\strand[line width=3pt, black]
(-1, -1) to[out=45,in=-135,looseness=1] (1, 1);
\strand[line width=3pt, black]
(-1, 1) to[out=-45,in=135,looseness=1] (1, -1);
\end{knot}
\end{tikzpicture}}
    \end{minipage} \right\rangle =
    (q^{-1} - q) \left\langle 
    \begin{minipage}{0.06\linewidth}
    \resizebox{\linewidth}{!}{\input{../tikz knots/R2_2}}
    \end{minipage} \right\rangle
    \label{eq:skein-kaufman-cross}
\end{equation}

In the same manner we can derive \eqref{eq:skein-kaufman-cross} 
from \eqref{eq:skein-kaufman}. Hence, this properties are equivalent. 
However \eqref{eq:skein-kaufman-cross}, 
in contrast to \eqref{eq:skein-kaufman}, is applied for oriented knots
 and with respect of normalization
\eqref{eq:jones} yields:

\begin{tcolorbox}
\begin{equation}
q^{-2} J\left (
  \begin{minipage}{0.06\linewidth}
    \resizebox{\linewidth}{!}{\input{../tikz knots/cross1+.tex}}
    \end{minipage}
\right ) - q^2 J\left (
  \begin{minipage}{0.06\linewidth}
    \resizebox{\linewidth}{!}{\input{../tikz knots/cross2-.tex}}
    \end{minipage}
\right ) = (q^{-1}-q) J\left (
  \begin{minipage}{0.06\linewidth}
    \resizebox{\linewidth}{!}{\begin{tikzpicture}
\begin{knot}[
consider self intersections,
clip width=5,
]
\strand[line width=3pt, ->, black]
(-1, -1) to[out=60,in=-90,looseness=1]
(-0.5, 0) to[out=90,in=-60,looseness=1] (-1, 1);
\strand[line width=3pt, ->, black]
(1, -1) to[out=120,in=-90,looseness=1]
(0.5, 0) to[out=90,in=-120,looseness=1] (1, 1);
\end{knot}
\end{tikzpicture}}
    \end{minipage}
\right )
\label{eq:skein}
\end{equation}
\end{tcolorbox}

Relation \eqref{eq:skein} combined with requirment $(J(L) = 1 ~~ \forall 
L \sim \begin{minipage}{0.03\linewidth}
    \vspace{0pt}
    \resizebox{\linewidth}{!}{\begin{tikzpicture}
\begin{knot}
\strand[line width=3pt, black] 
(0,0) circle[radius=1cm];
\end{knot}
\end{tikzpicture}}
    \end{minipage})$ 
    uniquely determine the Jones polynomial
     (see p.46 \cite{prasolov-sossinsky}).

\begin{question}
  (On the minimal additional requirements for the skein relation) By replacing $(-1)^{n_-}$ with $(-1)^{n_+ + 1}$, one can easily construct a polynomial that is not invariant under $\Omega_2$, but, despite this, is invariant under $\Omega_1$ and $\Omega_3$ and satisfies \eqref{eq:skein}. This confirms that $\Omega_2$-invariance does not follow from the skein relations and must be imposed artificially. Are $\Omega_1$- and $\Omega_3$-invariances implied by \eqref{eq:skein}, or do they also need to be added “from the outside”?
\end{question}

\begin{question}
  Add other powers to the left side of the skein relation. Will it still be an invariant?
\end{question}


\section{Khovanov homology as a categorification of the Jones polynomial}
\label{sec:Khovanov}

\subsection{Khovanov complex construction}
The statesum form (equation \eqref{eq:statsum}) of the Kauffman bracket allows the following interpretation \parencite{bar-natan}. Let us consider each circle in a completely smoothed diagram $\alpha \in \{0, 1\}^\chi$ as a graded vector space $V$ with grading $\{+, -\}$.  If $\alpha$ consists of $n$ circles, we associate it with $V^{\otimes n}$. Then we group complete smoothings according to the number of 1-resolutions $\gamma$ and consider these groups as a direct sum of vector spaces $V_\gamma = \bigoplus\limits_{\alpha(\gamma)} V^{\otimes n(\alpha)}$. 

The statesum takes the form
\begin{equation}
\langle L \rangle = \sum\limits_{\gamma = 0}^n (-1)^\gamma ,\mathrm{qdim}\bigl(V_\gamma {\gamma}\bigr),
\end{equation}
where $\gamma$ is the common degree shift. This suggests that if the constructed spaces $C_\gamma := V_\gamma \{\gamma\}$ were connected by differentials within a chain complex, then the state sum might be rewritten in homological notation:
\begin{equation}
\langle L \rangle = \sum\limits_{\gamma = 0}^n (-1)^\gamma ,\mathrm{qdim}(H_\gamma).
\end{equation}

The idea now is twofold.
First, to modify the spaces in such a way that the compensating factor is taken into account, so that the resulting alternating sum turns into the exact Jones polynomial, which is a complete isotopy invariant.
Second, to construct differentials between $C_\gamma$ such that not only the alternating sum of $\mathrm{qdim}(H_\gamma)$ is an isotopy invariant, but each homology group itself.

Such a construction was first proposed by \parencite{khovanov} and then simply presented in \parencite{bar-natan}. The second interpretation is given below.


[to be explained]

\subsubsection{Product and coproduct operators}
The only possible structure of $\Delta$ and $m$ can be concluded from "grading-conservation" reasons. $m$ and $\Delta$ act on the following spaces:

\begin{equation}
m: V^{\otimes 2} = \langle v_-\otimes v_- \{-2\}, v_- \otimes v_+ \{0\}, v_+ \otimes v_- \{0\}, v_+ \otimes v_+ \{2\}\rangle \to V\{1\} = \langle v_-\{0\}, v_+\{2\}\rangle
\end{equation}
\begin{equation}
\Delta: V = \langle v_-\{-1\}, v_+\{1\}\rangle \to V^{\otimes 2}\{1\} = \langle v_-\otimes v_- \{-1\}, v_- \otimes v_+ \{1\}, v_+ \otimes v_- \{1\}, v_+ \otimes v_+ \{3\}\rangle
\end{equation}

Hence, they are given by:
\begin{equation}
m = 
\begin{pmatrix}
0 & a & b & 0 \\
0 & 0 & 0 & c
\end{pmatrix}, 
\quad
\Delta = 
\begin{pmatrix}
f & 0 \\
0 & d \\
0 & e \\
0 & 0
\end{pmatrix}
\end{equation}

Now from associativity \eqref{eq:comm_1} and coassociativity \eqref{eq:comm_3} we obtain that $a = b = c$ and $d = e = f$ so, up to normalization, $m = 
\begin{pmatrix}
0 & 1 & 1 & 0 \\
0 & 0 & 0 & 1
\end{pmatrix}
$ and $\Delta = 
\begin{pmatrix}
1 & 0 \\
0 & 1 \\
0 & 1 \\
0 & 0
\end{pmatrix}
$.

\begin{figure}[h]
\includegraphics[width=0.9\textwidth]{../img/bar-natan-2.png}
\caption{Khovanof complex for trefoil \parencite{bar-natan}}
\label{fig:trefoil_complex}
\end{figure}

\subsection{Example: positive Hopf link}

\begin{wrapfigure}{r}{0.2\textwidth}
  \begin{center}
    \begin{minipage}{\linewidth}
      \resizebox{\linewidth}{!}{\begin{tikzpicture}
  \begin{knot}[
    clip width=5,
    flip crossing/.list={1}
  ]
    % первая окружность по дугам
    \strand[line width=3pt,->,black] 
      (0,0) arc[start angle=0,end angle=360,radius=1];
    
    % вторая окружность по дугам
    \strand[line width=3pt,<-,black] 
      (-1,0) arc[start angle=0,end angle=360,radius=-1];
  \end{knot}
\end{tikzpicture}}
  \end{minipage} 
    $n_{+} = 2$

  $n_{-} = 0$
  \end{center}
\end{wrapfigure}

\[
\mathcal{H}^0 = \Ker d_{01} = \langle v_- \otimes v_-, v_- \times v_+ - v_+ \times v_-\rangle \{2\} ~~ \Rightarrow ~~ \qdim\mathcal{H}^0 = 1 + q^2
\]

\[
\mathcal{H}^1 = \Ker d_{12}/\Im d_{01} = \frac{\langle v_{+(1)} + v_{+(2)}, v_{-(1)} + v_{-(2)}\rangle}{\langle v_{+(1)} + v_{+(2)}, v_{-(1)} + v_{-(2)}\rangle}\{3\} ~~ \Rightarrow ~~ \qdim\mathcal{H}^1 = 0
\]

\[
\mathcal{H}^2 = V/\Im d_{12} = (V/\langle v_- \otimes v_-, v_- \times v_+ + v_+ \times v_-\rangle)\{4\} ~~ \Rightarrow ~~ \qdim\mathcal{H}^2 =  q^4 + q^6
\]

\[\boxed{Kh(\begin{minipage}{0.06\linewidth}
      \resizebox{\linewidth}{!}{\begin{tikzpicture}
  \begin{knot}[
    clip width=5,
    flip crossing/.list={1}
  ]
    % первая окружность по дугам
    \strand[line width=3pt,->,black] 
      (0,0) arc[start angle=0,end angle=360,radius=1];
    
    % вторая окружность по дугам
    \strand[line width=3pt,<-,black] 
      (-1,0) arc[start angle=0,end angle=360,radius=-1];
  \end{knot}
\end{tikzpicture}}
  \end{minipage} ) = 1 + q^2 + t^2 q^4 + t^2 q^6}
\]

\begin{wrapfigure}{r}{0pt}
\end{wrapfigure}

\begin{figure}[h]
    \begin{center}
    \begin{tikzpicture}
    \node at (-0.5, 0) {0};
    \draw[->] (0,0) -- (1,0) node[midway, above] {$d_{00}$};

    \node[inner sep=0pt] at (2.5,0) {
    \resizebox{2cm}{!}{
        \begin{tikzpicture}
            \begin{knot}[clip width=5]
                \strand[line width=2pt, black]
                    (-0.2, 0.7) to[out=90,in=90,looseness=1]
                    (-1.5, 0) to[out=-90,in=-90,looseness=1]
                    (-0.2, -0.7) to[out=90,in=-90,looseness=1]
                    (-0.7, 0) to[out=90,in=-90,looseness=1] (-0.2, 0.7);

                \strand[line width=2pt, black]
                    (0.2, 0.7) to[out=90,in=90,looseness=1]
                    (1.5, 0) to[out=-90,in=-90,looseness=1]
                    (0.2, -0.7) to[out=90,in=-90,looseness=1]
                    (0.7, 0) to[out=90,in=-90,looseness=1] (0.2, 0.7);
            \end{knot}
        \end{tikzpicture}
    }
    };

    \draw[->] (4,0.5) -- (5,1.5) 
    node[pos=0.25, right = 3pt] {$d_{\bigstar 0}$}
    node[pos=0.5, left = 3pt] {$m$};
    \draw[->] (4,-0.5) -- (5,-1.5) 
    node[pos=0.25, right = 3pt] {$d_{0 \bigstar}$}
    node[pos=0.5, left = 3pt] {$m$};

    \node[inner sep=0pt] at (6.5, 1.5) {
    \resizebox{2cm}{!}{
        \begin{tikzpicture}
            \begin{knot}[clip width=5]
                \strand[line width=2pt, black]
                    (0, 0.8) to[out=180,in=0,looseness=1]
                    (-0.7, 1) to[out=180,in=180,looseness=1]
                    (-0.7, -1) to[out=0,in=-90,looseness=1]
                    (-0.2, -0.7) to[out=90,in=180,looseness=1]
                    (0, 0.4) to[out=0,in=90,looseness=1]
                    (0.2, -0.7) to[out=-90,in=180,looseness=1]
                    (0.7, -1) to[out=0,in=0,looseness=1]
                    (0.7, 1) to[out=180,in=0,looseness=1] (0, 0.8);
            \end{knot}
        \end{tikzpicture}
    }
    };

    \node[inner sep=0pt] at (6.5, -1.5) {
    \resizebox{2cm}{!}{
        \begin{tikzpicture}
            \begin{knot}[clip width=5]
                \strand[line width=2pt, black]
                    (0, -0.8) to[out=180,in=0,looseness=1]
                    (-0.7, -1) to[out=180,in=180,looseness=1]
                    (-0.7, 1) to[out=0,in=90,looseness=1]
                    (-0.2, 0.7) to[out=-90,in=180,looseness=1]
                    (0, -0.4) to[out=0,in=-90,looseness=1]
                    (0.2, 0.7) to[out=90,in=180,looseness=1]
                    (0.7, 1) to[out=0,in=0,looseness=1]
                    (0.7, -1) to[out=180,in=0,looseness=1] (0, -0.8);
            \end{knot}
        \end{tikzpicture}
    }
    };

    \draw[->] (8,1.5) -- (9,0.5) 
    node[pos=0.75, left = 3pt] {$d_{1 \bigstar}$}
    node[pos=0.25, right = 3pt] {$-\Delta$};
    \draw[->] (8,-1.5) -- (9,-0.5) 
    node[pos=0.75, left = 3pt] {$d_{\bigstar 1}$}
    node[pos=0.25, right = 3pt] {$\Delta$};

    \node[inner sep=0pt] at (10.5, 0) {
    \resizebox{2cm}{!}{
        \begin{tikzpicture}
            \begin{knot}[clip width=5]
                \strand[line width=2pt, black]
                    (0, 0.8) to[out=180,in=0,looseness=1]
                    (-0.7, 1) to[out=180,in=180,looseness=1]
                    (-0.7, -1) to[out=0,in=180,looseness=1]
                    (0, -0.8) to[out=0,in=180,looseness=1]
                    (0.7, -1) to[out=0,in=0,looseness=1]
                    (0.7, 1) to[out=180,in=0,looseness=1](0, 0.8);
                \strand[line width=2pt, black] 
                    (0,0) circle[radius=0.5cm];
            \end{knot}
        \end{tikzpicture}
    }
    };

    \draw[->] (12,0) -- (13,0) node[midway, above] {$d_{11}$};
    \node at (13.5, 0) {0};


    \draw[blue] (3.7,2) rectangle (5.3,-2);
    \draw[->, blue] (4,-2.7) -- (5,-2.7) node[midway, above, blue]{$d_{01}$};

    \draw[blue] (7.7,2) rectangle (9.3,-2);
    \draw[->, blue] (8,-2.7) -- (9,-2.7) node[midway, above, blue]{$d_{12}$};

    \node at (2.5, -2.7) {\scriptsize $V_0 = V^{\otimes 2} \{0\} \textcolor{red}{\{2\}}$};
    \node at (6.5, -2.7) {\scriptsize $V_1 = V^{\oplus 2} \{1\} \textcolor{red}{\{2\}}$};
    \node at (10.5, -2.7) {\scriptsize $V_2 = V^{\otimes 2} \{2\} \textcolor{red}{\{2\}}$};

\end{tikzpicture}
    \caption{Positive Hopf link Khovanov complex. $\textcolor{red}{\{n_{+}-2n_{-}\} = \{2\}}$ --- common degree shift}
    \label{fig:hopf-diagram}
    \end{center}
\end{figure}


\section{Cobordism-based approach to Khovanov homology \parencite{bar-natan-cob}}
[to be explained]

\section{$\mathcal{R}$-matrix construction of Jones and HOMFLY polynomials}

Along with taking the statistical sum, there is another approach to constructing knot invariants. It relies on the fact that any knot can be represented as a closed braid (\ref{th:braid}). This suggests searching for an invariant as a linear braid-group representation or, if we want to get a number or polynomial rather than a matrix, as the trace of such a representation.\footnote{The given approach appears naturally in Chern-Simons theory with gauge group $G$: observables turn out to be ribbon invariants and are calculated as a convolution of group-induced operators $\mathcal{R}$ and $\mathcal{Q}$ associated with special points of a knot diagram. However, this subject goes far beyond the scope of our discussion.}

In this section, we provide an off-the-shelf construction for the desired invariant. It is built on the basis of the quantized universal enveloping algebra $U_q$ for a simple Lie algebra $\mathfrak{g}$. The universal $\mathcal{R}$-matrix -- the operator in $U_q \otimes U_q$ -- then provides the braid group representation, while the quantum trace, taken via the $\mathcal{Q}$-operator, gives the knot invariant up to a compensating factor. Further intuition behind this comprehensive structure is set out in \parencite{anokhina}.


\subsection{Links as a braid closure}
\begin{wrapfigure}{r}{0.15\textwidth}
    \centering
  \includegraphics[width = 0.9\linewidth]{../img/braid-closure.png}
  \caption{Eight-knot as a braid closure \small{\parencite{ohtsuki}}}
  \label{fig:braid-closure}
\end{wrapfigure}

The closure of a braid is the link obtained from the braid by connecting upper ends and lower ends respectively as shown in Fig. \ref{fig:braid-closure}. The following theorem
assures us that such an expression always exists.
\begin{tcolorbox}
\begin{theorem}\label{thm:Alexander}
   (Alexander). Any (oriented) link is isotopic to the closure of some
braid (with downward orientation).  
\end{theorem}
\end{tcolorbox} 

However, as our intuition suggests, knots and braids have similar but not identical natures, so a single knot can be represented by a set of different braids. Nevertheless, this equivalence relation can be reformulated in terms of braid algebra operations, as stated by the following theorem.

\clearpage

\begin{tcolorbox}
\begin{theorem}\label{thm:Alexander}
   (Markov).  Let $b$ and $b'$ be two braids, and $L$ and $L'$ their closures.
Then, $L$ is isotopic to $L'$ as oriented links if and only if $b$ is related to $b'$ by a
sequence of the following MI and MII moves.

\begin{center}
\begin{tabular}{c c}
\textbf{MI:} & \textbf{MII:} \\
$ab \longleftrightarrow ba$ & $b \sigma_n \longleftrightarrow b \longleftrightarrow b \sigma_n^{-1}$ \\[2mm]  
\includegraphics[height=3cm]{../img/MI.png} &
\includegraphics[height=3cm]{../img/MII.png}
\end{tabular}
\end{center}

\end{theorem}

\end{tcolorbox} 


Further action is to propose a linear braid representation whose trace is invariant under MI and MII moves. For this purpose, with some reservations, one can utilize the universal quantum $\mathcal{R}$-matrix and the quantum trace, which will be discussed below.

\subsection{Quantized universal enveloping algebra $U_q(\mathfrak{g})$: $\mathcal{R}$ and $\mathcal{Q}$ operators}

Let $\mathfrak{g}$ be a simple Lie algebra of rank $r$ with root system $\Phi$.
Denote by $\Phi^{+}$ the subset of positive roots and let $n = |\Phi^{+}|$ be the number of positive roots.
Fix a set of simple roots $\{ \alpha_1, \ldots, \alpha_r \} = \Psi$.
The Cartan matrix of the algebra is defined by
$
a_{\alpha\beta}
=
\frac{2(\alpha,\beta)}{(\alpha,\alpha)}.
$, where $(\,\cdot\,,\,\cdot\,)$ denotes the Killing form restricted to the Cartan subalgebra.

\begin{tcolorbox}
\begin{definition}
Let $h$ be an indeterminate number. Then the \ul{\textit{quantized universal enveloping algebra}
$U_q(\mathfrak{g})$} is the unital algebra over $\mathbb{C}[q]$, where $q = e^{h}$,
generated by $3r$ elements $\{ h_\alpha, E_\alpha, F_\alpha \mid \alpha \in \Psi \}$, subject to the relations
\small

\begin{minipage}[t]{0.25\textwidth}
  \begin{align}
[ h_\alpha, h_\beta ] &= 0, \\
[ h_\alpha, E_\beta ] &= (\alpha,\beta)\, E_\beta, \\
[ h_\alpha, F_\beta ] &= (\alpha,\beta)\, F_\beta,
\end{align}
\begin{equation}
[ E_\alpha, F_\beta ]
= \delta_{\alpha\beta}\frac{q_\alpha^{h_\alpha} - q_\alpha^{-h_\alpha}}{q_\alpha - q_\alpha^{-1}}
\end{equation}
\end{minipage}
\hfill
\begin{minipage}[t]{0.65\textwidth}
\begin{equation}
\sum_{m=0}^{1-a_{\alpha\beta}}(-1)^m\begin{bmatrix}1-a_{\alpha\beta} \\ m
\end{bmatrix}_{q_\alpha} E_\alpha^{\,1-a_{\alpha\beta}-m} E_\beta E_\alpha^{\,m}= 0,
\qquad \alpha \neq \beta
\end{equation}
\begin{equation}
\sum_{m=0}^{1-a_{\alpha\beta}}(-1)^m\begin{bmatrix}1-a_{\alpha\beta} \\ m\end{bmatrix}_{q_\alpha}
F_\alpha^{\,1-a_{\alpha\beta}-m} F_\beta F_\alpha^{\,m}= 0,
\qquad \alpha \neq \beta.
\end{equation}
\end{minipage}

\normalsize
Here
\begin{equation}
\begin{bmatrix}
n \\ m
\end{bmatrix}_q
=
\frac{[n]_q!}{[m]_q!\,[n-m]_q!},
\qquad
[n]_q! = \prod_{k=1}^n [k]_q,
\qquad
[n]_q = \frac{q^n - q^{-n}}{q - q^{-1}},
\qquad
q_\alpha = q^{(\alpha,\alpha)/2}.
\end{equation}
\end{definition}
\end{tcolorbox}

Note that $U_q({\mathfrak{g}})$ turns into ordinary universal enveloping algebra $U(\mathfrak{g})$ at $h \to 0$.

Equipped with the quantized coproduct $\Delta$ (for more details see \parencite{isaev}), $U_q(\mathfrak{g})$ becomes a Hopf algebra. It turns out that there exists a universal $\mathcal{R}$-matrix — an invertible linear operator in $U_q(\mathfrak{g})^{\otimes 2}$ that commutes with the tensor square $\Delta U_q(\mathfrak{g})$ of the whole algebra and, most importantly, satisfies the Yang-Baxter equation:

\begin{equation}
    (\mathcal{R} \otimes \mathbb{I})(\mathbb{I} \otimes \mathcal{R})
    (\mathcal{R} \otimes \mathbb{I}) = (\mathbb{I} \otimes \mathcal{R})
    (\mathcal{R} \otimes \mathbb{I}) (\mathbb{I} \otimes \mathcal{R})
    \label{eq:YB}
\end{equation}

An explicit expression for universal $\mathcal{R}$-matrix is:

\begin{equation}
    \boxed{
R = \hat{P}\, q^{\sum_{\alpha \in \Psi} h_\alpha \otimes h_{\alpha^\vee}}
\overrightarrow{\prod_{\beta \in \Phi^+_{(n)}}}
\exp_{q_\beta} \Biggl(
(q_\beta - q_\beta^{-1}) E_\beta \otimes F_\beta
\Biggr)}
\label{eq:R}
\end{equation}

Here $\hat{P}$ is the permutation operator.
$\{ \alpha_1^\vee, \ldots, \alpha_r^\vee \mid \alpha \in \Psi \}$ is the set of elements dual to the simple roots, defined by $(\alpha^\vee, \beta) = \delta_{\alpha\beta}$. For an arbitrary element $\gamma$ from the Cartan subalgebra, the corresponding element of the algebra is defined by $h_\gamma=\sum_{\alpha \in \Psi}
h_\alpha\, (\alpha^\vee, \gamma)$. The product is taken in normal ordering $\Phi^+_{(n)}$, i.e., each composite root $\alpha + \beta \in \Phi^+$, where $\alpha, \beta \in \Phi^+$, is placed in the ordering between $\alpha$ and $\beta$. 

\begin{question}
    Are dual elements well defined?
\end{question}

As was already mentioned, one can in addition define a quantum trace using $\mathcal{C}$ and $\mathcal{D}$ "closure" operators, such that:

\begin{equation}
\text{Tr}_2(\mathcal{RC}) = \begin{minipage}{0.15\linewidth}\resizebox{\linewidth}{!}{\begin{tikzpicture}
\begin{knot}[
consider self intersections,
clip width=3,
flip crossing = 1
]
\strand[line width=2pt, black, ->]
(-0.5, 0.5) to[out=-45,in=135,looseness=1](0.5, -0.5);
\strand[line width=2pt, black, ->]
(0.5, -0.5) to[out=-45,in=180,looseness=1]
(0.8, -0.7) to[out=0,in=0,looseness=1]
(0.8, 0.7) to[out=180,in=45,looseness=1]
(0.5, 0.5) to[out=-135,in=45,looseness=1](-0.5, -0.5);
\end{knot}
\node at (1.5, 0) {$\mathcal{C}$};
\end{tikzpicture}}\end{minipage} = \begin{minipage}{0.045\linewidth}\resizebox{\linewidth}{!}{\begin{tikzpicture}
\begin{knot}[
consider self intersections,
clip width=3,
]
\strand[line width=2pt, black, ->]
(0, 0.5) to[out=-90,in=90,looseness=1](0, -0.5);

\end{knot}
\node at (0.5, 0) {$\mathds{1}$};
\end{tikzpicture}}\end{minipage} , ~~~ \text{Tr}_2(\mathcal{R}^{-1}\mathcal{D}) = \begin{minipage}{0.15\linewidth}\resizebox{\linewidth}{!}{\input{../tikz knots/negative_closing.tex}}\end{minipage} = \begin{minipage}{0.045\linewidth}\resizebox{\linewidth}{!}{\begin{tikzpicture}
\begin{knot}[
consider self intersections,
clip width=3,
]
\strand[line width=2pt, black, ->]
(0, 0.5) to[out=-90,in=90,looseness=1](0, -0.5);

\end{knot}
\node at (0.5, 0) {$\mathds{1}$};
\end{tikzpicture}}\end{minipage}
\label{eq:tr2}
\end{equation}

An explicit expressions for closure operators differ just by prefactor $q^{\pm\Omega_2}$ (meaning of $\Omega_2$ is discussed below):
\begin{equation}
    \mathcal{C} = \mathds{1}\otimes\mathcal{Q} \cdot q^{-\Omega_2}, ~~~ \mathcal{D} = \mathds{1}\otimes\mathcal{Q} \cdot q^{\Omega_2} ~~~~  \text{, where}~~\boxed{\mathcal{Q} = q^{\sum\limits_{\alpha \in \Phi^{+}} h_\alpha}}
    \label{eq:Q}
\end{equation}


\begin{question}
    In a seminal article \parencite{morozov-chern} the $\Omega_2$ operator is defined as
    \begin{equation}\Omega_2 \coloneqq
\sum_{\alpha \in \Delta}
h_\alpha^\vee h_\alpha
+
\sum_{\alpha \in \Phi^{+}}
\bigl( E_\alpha F_\alpha + F_\alpha E_\alpha \bigr)
\label{eq:O}
\end{equation}
, which is the standard quadratic Casimir for $\mathfrak{sl}_N$, where all roots have the same length. Which definition of $\Omega_2$ should be considered in the case of different root lengths: the classical Casimir or the one proposed in \eqref{eq:O}?
\end{question}

\subsection{$\mathcal{R}$-matrix braid group representation and knot invariants}

Now we can define the quantum $\mathfrak{g}$-representation of a braid on $n$ threads by assigning the operators $\mathds{1}^{\otimes k - 1} \otimes \mathcal{R} \otimes \mathds{1}^{\otimes n - k + 1}$ and 
$\mathds{1}^{\otimes k - 1} \otimes \mathcal{R}^{-1} \otimes \mathds{1}^{\otimes n - k + 1}$ to the generators
$\sigma_k = \begin{minipage}{0.15\linewidth}\resizebox{\linewidth}{!}{\input{../tikz knots/perm_p.tex}}\end{minipage}$ and 
$\sigma_k^{-1} = \begin{minipage}{0.15\linewidth}\resizebox{\linewidth}{!}{\begin{tikzpicture}
\begin{knot}[
consider self intersections,
clip width=5,
]
\strand[line width=3pt, black, ->]
(-3, 0.5) to[out=-90,in=90,looseness=1]
(-3, -0.5);
\strand[line width=3pt, black, ->]
(-1.5, 0.5) to[out=-90,in=90,looseness=1]
(-1.5, -0.5);
\strand[line width=3pt, black, ->]
(-0.5, 0.5) to[out=-90,in=90,looseness=1]
(0.5, -0.5);
\strand[line width=3pt, black, ->]
(0.5, 0.5) to[out=-90,in=90,looseness=1]
(-0.5, -0.5);
\strand[line width=3pt, black, ->]
(3, 0.5) to[out=-90,in=90,looseness=1]
(3, -0.5);
\strand[line width=3pt, black, ->]
(1.5, 0.5) to[out=-90,in=90,looseness=1]
(1.5, -0.5);
\end{knot}
\node at (-2.25, 0) {$\cdots$};
\node at (2.25, 0) {$\cdots$};
\node at (-3, -0.8) {1};
\node at (-1.5, -0.8) {$k - 1$};
\node at (-0.5, -0.8) {$k$};
\node at (0.5, -0.8) {$k+1$};
\node at (1.5, -0.8) {$k+2$};
\node at (3, -0.8) {$n$};
\end{tikzpicture}}\end{minipage}$ of the braid group, respectively. The YB equation \eqref{eq:YB} ensures close commutation relation for braids ($\sigma_k\sigma_{k + 1}\sigma_k = \sigma_{k + 1}\sigma_k\sigma_{k + 1}$).

\begin{question}
Is the quantum $\mathfrak{g}$-representation exact? Is the braid group even linear, i.e., does it have an exact linear representation at all?
\end{question}

The last step in knot invariant construction is to correctly convolute the resulting braid representation. The closure operator $\mathcal{Q}$ immediately provides such a trace.

\begin{tcolorbox}
\begin{proposition}
    Let knot $K$ be represented as a closure $\hat{b}$ of some element of a braid group $b \in B_n$. Then for gauge group $G$ and representation $\rho: \mathfrak{g} \to \text{End}(V)$ polynomial given by the following formula is isotopy invariant:
    \begin{equation}
    I_\rho(G,\rho, K) = \frac{1}{\operatorname{tr}(\mathcal{Q}_{\rho})}
\, q^{-\omega(\hat{b}) |\rho(\Omega_2)|}\,
\operatorname{tr}\!\bigl(\mathcal{Q}_{\rho}^{\otimes n}\,\rho(\hat{b})\bigr),
    \end{equation}
where $\omega(\hat{b}) = n_+ - n_-$,  $\rho(\Omega_2) = |\rho(\Omega_2)| \mathds{1}$.
\end{proposition}
\end{tcolorbox}
$\triangleleft$ The invariance under the MII follows from \eqref{eq:tr2} (the proof of \eqref{eq:tr2}itself can be found in \parencite{morozov-chern}):

\begin{multline}
    q^{-|\rho(\Omega_2)|} \text{tr}(\mathcal{Q}_\rho^{\otimes n}(\mathds{1}^{\otimes n - 2} \otimes\mathcal{R}_\rho)(\rho(\hat{b})\otimes \mathds{1}) = Q^{i_1}_{j_1}\cdots Q^{i_{n-1}}_{j_{n-1}}Q^{i_n}_{j_n}R^{j_{n-1}j_n}_{k i_n}b^{j_1\cdots k}_{i_1 \cdots i_{n-1}} = \\=\left[Q^a_b R^{c b}_{d a} = q^{|\rho(\Omega_2)|} \delta^c_d\right] =Q^{i_1}_{j_1}\cdots Q^{i_{n-1}}_{j_{n-1}} b^{j_1\cdots j_{n-1}}_{i_1 \cdots i_{n-1}} = \text{tr}(\mathcal{Q}_\rho^{\otimes n - 1} \rho(\hat{b}))
\end{multline}
\begin{question}
    Invariance of $I_\rho$ under MI relies on the following property:
\begin{equation}
(\mathcal{Q}\otimes \mathcal{Q}) \mathcal{R} = \mathcal{R}(\mathcal{Q}\otimes \mathcal{Q}).
\end{equation}
Its statement and proof are omitted in the seminal article. How can it be derived from the definitions \eqref{eq:R} and \eqref{eq:Q}?
\end{question}$\triangleright$

At this point, we can already see all the fruitful results that quantization of $U(\mathfrak{g})$ gives and apply them to concrete Lie algebras, namely $\mathfrak{sl}_2$ and $\mathfrak{sl}_N$.

\subsection{Fundamental representation of $\mathfrak{sl}_2 ~~ \to$  Jones polynomial}
\subsubsection{Explicit formulas for operators}

\begin{equation}
\begin{aligned}
\mathcal{R}_{\mathfrak{sl}_2, \square} &= 
  \hat{P}q^{\frac{H \otimes H}{2}} (1 + (q-q^{-1})E\otimes F) = \begin{pmatrix}
  q^{\frac{1}{2}} & 0 & 0 & 0 \\
  0 & 0 & q^{-\frac{1}{2}} & 0 \\
  0 & q^{-\frac{1}{2}} & q^{\frac{1}{2}}-q^{-\frac{3}{2}} & 0 \\
  0 & 0 & 0 & q^{\frac{1}{2}}
  \end{pmatrix},  \\ 
\mathcal{Q}_{\mathfrak{sl}_2, \square} &= q^{H} = \begin{pmatrix}
    q & 0\\
    0 & q^{-1}
  \end{pmatrix}, ~~~~~~~~~  |\square_{\mathfrak{sl}_2}(\Omega_2)| = \frac{3}{2}
\end{aligned}
\end{equation}

\subsubsection{Example: positive and negative trefoils}

    \begin{equation}
        \begin{minipage}{0.1\linewidth}
            \resizebox{\linewidth}{!}{\input{../tikz knots/3_1-brade.tex}}
            \end{minipage} \ \ 
            \leftrightarrow
        \begin{minipage}{0.13\linewidth}
            \vspace{7pt}
            \resizebox{\linewidth}{!}{\begin{tikzpicture}
\begin{knot}[
consider self intersections,
flip crossing/.list={1, 3}, %переворот пересечений
clip width=5,
]
\strand[line width=3pt, black]
(0, 1) to[out=180,in=-120,looseness=2]
%декартовы координаты
(-30:1) to[out=60,in=120,looseness=2]
%полярные координаты
(210:1) to[out=-60,in=0,looseness=2] (90:1);
\end{knot}
\end{tikzpicture}}
            \end{minipage} \ \ \
        I(
            \begin{minipage}{0.06\linewidth}
            \vspace{2pt}
            \resizebox{\linewidth}{!}{\begin{tikzpicture}
\begin{knot}[
consider self intersections,
flip crossing/.list={1, 3}, %переворот пересечений
clip width=5,
]
\strand[line width=3pt, black]
(0, 1) to[out=180,in=-120,looseness=2]
%декартовы координаты
(-30:1) to[out=60,in=120,looseness=2]
%полярные координаты
(210:1) to[out=-60,in=0,looseness=2] (90:1);
\end{knot}
\end{tikzpicture}}
            \end{minipage}
        ) = \frac{1}{q + q^{-1}} q^{-\frac{9}{2}} tr(\mathcal{Q}_{\mathfrak{sl}_2, \square}^{\otimes 2}\mathcal{R}_{\mathfrak{sl}_2, \square}^3) = q^{-2}+q^{-6}-q^{-8}
    \end{equation}

        \begin{equation}
        \begin{minipage}{0.1\linewidth}
            \resizebox{\linewidth}{!}{\begin{tikzpicture}
\begin{knot}[
consider self intersections,
flip crossing/.list={2},
clip width=5,
flip crossing/.list= {1, 2, 3}
]
\strand[line width=3pt, black]
(0, 0) to[out=90,in=-90,looseness=1]
(1, 1) to[out=90,in=-90,looseness=1] 
(0, 2) to[out=90,in=-90,looseness=1]
(1, 3) to[out=45,in=90,looseness=1]
(2, 1.5) to[out=-90,in=-45,looseness=1]
(1, 0) to[out=90,in=-90,looseness=1]
(0, 1) to[out=90,in=-90,looseness=1]
(1, 2) to[out=90,in=-90,looseness=1]
(0,3) to[out=135,in=90,looseness=1]
(-1, 1.5) to[out=-90,in=-135,looseness=1]
(0,0);

\draw[->, line width=3pt, black] (0, 2.1) -- (0, 1.9);
\draw[->, line width=3pt, black] (1, 1.1) -- (1, 0.9);
\draw[->, line width=3pt, black] (1, 2.1) -- (1, 1.9);
\draw[->, line width=3pt, black] (0, 1.1) -- (0, 0.9);

\end{knot}
\end{tikzpicture}}
            \end{minipage} \ \ 
            \leftrightarrow
        \begin{minipage}{0.13\linewidth}
            \vspace{7pt}
            \resizebox{\linewidth}{!}{\begin{tikzpicture}
\begin{knot}[
consider self intersections,
flip crossing=2, %переворот пересечений
clip width=10,
]
\strand[line width=3pt, black]
(0, 2) to[out=180,in=-120,looseness=2]
%декартовы координаты
(-30:2) to[out=60,in=120,looseness=2]
%полярные координаты
(210:2) to[out=-60,in=0,looseness=2] (90:2);
\end{knot}
\end{tikzpicture}}
            \end{minipage} \ \ \
        I(
            \begin{minipage}{0.06\linewidth}
            \vspace{2pt}
            \resizebox{\linewidth}{!}{\begin{tikzpicture}
\begin{knot}[
consider self intersections,
flip crossing/.list={1, 3}, %переворот пересечений
clip width=5,
]
\strand[line width=3pt, black]
(0, 1) to[out=180,in=-120,looseness=2]
%декартовы координаты
(-30:1) to[out=60,in=120,looseness=2]
%полярные координаты
(210:1) to[out=-60,in=0,looseness=2] (90:1);
\end{knot}
\end{tikzpicture}}
            \end{minipage}
        ) = \frac{1}{q + q^{-1}} q^{-\frac{9}{2}} tr(\mathcal{Q}_{\mathfrak{sl}_2, \square}^{\otimes 2}\mathcal{R}_{\mathfrak{sl}_2, \square}^{-3}) = q^{2}+q^{6}-q^{8}
    \end{equation}
    
The examples illustrate that the resulting polynomial coincides with the Jones polynomial (see examples in \ref{subsec:Jones_ex}). To prove this rigorously, we turn to the next point.

\subsubsection{Skein-relations}
\label{subsec:skein_sl2}
Let's search for the relation on $I$ in the following form:

    \begin{equation}
    \alpha_+ \cdot I \left (
    \begin{minipage}{0.06\linewidth}
        \resizebox{\linewidth}{!}{\begin{tikzpicture}[rotate=180]
\begin{knot}[
consider self intersections,
clip width=5,
]
\strand[line width=3pt, ->, black]
(-1, -1) to[out=45,in=-135,looseness=1] (1, 1);
\strand[line width=3pt, <-, black]
(-1, 1) to[out=-45,in=135,looseness=1] (1, -1);
\end{knot}
\end{tikzpicture}}
        \end{minipage}
    \right ) + \alpha_- \cdot I\left (
    \begin{minipage}{0.06\linewidth}
        \resizebox{\linewidth}{!}{\input{../tikz knots/cross2-_rotate.tex}}
        \end{minipage}
    \right ) = \alpha_0 \cdot I\left (
    \begin{minipage}{0.06\linewidth}
        \resizebox{\linewidth}{!}{\input{../tikz knots/R2_2_down_down.tex}}
        \end{minipage}
    \right )
    \label{eq:skein-r}
    \end{equation}

    In other words, we have to find such $\alpha_+,\alpha_-,\alpha_0$ that:
    
    \begin{equation}
        \alpha_+ q^{-\Omega_2} \mathcal{R} + 
        \alpha_- q^{\Omega_2} \mathcal{R}^{-1} = 
        \alpha_0 \ 1 \otimes 1, \ \Omega_2 = \frac{3}{2}
    \end{equation}

    Finally we get that $I$ has the following properties:

    \begin{tcolorbox}
        \begin{equation}
            q^2 I \left (
        \begin{minipage}{0.06\linewidth}
            \resizebox{\linewidth}{!}{\begin{tikzpicture}[rotate=180]
\begin{knot}[
consider self intersections,
clip width=5,
]
\strand[line width=3pt, ->, black]
(-1, -1) to[out=45,in=-135,looseness=1] (1, 1);
\strand[line width=3pt, <-, black]
(-1, 1) to[out=-45,in=135,looseness=1] (1, -1);
\end{knot}
\end{tikzpicture}}
            \end{minipage}
        \right ) - q^{-2} I\left (
        \begin{minipage}{0.06\linewidth}
            \resizebox{\linewidth}{!}{\input{../tikz knots/cross2-_rotate.tex}}
            \end{minipage}
        \right ) = (q-q^{-1}) I\left (
        \begin{minipage}{0.06\linewidth}
            \resizebox{\linewidth}{!}{\input{../tikz knots/R2_2_down_down.tex}}
            \end{minipage}
        \right )
        \end{equation}

        \begin{equation}
            I(
                \begin{minipage}{0.04\linewidth}
                \resizebox{\linewidth}{!}{\begin{tikzpicture}
\begin{knot}
\strand[line width=3pt, black] 
(0,0) circle[radius=1cm];
\end{knot}
\end{tikzpicture}}
                \end{minipage}
            ) = 1
        \end{equation}

        \begin{equation}
            I(L_1 \cup L_2) = (q+q^{-1})I(L_1) \cdot I(L_2)
        \end{equation}
    \end{tcolorbox}

    Those exactly match the definition of the Jones polynomial \eqref{eq:skein}.


    

\subsection{Fundamental representation of $\mathfrak{sl}_N ~~ \to$ -- HOMFLY polynomial}
\subsubsection{Explicit formulas for operators}
Tetraedral fundamental representation of $\mathfrak{sl}_N$ has property $\square_{\mathfrak{sl}_N}(E_\alpha)^2 = \square_{\mathfrak{sl}_N}(F_\alpha)^2 = 0$ for any $\alpha \in \Phi^+$. Hence \eqref{eq:R} takes form:
\begin{equation}
\mathcal{R}_{\mathfrak{sl}_N, \square} = \hat{P}\, q^{\sum_{\alpha \in \Psi} h_\alpha \otimes h_\alpha^\vee}
\overrightarrow{\prod_{\beta \in \Phi^+_{(n)}}}
Biggl(1 + 
(q - q^{-1}) E_\beta \otimes F_\beta
\Biggr)
\end{equation}

Nonzero elements of $\mathcal{R}_{\mathfrak{sl}_N, \square}$ \eqref{eq:R}are:

    \begin{equation}
    	R^{ii}_{ii} = q^{-\frac{1}{N}}\cdot q~~~~~~~~
    	R^{ij}_{ji} = q^{-\frac{1}{N}}~,~~~i \neq j~~~~~~~~
    	R^{ij}_{ij} = q^{-\frac{1}{N}}(q - q^{-1})~,~~~i>j 
      \label{eq:RslN}
    \end{equation}

    \begin{equation}
        \mathcal{Q}_{\mathfrak{sl}_N, \square} = \text{diag}(q^{N + 1 - 2i})~,~~~i=1, \cdots, N  ~~~~~~~~ |\square_{\mathfrak{sl}_N}(\Omega_2)| = N - \frac{1}{N}, ~~~~~ \text{tr}(\mathcal{Q}_{\mathfrak{sl}_N, \square}) = [N]_q
    \end{equation}

    \tiny
    \begin{question}
      Although the expressions \eqref{eq:RslN}, up to normalization, are given in many articles on the $\mathcal{R}$-matrix, we failed to derive them explicitly. For example, for $N = 3$ we get:
\[
\begin{bmatrix}
q\cdot q^{1/3} & 0 & 0 & 0 & 0 & 0 & 0 & 0 & 0 \\
0 & 0 & 0 & q^{-4/3} & 0 & 0 & 0 & 0 & 0 \\
0 & 0 & 0 & 0 & 0 & 0 & 1 & 0 & 0 \\
0 & q^{-2/3} & 0 & (q - q^{-1})q^{-2/3} & 0 & 0 & 0 & 0 & 0 \\
0 & 0 & 0 & 0 & q\cdot q^{-1/3} & 0 & 0 & 0 & 0 \\
0 & 0 & 0 & 0 & 0 & 0 & 0 & 1 & 0 \\
0 & 0 & q^{-2/3} & 0 & 0 & 0 & (q - q^{-1})q^{-2/3} & 0 & 0 \\
0 & 0 & 0 & 0 & 0 & q^{2/3} & 0 & (q - q^{-1})q^{2/3} & 0 \\
0 & 0 & 0 & 0 & 0 & 0 & 0 & 0 & q\cdot q^{-1}
\end{bmatrix}\]
      Further, we use the given results regardless of our incomplete understanding of their origins. However, this question still remains.
    \end{question}
  \normalsize  

As far as prefactor $\mathcal{R}$ always goes with $q^{-|\Omega_2|}$ we can discard $q^{-1/N}$ and $1/N$ terms.

\subsubsection{Example: positive trefoil}
\begin{multline}
  \text{tr}(\mathcal{Q}^{\otimes 2}\mathcal{R}^3) = q^{2N + 2} \sum\limits_{i, j, k, l, m, n} q^{-2(i + j)} R^{ij}_{kl}R^{kl}_{mn}R^{mn}_{ij} =\\= q^{2N + 2} \left(\sum\limits_{i = 1}^N q^{-4i} q \cdot q\cdot q + \sum\limits_{\substack{
  i, j = 1 \\
  i > j
}}^N q^{-2(i + j)} ((q - q^{-1})\cdot 1\cdot 1 + 1 \cdot (q - q^{-1})\cdot 1 + 1 \cdot 1 \cdot (q - q^{-1}) + (q - q^{-1})^3)\right) 
\end{multline}

\begin{equation}
I(
        \begin{minipage}{0.06\linewidth}
        \vspace{2pt}
        \resizebox{\linewidth}{!}{\begin{tikzpicture}
\begin{knot}[
consider self intersections,
flip crossing/.list={1, 3}, %переворот пересечений
clip width=5,
]
\strand[line width=3pt, black]
(0, 1) to[out=180,in=-120,looseness=2]
%декартовы координаты
(-30:1) to[out=60,in=120,looseness=2]
%полярные координаты
(210:1) to[out=-60,in=0,looseness=2] (90:1);
\end{knot}
\end{tikzpicture}}
        \end{minipage}
    ) = \frac{1}{[N]_q} q^{-3N} \text{tr}(\mathcal{Q}^{\otimes 2}\mathcal{R}^3) = q^{-2N -2}(1 + q^4 - q^{-2N +2})
\end{equation}

\subsubsection{Skein-relations}
Analogously to \ref{subsec:skein_sl2} we get that:
        \begin{tcolorbox}
        \begin{equation}
            q^N I \left (
        \begin{minipage}{0.06\linewidth}
            \resizebox{\linewidth}{!}{\begin{tikzpicture}[rotate=180]
\begin{knot}[
consider self intersections,
clip width=5,
]
\strand[line width=3pt, ->, black]
(-1, -1) to[out=45,in=-135,looseness=1] (1, 1);
\strand[line width=3pt, <-, black]
(-1, 1) to[out=-45,in=135,looseness=1] (1, -1);
\end{knot}
\end{tikzpicture}}
            \end{minipage}
        \right ) - q^{-N} I\left (
        \begin{minipage}{0.06\linewidth}
            \resizebox{\linewidth}{!}{\input{../tikz knots/cross2-_rotate.tex}}
            \end{minipage}
        \right ) = (q-q^{-1}) I\left (
        \begin{minipage}{0.06\linewidth}
            \resizebox{\linewidth}{!}{\input{../tikz knots/R2_2_down_down.tex}}
            \end{minipage}
        \right )
        \end{equation}

        \begin{equation}
            I(
                \begin{minipage}{0.04\linewidth}
                \resizebox{\linewidth}{!}{\begin{tikzpicture}
\begin{knot}
\strand[line width=3pt, black] 
(0,0) circle[radius=1cm];
\end{knot}
\end{tikzpicture}}
                \end{minipage}
            ) = \frac{q^N-q^{-N}}{q - q^{-1}}
        \end{equation}

        \begin{equation}
            I(L_1 \cup L_2) = I(L_1) \cdot I(L_2)
        \end{equation}
    \end{tcolorbox}

     By substituting $\alpha = q^N$ и $\beta = q - q^{-1}$ relulting relations match Jim Hoste's HOMFLY polynomial definition.



\section{Reduction of Khovanov complexes for antiparallel lock}

Probable motivation: $\mathcal{R}$-matrix planar decomposition for HOMFLY \parencite{lena-homfly}. 
\subsection{Tangle Khovanov complex reduction for 4 types of antiparallel locks}

\begin{figure}[h]
\includegraphics[width=0.9\textwidth]{../img/antiparallel locks.png}
\caption{Four types of antiparallel locks}
\label{fig:locks}
\end{figure}

\begin{question}
 How to repeatedly apply Lemma from \parencite{bar-natan-bipartite} to resulting complex, so at the end we get zero differentials?
\end{question}

\subsection{Examples}
\subsubsection{positive Hopf-link}
[to be explained]

\subsubsection{eight-knot} see \parencite{bar-natan-bipartite}

\subsection{Unsuccessfull attempt to translate cobordism-approach on linear operators}
The approach given in \parencite{bar-natan} motivates following the same path, but acting directly on the Khovanov complex. However, the naive attempt faces difficulties. First, the main lemma works poorly for graded spaces, as it uses a basis change. Secondly, simple matrix operators do not distinguish connected components. The general problem lies in the intricacy of their linear representation Does it even exists -- it is an open \begin{question}.
\end{question}

[to be explained]


\printbibliography

\newpage

\section{A: explicit expressions for diagram commutation check}

In this section, we use space numbering method from \cite{bar-natan} and always assume that the spaces in the tensor product $V_{i_1} \otimes \cdots \otimes V_{i_n} \stackrel{\text{not}}{=} i_1 \otimes \cdots \otimes i_n$ are ordered such that $i_1<\cdots <i_n$ ($i_1 \equiv 1$ by numbering definition). We are to analyse diagrams $^{gf}_{hk}$

\[
\xymatrix{
 & B \ar[dr]^{f} & \\
A \ar[ur]^{g} \ar[dr]_{h} & & D \\
 & C \ar[ur]_{k} &
}
\]
,where $g, f, h, k \in {m, \Delta}$, and $ABDC$ --- a face of the hypercube.

\textbf{Case 1: $\mathbf{^{mm}_{mm}}$}

Only two essentially distinct types of diagrams exist:
\[
\begin{array}{c c}
    \left\{\begin{minipage}{0.12\linewidth}
      \resizebox{\linewidth}{!}{\begin{tikzpicture}[very thick]
    \draw (0,0) circle (0.4);
    \draw (1,0) circle (0.4);
    \draw (2,0) circle (0.4);
    \draw (3,0) circle (0.4);

    \draw[->] (0.5, -0.5) -- (0.5, -1) node[midway, right]{$m$};

    \draw (0.5,-1.5) circle (0.4);
    \draw (2,-1.5) circle (0.4);
    \draw (3,-1.5) circle (0.4);

    \draw[->] (2.5, -2) -- (2.5, -2.5) node[midway, right]{$m$};

    \draw (0.5,-3) circle (0.4);
    \draw (2.5,-3) circle (0.4);
\end{tikzpicture}}
  \end{minipage}\right\} = 
\left\{\begin{minipage}{0.12\linewidth}
      \resizebox{\linewidth}{!}{\input{../tikz diagrams/mm_mm2.tex}}
  \end{minipage}\right\} ~~&~~
  \left\{\begin{minipage}{0.09\linewidth}
      \resizebox{\linewidth}{!}{\input{../tikz diagrams/mm_mm3.tex}}
  \end{minipage}\right\} = 
\left\{\begin{minipage}{0.09\linewidth}
      \resizebox{\linewidth}{!}{\begin{tikzpicture}[very thick]
    \draw (0,0) circle (0.4);
    \draw (1,0) circle (0.4);
    \draw (2,0) circle (0.4);

    \draw[->] (1.5, -0.5) -- (1.5, -1) node[midway, right]{$m$};

    \draw (0,-1.5) circle (0.4);
    \draw (1.5,-1.5) circle (0.4);

    \draw[->] (0.75, -2) -- (0.75, -2.5) node[midway, right]{$m$};

    \draw (0.75,-3) circle (0.4);
\end{tikzpicture}}
  \end{minipage}\right\} \\
  \text{far commutativity} ~~& ~~\text{near commutativity}
\end{array}
\]

The first equality is obvious, while the second expresses the associativity property of multiplication:

\begin{equation}
    \boxed{
m(1\otimes m) = m(m\otimes 1) }
\label{eq:comm_1}
\end{equation}

\textbf{Case 2: $\mathbf{^{m\Delta}_{m\Delta}}$}

Diagram types:

\[
\begin{array}{c c}
    \left\{\begin{minipage}{0.12\linewidth}
      \resizebox{\linewidth}{!}{\begin{tikzpicture}[very thick]
    \draw[dashed] (0,0) circle (0.4);
    \draw (1,0) circle (0.4);
    \draw[dotted] (2.5,0) circle (0.4);

    \draw[->] (0.5, -0.5) -- (0.5, -1) node[midway, right]{$m$};

    \draw (0.5,-1.5) circle (0.4);
    \draw[dotted] (2.5,-1.5) circle (0.4);

    \draw[->] (2.5, -2) -- (2.5, -2.5) node[midway, right]{$\Delta$};

    \draw (0.5,-3) circle (0.4);
    \draw[dotted] (2,-3) circle (0.4);
    \draw[dotted] (3,-3) circle (0.4);
\end{tikzpicture}}
  \end{minipage}\right\} \not= 
\left\{\begin{minipage}{0.12\linewidth}
      \resizebox{\linewidth}{!}{\begin{tikzpicture}[very thick]
    \draw[dashed] (-0.5,0) circle (0.4);
    \draw (1,0) circle (0.4);
    \draw[dotted] (2,0) circle (0.4);

    \draw[->] (1.5, -0.5) -- (1.5, -1) node[midway, right]{$m$};

    \draw[dashed] (-0.5,-1.5) circle (0.4);
    \draw (1.5,-1.5) circle (0.4);

    \draw[->] (-0.5, -2) -- (-0.5, -2.5) node[midway, right]{$\Delta$};

    \draw[dashed] (-1,-3) circle (0.4);
    \draw[dashed] (0,-3) circle (0.4);
    \draw (1.5,-3) circle (0.4);
\end{tikzpicture}}
  \end{minipage}\right\} ~~&~~
  \left\{\begin{minipage}{0.06\linewidth}
      \resizebox{\linewidth}{!}{\input{../tikz diagrams/mD_mD3.tex}}
  \end{minipage}\right\} = 
\left\{\begin{minipage}{0.06\linewidth}
      \resizebox{\linewidth}{!}{\begin{tikzpicture}[very thick]
    \draw (0,0) circle (0.4);
    \draw (1,0) circle (0.4);

    \draw[->] (0.5, -0.5) -- (0.5, -1) node[midway, right]{$m_2$};

    \draw (0.5,-1.5) circle (0.4);

    \draw[->] (0.5, -2) -- (0.5, -2.5) node[midway, right]{$\Delta_2$};

    \draw (0,-3) circle (0.4);
    \draw (1,-3) circle (0.4);
\end{tikzpicture}}
  \end{minipage}\right\} \\
  \text{far commutativity} ~~& ~~\text{near commutativity}
\end{array}
\]

The first diagram never appears. To see this, note that on the right-hand side the resulting space consists of two smoothings with only dotted spatial edges, while on the left --- with only dashed ones. Consequently, the resulting spaces correspond to different vertices of the hypercube. 

The second equation is obvious. It may seem counterintuitive that we can resolve a smoothing using two different $\Delta \cdot m$ operators and still arrive at the same point, but the Hopf link example will likely dispel this doubt(see Fig. \ref{fig:hopf-diagram}).

\textbf{Case 3: $\mathbf{^{\Delta m}_{\Delta m}}$}

Similar to case 2, the far commutativity diagram does not appear:
\[
    \left\{\begin{minipage}{0.12\linewidth}
      \resizebox{\linewidth}{!}{\input{../tikz diagrams/Dm_Dm1.tex}}
  \end{minipage}\right\} \not= 
\left\{\begin{minipage}{0.12\linewidth}
      \resizebox{\linewidth}{!}{\input{../tikz diagrams/Dm_Dm2.tex}}
  \end{minipage}\right\}
\]

Now we have to consider two essentialy distinct near commutativity diagrams.

\[
\begin{array}{c c}
    \left\{\begin{minipage}{0.09\linewidth}
      \resizebox{\linewidth}{!}{\input{../tikz diagrams/Dm_Dm3.tex}}
  \end{minipage}\right\} = 
\left\{\begin{minipage}{0.09\linewidth}
      \resizebox{\linewidth}{!}{\begin{tikzpicture}[very thick]
    \draw (0,0) circle (0.4);
    \draw (1.5,0) circle (0.4);

    \draw[->] (1.5, -0.5) -- (1.5, -1) node[midway, right]{$\Delta$};

    \draw (0,-1.5) circle (0.4);
    \draw (1,-1.5) circle (0.4);
    \draw (2,-1.5) circle (0.4);

    \draw[->] (0.5, -2) -- (0.5, -2.5) node[midway, right]{$m$};

    \draw (0.5,-3) circle (0.4);
    \draw (2,-3) circle (0.4);
\end{tikzpicture}}
  \end{minipage}\right\} ~~&~~
  \left\{\begin{minipage}{0.06\linewidth}
      \resizebox{\linewidth}{!}{\input{../tikz diagrams/Dm_Dm5.tex}}
  \end{minipage}\right\} = 
\left\{\begin{minipage}{0.06\linewidth}
      \resizebox{\linewidth}{!}{\input{../tikz diagrams/Dm_Dm6.tex}}
  \end{minipage}\right\}
\end{array}
\]

The first diagram corresponds to the following chains ($P$ --- permutation operator):

\begin{equation}
    1\otimes i ~~ \left[\begin{array}{ccl}

    \xrightarrow{1 \otimes \Delta} &1\otimes i \otimes k &\left[
    \begin{array}{ccc}
        \xrightarrow{m \otimes 1}& 1 \otimes k&, ~k >i\\
        \xrightarrow{(m \otimes 1)(1 \otimes P)}& 1 \otimes i&
    \end{array}
    \right.\\[8mm]

    \xrightarrow{\Delta \otimes 1} &1\otimes k \otimes i &\left[
    \begin{array}{ccc}
        \xrightarrow{1 \otimes m}& 1 \otimes k&, ~k<i\\
        \xrightarrow{(m \otimes 1)(1 \otimes P)}& 1 \otimes k&, ~k<i
    \end{array}
    \right.\\[8mm]

    \xrightarrow{(1 \otimes P)(\Delta \otimes 1)} &1\otimes i \otimes k &\left[
    \begin{array}{ccc}
        \xrightarrow{m \otimes 1}& 1 \otimes k&, ~k>i\\
        \xrightarrow{1 \otimes m}& 1 \otimes i&
    \end{array}
    \right.
\end{array}
\right.
\label{eq:mD_chain}
\end{equation}

Thus the first diagram commutativity implies that $m$ and $\Delta$ satisfy the following equations:

\begin{equation}
(m\otimes 1)(1\otimes \Delta) = (m\otimes 1)(1 \otimes P)(\Delta \otimes 1)
\end{equation}
\begin{equation}
(m\otimes 1)(1\otimes P)(1\otimes\Delta) = (1\otimes m)(1 \otimes P)(\Delta \otimes 1)
\end{equation}
\begin{equation}
(1\otimes m)(\Delta\otimes 1) = (m\otimes 1)(1 \otimes P)(\Delta \otimes 1)
\end{equation}

The second equation is obvious. One can verify that such a diagram appears when the overlapping unknots diagram $\begin{minipage}{0.06\linewidth}
      \resizebox{\linewidth}{!}{\input{../tikz knots/0_2.tex}}
  \end{minipage}$ is smoothed.

\textbf{Case 4: $\mathbf{^{m\Delta}_{\Delta m}}$}

Diagram types:

\[
\begin{array}{c c}
    \left\{\begin{minipage}{0.12\linewidth}
      \resizebox{\linewidth}{!}{\input{../tikz diagrams/mD_Dm1.tex}}
  \end{minipage}\right\} = 
\left\{\begin{minipage}{0.12\linewidth}
      \resizebox{\linewidth}{!}{\begin{tikzpicture}[very thick]
    \draw (0,0) circle (0.4);
    \draw (1,0) circle (0.4);
    \draw (2.5,0) circle (0.4);

    \draw[->] (2.5, -0.5) -- (2.5, -1) node[midway, right]{$\Delta$};

    \draw (0,-1.5) circle (0.4);
    \draw (1,-1.5) circle (0.4);
    \draw (2,-1.5) circle (0.4);
    \draw (3,-1.5) circle (0.4);

    \draw[->] (0.5, -2) -- (0.5, -2.5) node[midway, right]{$m$};

    \draw (0.5,-3) circle (0.4);
    \draw (2,-3) circle (0.4);
    \draw (3,-3) circle (0.4);
\end{tikzpicture}}
  \end{minipage}\right\} ~~&~~
  \left\{\begin{minipage}{0.06\linewidth}
      \resizebox{\linewidth}{!}{\begin{tikzpicture}[very thick]
    \draw (0,0) circle (0.4);
    \draw (1,0) circle (0.4);

    \draw[->] (0.5, -0.5) -- (0.5, -1) node[midway, right]{$m$};

    \draw (0.5,-1.5) circle (0.4);

    \draw[->] (0.5, -2) -- (0.5, -2.5) node[midway, right]{$\Delta$};

    \draw (0,-3) circle (0.4);
    \draw (1,-3) circle (0.4);
\end{tikzpicture}}
  \end{minipage}\right\} = 
\left\{\begin{minipage}{0.09\linewidth}
      \resizebox{\linewidth}{!}{\input{../tikz diagrams/Dm_Dm3.tex}}
  \end{minipage}\right\} \\
  \text{far commutativity} ~~& ~~\text{near commutativity}
\end{array}
\]

The first equation is obvious. The second one implies that any chain from (\ref{eq:mD_chain}) is equal to $\Delta\cdot m$:

\begin{equation}
\boxed{
\begin{array}{c}
\Delta m = (m\otimes 1)(1\otimes \Delta) = (1\otimes m)(\Delta\otimes 1)=\\
=(m\otimes 1)(1 \otimes P)(\Delta \otimes 1)=\\
=(m\otimes 1)(1\otimes P)(1\otimes\Delta) =\\
=(m\otimes 1)(1 \otimes P)(\Delta \otimes 1)=\\ 
=(1\otimes m)(1 \otimes P)(\Delta \otimes 1)
\end{array}
}
\label{eq:comm_2}
\end{equation}



\textbf{Case 5: $\mathbf{^{\Delta\Delta}_{\Delta \Delta}}$}

The diagrams are analogous to those in case 1: 

\[
\begin{array}{c c}
    \left\{\begin{minipage}{0.12\linewidth}
      \resizebox{\linewidth}{!}{\input{../tikz diagrams/DD_DD1.tex}}
  \end{minipage}\right\} = 
\left\{\begin{minipage}{0.12\linewidth}
      \resizebox{\linewidth}{!}{\begin{tikzpicture}[very thick]
    \draw (0,0) circle (0.4);
    \draw (2,0) circle (0.4);

    \draw[->] (2, -0.5) -- (2, -1) node[midway, right]{$\Delta$};

    \draw (0,-1.5) circle (0.4);
    \draw (1.5,-1.5) circle (0.4);
    \draw (2.5,-1.5) circle (0.4);

    \draw[->] (0, -2) -- (0, -2.5) node[midway, right]{$\Delta$};

    \draw (-0.5,-3) circle (0.4);
    \draw (0.5,-3) circle (0.4);
    \draw (1.5,-3) circle (0.4);
    \draw (2.5,-3) circle (0.4);
\end{tikzpicture}}
  \end{minipage}\right\} ~~&~~
  \left\{\begin{minipage}{0.09\linewidth}
      \resizebox{\linewidth}{!}{\begin{tikzpicture}[very thick]
    \draw (0,0) circle (0.4);

    \draw[->] (0, -0.5) -- (0, -1) node[midway, right]{$\Delta$};

    \draw (-0.75,-1.5) circle (0.4);
    \draw (0.75,-1.5) circle (0.4);

    \draw[->] (0.75, -2) -- (0.75, -2.5) node[midway, right]{$\Delta$};

    \draw (-0.75,-3) circle (0.4);
    \draw (0.25,-3) circle (0.4);
    \draw (1.25,-3) circle (0.4);
\end{tikzpicture}}
  \end{minipage}\right\} = 
\left\{\begin{minipage}{0.09\linewidth}
      \resizebox{\linewidth}{!}{\input{../tikz diagrams/DD_DD4.tex}}
  \end{minipage}\right\} \\
  \text{far commutativity} ~~& ~~\text{near commutativity}
\end{array}
\]


The near commutativity expresses the coassociativity of comultiplication:

\begin{equation}
    \boxed{
        (1\otimes \Delta)\Delta = (\Delta\otimes 1)\Delta
    }
    \label{eq:comm_3}
\end{equation}

Now identities (\ref{eq:comm_1}), (\ref{eq:comm_2}), (\ref{eq:comm_3}) can be explicitly verified using the definition of $m$ and $\Delta$.




\end{document}