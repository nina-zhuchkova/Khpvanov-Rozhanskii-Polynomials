\documentclass[12pt,a4paper]{article}
%%% Работа с русским языком
\usepackage{cmap}					% поиск в PDF
\usepackage{mathtext} 				% русские буквы в формулах
\usepackage[T2A]{fontenc}			% кодировка
\usepackage[utf8]{inputenc}			% кодировка исходного текста
\usepackage[main=english,russian]{babel}	% локализация и переносы
\selectlanguage{english}
\usepackage{caption}

%%% Дополнительная работа с математикой
\usepackage{amsmath,amsfonts,amssymb,amsthm,mathtools} % AMS
\usepackage{icomma}
\usepackage{physics}
\usepackage{multicol}
\usepackage{bm}
\usepackage{mathrsfs}
\usepackage{verbatim}

%%% Номера формул
%\mathtoolsset{showonlyrefs=true} 
%\usepackage{leqno} 

%%% Свои команды
\DeclareMathOperator{\sgn}{sgn}

\usepackage{csquotes} 
\usepackage[backend=biber,style=authoryear]{biblatex}


%%% Работа с графикой
\usepackage{graphicx}
\graphicspath{{images/}}  
\setlength\fboxsep{3pt} 
\setlength\fboxrule{1pt} 
\usepackage{wrapfig} 
\usepackage{tikz}
\usepackage{pgfplots}
\usepackage{pgfplotstable}
\usepgfplotslibrary{polar}
\pgfplotsset{compat=1.18} 

%%% Работа с таблицами
\usepackage{array,tabularx,tabulary,booktabs}
\usepackage{longtable}  
\usepackage{multirow} 
\usepackage{soul} 

%%% Теоремы
\theoremstyle{plain} 
\newtheorem{theorem}{Th}[section]
\newtheorem{proposition}[theorem]{Proposition}
\newtheorem{question}{Question}[section]
 
\theoremstyle{definition} 
\newtheorem{corollary}{Corollary}[theorem]
\newtheorem{problem}{Problem}[section]
\newtheorem{definition}{Def}[section]
 
\theoremstyle{remark} 
\newtheorem*{nonum}{Solution}

%%% Программирование
\usepackage{etoolbox} 

%%% Гиперссылки
\usepackage{hyperref}

\usetikzlibrary{knots}
\usepackage{tcolorbox}

%%% Страница
\usepackage{geometry} 
	\geometry{top=20mm, bottom=20mm, left=15mm, right=20mm}

\usepackage{setspace} 

\usepackage{lastpage} 
\usepackage{amssymb}
\usepackage{xcolor}
\DeclareMathOperator{\Ker}{Ker}
\DeclareMathOperator{\qdim}{qdim}
\usepackage[all]{xy}
\usepackage{ dsfont }

\addbibresource{../source.bib}
\begin{document}

\begin{center}
    \Large \textbf{Название надо придумать} \\[1em]
    \small
    Moscow Institute of Physics and Technology \\[0.5em]
    Laboratory of Mathematical and Theoretical Physics
\end{center}

\small
\begin{flushright}
\begin{tabular}{c c}
\textbf{Authors:} & \textbf{Scientific advisors:} \\[0.5em]
Artem Novokhatnii & Elena Lanina \\
Ekaterina Tsygankova & Radomir Stepanov \\
Eldar Miftakhov & \\
\end{tabular}

\centering{\small \textbf{Abstract}\\
няняняняянняня}

\vspace{1em}
\normalsize
\today
\end{flushright}
\normalsize
\vspace{2em}


\tableofcontents
\vspace{2em}

\section{Introduction}

\section{Jones polynomial via the Kauffman bracket}

\subsection{Knot invariants}

\begin{tcolorbox}
\begin{theorem}\label{thm:Reidemeister}
(Reidemeister 1927) 

Two diagrams represent isotopic links if and only if one 
can be transformed into the other by a finite number of 
planar isotopies and transformations of the following three types:
\begin{equation}
\Omega_1:  ~~~~~~ 
  \begin{minipage}{0.06\linewidth}
      \resizebox{\linewidth}{!}{\input{../tikz knots/R1_1}}
  \end{minipage} 
  \leftrightarrow
  \begin{minipage}{0.06\linewidth}
      \resizebox{\linewidth}{!}{\begin{tikzpicture}
\begin{knot}[
consider self intersections,
clip width=5,
]
\strand[line width=3pt, black]
(-1, -1) to[out=60,in=180,looseness=1]
(0, 1) to[out=0,in=120,looseness=1] (1, -1);
\end{knot}
\end{tikzpicture}}
  \end{minipage} 
  \leftrightarrow
  \begin{minipage}{0.06\linewidth}
      \resizebox{\linewidth}{!}{\input{../tikz knots/R1_3}}
  \end{minipage} 
\end{equation}

\begin{equation}
\Omega_2:  ~~~~~~ 
  \begin{minipage}{0.06\linewidth}
      \resizebox{\linewidth}{!}{\begin{tikzpicture}
\begin{knot}[
consider self intersections,
clip width=5,
]
\strand[line width=3pt, black]
(-1, -1) to[out=45,in=-90,looseness=1]
(0.5, 0) to[out=90,in=-45,looseness=1] (-1, 1);
\strand[line width=3pt, black]
(1, -1) to[out=135,in=-90,looseness=1]
(-0.5, 0) to[out=90,in=-135,looseness=1] (1, 1);
\end{knot}
\end{tikzpicture}}
  \end{minipage} 
  \leftrightarrow
  \begin{minipage}{0.06\linewidth}
      \resizebox{\linewidth}{!}{\input{../tikz knots/R2_2}}
  \end{minipage} 
\end{equation}

\begin{equation}
\Omega_3:  ~~~~~~ 
  \begin{minipage}{0.06\linewidth}
      \resizebox{\linewidth}{!}{\input{../tikz knots/R3_1}}
  \end{minipage} 
  \leftrightarrow
  \begin{minipage}{0.06\linewidth}
      \resizebox{\linewidth}{!}{\begin{tikzpicture}
\begin{knot}[
consider self intersections,
clip width=5,
]
\strand[line width=3pt, black]
(-1, 0) to[out=-30,in=180,looseness=1] 
(0, -0.5) to[out=0,in=-150,looseness=1] (1, 0);
\strand[line width=3pt, black]
(1, 1) to[out=-160,in=60,looseness=1] (-1, -1);
\strand[line width=3pt, black]
(-1, 1) to[out=-20,in=120,looseness=1] (1, -1);
\end{knot}
\end{tikzpicture}}
  \end{minipage} 
\end{equation}

\end{theorem}
\end{tcolorbox}

Theorem \ref{thm:Reidemeister} shows us that when constructing
a knot invariant we need to check for it's conservation on
Reidemeister moves $\Omega_1, \Omega_2, \Omega_3$.

For the proof of Theorem \ref{thm:Reidemeister} see e.g. \cite{prasolov-sossinsky}.

\subsection{Kauffman bracket}

\begin{tcolorbox}
\begin{definition}
Kauffman bracket axioms

\parencite{khovanov}:

    \begin{equation}
    \left\langle 
    \begin{minipage}{0.06\linewidth}
    \resizebox{\linewidth}{!}{\input{../tikz knots/cross1}}
    \end{minipage} \right\rangle = 
    \left\langle 
    \begin{minipage}{0.06\linewidth}
    \resizebox{\linewidth}{!}{\input{../tikz knots/R2_2}}
    \end{minipage} \right\rangle - q
    \left\langle 
    \begin{minipage}{0.06\linewidth}
    \resizebox{\linewidth}{!}{\input{../tikz knots/R2_2_rot}}
    \end{minipage} \right\rangle
    \label{eq:skein-kaufman}
    \end{equation}

    \begin{equation}
    \langle L_1 \cup L_2\rangle = \langle L_1\rangle \langle L_2 \rangle
    \end{equation}

    \begin{equation}
    \left\langle 
    \begin{minipage}{0.06\linewidth}
    \resizebox{\linewidth}{!}{\begin{tikzpicture}
\begin{knot}
\strand[line width=3pt, black] 
(0,0) circle[radius=1cm];
\end{knot}
\end{tikzpicture}}
    \end{minipage} \right\rangle = q + q^{-1}
    \end{equation}
\label{def:kauf-brack}
\end{definition}
\end{tcolorbox}

We will understand the negative smoothing to be the first resolution in
\eqref{eq:skein-kaufman}, and the positive smoothing to be the second
 one\footnote{It is important to note that if a crossing is reversed,
  the designations of the smoothings will also swap places}. Each 
  crossing in a knot can be smoothed in one of these two ways. 
  By smoothing all $n$ crossings, we obtain a diagram consisting of
   a certain number $p$ of circles. In total, there are $2^n$ 
   such diagrams, which we will call states $s$, Each state corresponds
    to a particular choice of smoothings, containing $\gamma$ positive smoothings.
Thus, from \ref{def:kauf-brack}, we obtain an expression for the 
Kauffman bracket as a state sum:

 \begin{equation}
  \langle L \rangle=\sum_s (-q)^{\gamma} (q + q^{-1})^p
  \label{eq:statsum}
 \end{equation}


Note that we can associate a two-dimensional graded vector space $V$,
to each circle, whose dimension is $\text{dim} V = q + q^{-1}$.
Then, \eqref{eq:statsum} can be rewritten as:

\begin{equation}
  \langle L \rangle=\sum_s (-q)^{\gamma} (\text{dim} V)^p
  \label{eq:statsum-V}
\end{equation}

This discussion gives us a significant insight on how the
Khovanov polynomial will be constructed. We will proceed with this
discussion later in Sec. \ref{sec:Khovanov}.

\subsection{Jones polynomial as a compensation of Kauffman bracket}

Let's check how Kauffman bracket behaves under Reidemeister moves:

\begin{equation}
  \left\langle 
    \begin{minipage}{0.06\linewidth}
    \vspace{0pt}
    \resizebox{\linewidth}{!}{\input{../tikz knots/R1_1}}
    \end{minipage} \right\rangle = -q^{2}
    \left\langle 
    \begin{minipage}{0.06\linewidth}
    \vspace{0pt}
    \resizebox{\linewidth}{!}{\begin{tikzpicture}
\begin{knot}[
consider self intersections,
clip width=5,
]
\strand[line width=3pt, black]
(-1, -1) to[out=60,in=180,looseness=1]
(0, 1) to[out=0,in=120,looseness=1] (1, -1);
\end{knot}
\end{tikzpicture}}
    \end{minipage} \right\rangle \ , \
    \left\langle 
    \begin{minipage}{0.06\linewidth}
    \vspace{0pt}
    \resizebox{\linewidth}{!}{\input{../tikz knots/R1_3}}
    \end{minipage} \right\rangle = q^{-1}
    \left\langle 
    \begin{minipage}{0.06\linewidth}
    \vspace{0pt}
    \resizebox{\linewidth}{!}{\begin{tikzpicture}
\begin{knot}[
consider self intersections,
clip width=5,
]
\strand[line width=3pt, black]
(-1, -1) to[out=60,in=180,looseness=1]
(0, 1) to[out=0,in=120,looseness=1] (1, -1);
\end{knot}
\end{tikzpicture}}
    \end{minipage} \right\rangle
\end{equation}

\begin{equation}
  \left\langle 
    \begin{minipage}{0.06\linewidth}
    \vspace{0pt}
    \resizebox{\linewidth}{!}{\begin{tikzpicture}
\begin{knot}[
consider self intersections,
clip width=5,
]
\strand[line width=3pt, black]
(-1, -1) to[out=45,in=-90,looseness=1]
(0.5, 0) to[out=90,in=-45,looseness=1] (-1, 1);
\strand[line width=3pt, black]
(1, -1) to[out=135,in=-90,looseness=1]
(-0.5, 0) to[out=90,in=-135,looseness=1] (1, 1);
\end{knot}
\end{tikzpicture}}
    \end{minipage} \right\rangle = -q
    \left\langle 
    \begin{minipage}{0.06\linewidth}
    \vspace{0pt}
    \resizebox{\linewidth}{!}{\input{../tikz knots/R2_2}}
    \end{minipage} \right\rangle
\end{equation}

\begin{equation}
  \left\langle 
    \begin{minipage}{0.06\linewidth}
    \vspace{0pt}
    \resizebox{\linewidth}{!}{\input{../tikz knots/R3_1}}
    \end{minipage} \right\rangle = 
    \left\langle 
    \begin{minipage}{0.06\linewidth}
    \vspace{0pt}
    \resizebox{\linewidth}{!}{\begin{tikzpicture}
\begin{knot}[
consider self intersections,
clip width=5,
]
\strand[line width=3pt, black]
(-1, 0) to[out=-30,in=180,looseness=1] 
(0, -0.5) to[out=0,in=-150,looseness=1] (1, 0);
\strand[line width=3pt, black]
(1, 1) to[out=-160,in=60,looseness=1] (-1, -1);
\strand[line width=3pt, black]
(-1, 1) to[out=-20,in=120,looseness=1] (1, -1);
\end{knot}
\end{tikzpicture}}
    \end{minipage} \right\rangle
\end{equation}

Hence, Kauffman bracket is not a knot invariant itself. Thus, we
introduce normalization coefficient that will fix invariance. In
oreder to do so we define positive and negative crossings on
oriented knot diagram:

\begin{equation}
  +: \ \
  \begin{minipage}{0.06\linewidth}
    \vspace{0pt}
    \resizebox{\linewidth}{!}{\input{../tikz knots/cross1+.tex}}
    \end{minipage} \ , \ \ -: \ \
    \begin{minipage}{0.06\linewidth}
    \vspace{0pt}
    \resizebox{\linewidth}{!}{\input{../tikz knots/cross2-.tex}}
    \end{minipage}
\end{equation}

The amount of positive and negative crossings on a knot diagram is $n_+$ и $n_-$ 
respectively. Let us multiply the result of Kauffman bracket by the
factor $(-1)^{a n_+ + b n_-}q^{c n_+ + d n_-}$ in order to achieve
invariance. This approach yields an invariant polynomial:

\begin{tcolorbox}
\begin{equation}
  \displaystyle
  J(q, L)=(-1)^{n_-}q^{n_+ - 2 n_-}\frac{\langle L \rangle}{\langle
    \begin{minipage}{0.03\linewidth}
    \vspace{0pt}
    \resizebox{\linewidth}{!}{\begin{tikzpicture}
\begin{knot}
\strand[line width=3pt, black] 
(0,0) circle[radius=1cm];
\end{knot}
\end{tikzpicture}}
    \end{minipage}
  \rangle}
  \label{eq:jones}
\end{equation}
\end{tcolorbox}

By Jones polynomial we will understand 
\eqref{eq:jones} \footnote{Usually Jones polynomial $\tilde{J}$
is defined as $\tilde{J}(t, L) := 
J(-\sqrt{t}, L)$}.

\subsection{Skein-relations}

Using \eqref{eq:skein-kaufman} for the opposite crossings we obtain:

\begin{equation}
\left\{
\begin{array}{l}
    \left\langle 
    \begin{minipage}{0.06\linewidth}
    \resizebox{\linewidth}{!}{\input{../tikz knots/cross1}}
    \end{minipage} \right\rangle = 
    \left\langle 
    \begin{minipage}{0.06\linewidth}
    \resizebox{\linewidth}{!}{\input{../tikz knots/R2_2}}
    \end{minipage} \right\rangle - q
    \left\langle 
    \begin{minipage}{0.06\linewidth}
    \resizebox{\linewidth}{!}{\input{../tikz knots/R2_2_rot}}
    \end{minipage} \right\rangle  ~~ |\cdot q^{-1}\\ \\
    
    \left\langle 
    \begin{minipage}{0.06\linewidth}
    \resizebox{\linewidth}{!}{\begin{tikzpicture}
\begin{knot}[
consider self intersections,
flip crossing = 1,
clip width=5,
]
\strand[line width=3pt, black]
(-1, -1) to[out=45,in=-135,looseness=1] (1, 1);
\strand[line width=3pt, black]
(-1, 1) to[out=-45,in=135,looseness=1] (1, -1);
\end{knot}
\end{tikzpicture}}
    \end{minipage} \right\rangle = 
    \left\langle 
    \begin{minipage}{0.06\linewidth}
    \resizebox{\linewidth}{!}{\input{../tikz knots/R2_2_rot}}
    \end{minipage} \right\rangle - q
    \left\langle 
    \begin{minipage}{0.06\linewidth}
    \resizebox{\linewidth}{!}{\input{../tikz knots/R2_2}}
    \end{minipage} \right\rangle
\end{array}
\right.
\;+\;
\end{equation}

\begin{equation}
  q^{-1} \left\langle 
    \begin{minipage}{0.06\linewidth}
    \resizebox{\linewidth}{!}{\input{../tikz knots/cross1}}
    \end{minipage} \right\rangle +
  \left\langle 
    \begin{minipage}{0.06\linewidth}
    \resizebox{\linewidth}{!}{\begin{tikzpicture}
\begin{knot}[
consider self intersections,
flip crossing = 1,
clip width=5,
]
\strand[line width=3pt, black]
(-1, -1) to[out=45,in=-135,looseness=1] (1, 1);
\strand[line width=3pt, black]
(-1, 1) to[out=-45,in=135,looseness=1] (1, -1);
\end{knot}
\end{tikzpicture}}
    \end{minipage} \right\rangle =
    (q^{-1} - q) \left\langle 
    \begin{minipage}{0.06\linewidth}
    \resizebox{\linewidth}{!}{\input{../tikz knots/R2_2}}
    \end{minipage} \right\rangle
    \label{eq:skein-kaufman-cross}
\end{equation}

In the same manner we can derive \eqref{eq:skein-kaufman-cross} 
from \eqref{eq:skein-kaufman}. Hence, this properties are equivalent. 
However \eqref{eq:skein-kaufman-cross}, 
in contrast to \eqref{eq:skein-kaufman}, is applied for oriented knots
 and with respect of normalization
\eqref{eq:jones} yields:

\begin{tcolorbox}
\begin{equation}
q^{-2} J\left (
  \begin{minipage}{0.06\linewidth}
    \resizebox{\linewidth}{!}{\input{../tikz knots/cross1+.tex}}
    \end{minipage}
\right ) - q^2 J\left (
  \begin{minipage}{0.06\linewidth}
    \resizebox{\linewidth}{!}{\input{../tikz knots/cross2-.tex}}
    \end{minipage}
\right ) = (q^{-1}-q) J\left (
  \begin{minipage}{0.06\linewidth}
    \resizebox{\linewidth}{!}{\begin{tikzpicture}
\begin{knot}[
consider self intersections,
clip width=5,
]
\strand[line width=3pt, ->, black]
(-1, -1) to[out=60,in=-90,looseness=1]
(-0.5, 0) to[out=90,in=-60,looseness=1] (-1, 1);
\strand[line width=3pt, ->, black]
(1, -1) to[out=120,in=-90,looseness=1]
(0.5, 0) to[out=90,in=-120,looseness=1] (1, 1);
\end{knot}
\end{tikzpicture}}
    \end{minipage}
\right )
\label{eq:skein}
\end{equation}
\end{tcolorbox}

Relation \eqref{eq:skein} combined with requirment $(J(L) = 1 ~~ \forall 
L \sim \begin{minipage}{0.03\linewidth}
    \vspace{0pt}
    \resizebox{\linewidth}{!}{\begin{tikzpicture}
\begin{knot}
\strand[line width=3pt, black] 
(0,0) circle[radius=1cm];
\end{knot}
\end{tikzpicture}}
    \end{minipage})$ 
    uniquely determine the Jones polynomial
     (see p.46 \cite{prasolov-sossinsky}).

\subsection{Questions}
\begin{itemize}
    \item  Can Kauffman polynomial be derived from Kauffman bracket?
    \item  What additional axioms are needed to define the Jones polynomial via the skein relation on crossings?
    \item  Add other powers to the left side of the skein relation. Will it still be an invariant?
    \item Jones is normalized on the unknot. How to normalize Khovanov homology?
\end{itemize}


\section{Khovanov homology as a categorification of the Jones polynomial}
\label{sec:Khovanov}
\subsection{Khovanov complex construction}
\subsection{Example: Hopf-link}

после этих слов старый текст
The statesum form (equation ...) of the Kauffman bracket allows the following interpretation \parencite{bar-natan}. Let us consider each circle in a completely smoothed diagram $\alpha \in \{0, 1\}^\chi$ as a graded vector space $V$ with grading $\{+, -\}$.  If $\alpha$ consists of $n$ circles, we associate it with $V^{\otimes n}$. Then we group complete smoothings according to the number of 1-resolutions $\gamma$ and consider these groups as a direct sum of vector spaces $V_\gamma = \bigoplus\limits_{\alpha(\gamma)} V^{\otimes n(\alpha)}$. The dimension of the resulting space is exactly equal to the $(-q)^\gamma$ prefactor in statesum.

\subsection{Khovanov complex construction}
\subsubsection{Diagram commutation proof}
In this section, we use space numbering method from \cite{bar-natan} and always assume that the spaces in the tensor product $V_{i_1} \otimes \cdots \otimes V_{i_n} \stackrel{\text{not}}{=} i_1 \otimes \cdots \otimes i_n$ are ordered such that $i_1<\cdots <i_n$ ($i_1 \equiv 1$ by numbering definition). We are to analyse diagrams $^{gf}_{hk}$

\[
\xymatrix{
 & B \ar[dr]^{f} & \\
A \ar[ur]^{g} \ar[dr]_{h} & & D \\
 & C \ar[ur]_{k} &
}
\]
,where $g, f, h, k \in {m, \Delta}$, and $ABDC$ --- a face of the hypercube.

\textbf{Case 1: $\mathbf{^{mm}_{mm}}$}

Only two essentially distinct types of diagrams exist:
\[
\begin{array}{c c}
    \left\{\begin{minipage}{0.12\linewidth}
      \resizebox{\linewidth}{!}{\begin{tikzpicture}[very thick]
    \draw (0,0) circle (0.4);
    \draw (1,0) circle (0.4);
    \draw (2,0) circle (0.4);
    \draw (3,0) circle (0.4);

    \draw[->] (0.5, -0.5) -- (0.5, -1) node[midway, right]{$m$};

    \draw (0.5,-1.5) circle (0.4);
    \draw (2,-1.5) circle (0.4);
    \draw (3,-1.5) circle (0.4);

    \draw[->] (2.5, -2) -- (2.5, -2.5) node[midway, right]{$m$};

    \draw (0.5,-3) circle (0.4);
    \draw (2.5,-3) circle (0.4);
\end{tikzpicture}}
  \end{minipage}\right\} = 
\left\{\begin{minipage}{0.12\linewidth}
      \resizebox{\linewidth}{!}{\input{../tikz diagrams/mm_mm2.tex}}
  \end{minipage}\right\} ~~&~~
  \left\{\begin{minipage}{0.09\linewidth}
      \resizebox{\linewidth}{!}{\input{../tikz diagrams/mm_mm3.tex}}
  \end{minipage}\right\} = 
\left\{\begin{minipage}{0.09\linewidth}
      \resizebox{\linewidth}{!}{\begin{tikzpicture}[very thick]
    \draw (0,0) circle (0.4);
    \draw (1,0) circle (0.4);
    \draw (2,0) circle (0.4);

    \draw[->] (1.5, -0.5) -- (1.5, -1) node[midway, right]{$m$};

    \draw (0,-1.5) circle (0.4);
    \draw (1.5,-1.5) circle (0.4);

    \draw[->] (0.75, -2) -- (0.75, -2.5) node[midway, right]{$m$};

    \draw (0.75,-3) circle (0.4);
\end{tikzpicture}}
  \end{minipage}\right\} \\
  \text{far commutativity} ~~& ~~\text{near commutativity}
\end{array}
\]

The first equality is obvious, while the second expresses the associativity property of multiplication:

\begin{equation}
    \boxed{
m(1\otimes m) = m(m\otimes 1) }
\label{eq:comm_1}
\end{equation}

\textbf{Case 2: $\mathbf{^{m\Delta}_{m\Delta}}$}

Diagram types:

\[
\begin{array}{c c}
    \left\{\begin{minipage}{0.12\linewidth}
      \resizebox{\linewidth}{!}{\begin{tikzpicture}[very thick]
    \draw[dashed] (0,0) circle (0.4);
    \draw (1,0) circle (0.4);
    \draw[dotted] (2.5,0) circle (0.4);

    \draw[->] (0.5, -0.5) -- (0.5, -1) node[midway, right]{$m$};

    \draw (0.5,-1.5) circle (0.4);
    \draw[dotted] (2.5,-1.5) circle (0.4);

    \draw[->] (2.5, -2) -- (2.5, -2.5) node[midway, right]{$\Delta$};

    \draw (0.5,-3) circle (0.4);
    \draw[dotted] (2,-3) circle (0.4);
    \draw[dotted] (3,-3) circle (0.4);
\end{tikzpicture}}
  \end{minipage}\right\} \not= 
\left\{\begin{minipage}{0.12\linewidth}
      \resizebox{\linewidth}{!}{\begin{tikzpicture}[very thick]
    \draw[dashed] (-0.5,0) circle (0.4);
    \draw (1,0) circle (0.4);
    \draw[dotted] (2,0) circle (0.4);

    \draw[->] (1.5, -0.5) -- (1.5, -1) node[midway, right]{$m$};

    \draw[dashed] (-0.5,-1.5) circle (0.4);
    \draw (1.5,-1.5) circle (0.4);

    \draw[->] (-0.5, -2) -- (-0.5, -2.5) node[midway, right]{$\Delta$};

    \draw[dashed] (-1,-3) circle (0.4);
    \draw[dashed] (0,-3) circle (0.4);
    \draw (1.5,-3) circle (0.4);
\end{tikzpicture}}
  \end{minipage}\right\} ~~&~~
  \left\{\begin{minipage}{0.06\linewidth}
      \resizebox{\linewidth}{!}{\input{../tikz diagrams/mD_mD3.tex}}
  \end{minipage}\right\} = 
\left\{\begin{minipage}{0.06\linewidth}
      \resizebox{\linewidth}{!}{\begin{tikzpicture}[very thick]
    \draw (0,0) circle (0.4);
    \draw (1,0) circle (0.4);

    \draw[->] (0.5, -0.5) -- (0.5, -1) node[midway, right]{$m_2$};

    \draw (0.5,-1.5) circle (0.4);

    \draw[->] (0.5, -2) -- (0.5, -2.5) node[midway, right]{$\Delta_2$};

    \draw (0,-3) circle (0.4);
    \draw (1,-3) circle (0.4);
\end{tikzpicture}}
  \end{minipage}\right\} \\
  \text{far commutativity} ~~& ~~\text{near commutativity}
\end{array}
\]

The first diagram never appears. To see this, note that on the right-hand side the resulting space consists of two smoothings with only dotted spatial edges, while on the left --- with only dashed ones. Consequently, the resulting spaces correspond to different vertices of the hypercube. 

The second equation is obvious. It may seem counterintuitive that we can resolve a smoothing using two different $\Delta \cdot m$ operators and still arrive at the same point, but the Hopf link example will likely dispel this doubt(see Fig. \ref{fig:hopf-diagram}).

\textbf{Case 3: $\mathbf{^{\Delta m}_{\Delta m}}$}

Similar to case 2, the far commutativity diagram does not appear:
\[
    \left\{\begin{minipage}{0.12\linewidth}
      \resizebox{\linewidth}{!}{\input{../tikz diagrams/Dm_Dm1.tex}}
  \end{minipage}\right\} \not= 
\left\{\begin{minipage}{0.12\linewidth}
      \resizebox{\linewidth}{!}{\input{../tikz diagrams/Dm_Dm2.tex}}
  \end{minipage}\right\}
\]

Now we have to consider two essentialy distinct near commutativity diagrams.

\[
\begin{array}{c c}
    \left\{\begin{minipage}{0.09\linewidth}
      \resizebox{\linewidth}{!}{\input{../tikz diagrams/Dm_Dm3.tex}}
  \end{minipage}\right\} = 
\left\{\begin{minipage}{0.09\linewidth}
      \resizebox{\linewidth}{!}{\begin{tikzpicture}[very thick]
    \draw (0,0) circle (0.4);
    \draw (1.5,0) circle (0.4);

    \draw[->] (1.5, -0.5) -- (1.5, -1) node[midway, right]{$\Delta$};

    \draw (0,-1.5) circle (0.4);
    \draw (1,-1.5) circle (0.4);
    \draw (2,-1.5) circle (0.4);

    \draw[->] (0.5, -2) -- (0.5, -2.5) node[midway, right]{$m$};

    \draw (0.5,-3) circle (0.4);
    \draw (2,-3) circle (0.4);
\end{tikzpicture}}
  \end{minipage}\right\} ~~&~~
  \left\{\begin{minipage}{0.06\linewidth}
      \resizebox{\linewidth}{!}{\input{../tikz diagrams/Dm_Dm5.tex}}
  \end{minipage}\right\} = 
\left\{\begin{minipage}{0.06\linewidth}
      \resizebox{\linewidth}{!}{\input{../tikz diagrams/Dm_Dm6.tex}}
  \end{minipage}\right\}
\end{array}
\]

The first diagram corresponds to the following chains ($P$ --- permutation operator):

\begin{equation}
    1\otimes i ~~ \left[\begin{array}{ccl}

    \xrightarrow{1 \otimes \Delta} &1\otimes i \otimes k &\left[
    \begin{array}{ccc}
        \xrightarrow{m \otimes 1}& 1 \otimes k&, ~k >i\\
        \xrightarrow{(m \otimes 1)(1 \otimes P)}& 1 \otimes i&
    \end{array}
    \right.\\[8mm]

    \xrightarrow{\Delta \otimes 1} &1\otimes k \otimes i &\left[
    \begin{array}{ccc}
        \xrightarrow{1 \otimes m}& 1 \otimes k&, ~k<i\\
        \xrightarrow{(m \otimes 1)(1 \otimes P)}& 1 \otimes k&, ~k<i
    \end{array}
    \right.\\[8mm]

    \xrightarrow{(1 \otimes P)(\Delta \otimes 1)} &1\otimes i \otimes k &\left[
    \begin{array}{ccc}
        \xrightarrow{m \otimes 1}& 1 \otimes k&, ~k>i\\
        \xrightarrow{1 \otimes m}& 1 \otimes i&
    \end{array}
    \right.
\end{array}
\right.
\label{eq:mD_chain}
\end{equation}

Thus the first diagram commutativity implies that $m$ and $\Delta$ satisfy the following equations:

\begin{equation}
(m\otimes 1)(1\otimes \Delta) = (m\otimes 1)(1 \otimes P)(\Delta \otimes 1)
\end{equation}
\begin{equation}
(m\otimes 1)(1\otimes P)(1\otimes\Delta) = (1\otimes m)(1 \otimes P)(\Delta \otimes 1)
\end{equation}
\begin{equation}
(1\otimes m)(\Delta\otimes 1) = (m\otimes 1)(1 \otimes P)(\Delta \otimes 1)
\end{equation}

The second equation is obvious. One can verify that such a diagram appears when the overlapping unknots diagram $\begin{minipage}{0.06\linewidth}
      \resizebox{\linewidth}{!}{\input{../tikz knots/0_2.tex}}
  \end{minipage}$ is smoothed.

\textbf{Case 4: $\mathbf{^{m\Delta}_{\Delta m}}$}

Diagram types:

\[
\begin{array}{c c}
    \left\{\begin{minipage}{0.12\linewidth}
      \resizebox{\linewidth}{!}{\input{../tikz diagrams/mD_Dm1.tex}}
  \end{minipage}\right\} = 
\left\{\begin{minipage}{0.12\linewidth}
      \resizebox{\linewidth}{!}{\begin{tikzpicture}[very thick]
    \draw (0,0) circle (0.4);
    \draw (1,0) circle (0.4);
    \draw (2.5,0) circle (0.4);

    \draw[->] (2.5, -0.5) -- (2.5, -1) node[midway, right]{$\Delta$};

    \draw (0,-1.5) circle (0.4);
    \draw (1,-1.5) circle (0.4);
    \draw (2,-1.5) circle (0.4);
    \draw (3,-1.5) circle (0.4);

    \draw[->] (0.5, -2) -- (0.5, -2.5) node[midway, right]{$m$};

    \draw (0.5,-3) circle (0.4);
    \draw (2,-3) circle (0.4);
    \draw (3,-3) circle (0.4);
\end{tikzpicture}}
  \end{minipage}\right\} ~~&~~
  \left\{\begin{minipage}{0.06\linewidth}
      \resizebox{\linewidth}{!}{\begin{tikzpicture}[very thick]
    \draw (0,0) circle (0.4);
    \draw (1,0) circle (0.4);

    \draw[->] (0.5, -0.5) -- (0.5, -1) node[midway, right]{$m$};

    \draw (0.5,-1.5) circle (0.4);

    \draw[->] (0.5, -2) -- (0.5, -2.5) node[midway, right]{$\Delta$};

    \draw (0,-3) circle (0.4);
    \draw (1,-3) circle (0.4);
\end{tikzpicture}}
  \end{minipage}\right\} = 
\left\{\begin{minipage}{0.09\linewidth}
      \resizebox{\linewidth}{!}{\input{../tikz diagrams/Dm_Dm3.tex}}
  \end{minipage}\right\} \\
  \text{far commutativity} ~~& ~~\text{near commutativity}
\end{array}
\]

The first equation is obvious. The second one implies that any chain from (\ref{eq:mD_chain}) is equal to $\Delta\cdot m$:

\begin{equation}
\boxed{
\begin{array}{c}
\Delta m = (m\otimes 1)(1\otimes \Delta) = (1\otimes m)(\Delta\otimes 1)=\\
=(m\otimes 1)(1 \otimes P)(\Delta \otimes 1)=\\
=(m\otimes 1)(1\otimes P)(1\otimes\Delta) =\\
=(m\otimes 1)(1 \otimes P)(\Delta \otimes 1)=\\ 
=(1\otimes m)(1 \otimes P)(\Delta \otimes 1)
\end{array}
}
\label{eq:comm_2}
\end{equation}



\textbf{Case 5: $\mathbf{^{\Delta\Delta}_{\Delta \Delta}}$}

The diagrams are analogous to those in case 1: 

\[
\begin{array}{c c}
    \left\{\begin{minipage}{0.12\linewidth}
      \resizebox{\linewidth}{!}{\input{../tikz diagrams/DD_DD1.tex}}
  \end{minipage}\right\} = 
\left\{\begin{minipage}{0.12\linewidth}
      \resizebox{\linewidth}{!}{\begin{tikzpicture}[very thick]
    \draw (0,0) circle (0.4);
    \draw (2,0) circle (0.4);

    \draw[->] (2, -0.5) -- (2, -1) node[midway, right]{$\Delta$};

    \draw (0,-1.5) circle (0.4);
    \draw (1.5,-1.5) circle (0.4);
    \draw (2.5,-1.5) circle (0.4);

    \draw[->] (0, -2) -- (0, -2.5) node[midway, right]{$\Delta$};

    \draw (-0.5,-3) circle (0.4);
    \draw (0.5,-3) circle (0.4);
    \draw (1.5,-3) circle (0.4);
    \draw (2.5,-3) circle (0.4);
\end{tikzpicture}}
  \end{minipage}\right\} ~~&~~
  \left\{\begin{minipage}{0.09\linewidth}
      \resizebox{\linewidth}{!}{\begin{tikzpicture}[very thick]
    \draw (0,0) circle (0.4);

    \draw[->] (0, -0.5) -- (0, -1) node[midway, right]{$\Delta$};

    \draw (-0.75,-1.5) circle (0.4);
    \draw (0.75,-1.5) circle (0.4);

    \draw[->] (0.75, -2) -- (0.75, -2.5) node[midway, right]{$\Delta$};

    \draw (-0.75,-3) circle (0.4);
    \draw (0.25,-3) circle (0.4);
    \draw (1.25,-3) circle (0.4);
\end{tikzpicture}}
  \end{minipage}\right\} = 
\left\{\begin{minipage}{0.09\linewidth}
      \resizebox{\linewidth}{!}{\input{../tikz diagrams/DD_DD4.tex}}
  \end{minipage}\right\} \\
  \text{far commutativity} ~~& ~~\text{near commutativity}
\end{array}
\]


The near commutativity expresses the coassociativity of comultiplication:

\begin{equation}
    \boxed{
        (1\otimes \Delta)\Delta = (\Delta\otimes 1)\Delta
    }
    \label{eq:comm_3}
\end{equation}

Now identities (\ref{eq:comm_1}), (\ref{eq:comm_2}), (\ref{eq:comm_3}) can be explicitly verified using the definition of $m$ and $\Delta$.


\subsection{Example: positive Hopf link}

\begin{wrapfigure}{r}{0.2\textwidth}
  \begin{center}
    \begin{minipage}{\linewidth}
      \resizebox{\linewidth}{!}{\begin{tikzpicture}
  \begin{knot}[
    clip width=5,
    flip crossing/.list={1}
  ]
    % первая окружность по дугам
    \strand[line width=3pt,->,black] 
      (0,0) arc[start angle=0,end angle=360,radius=1];
    
    % вторая окружность по дугам
    \strand[line width=3pt,<-,black] 
      (-1,0) arc[start angle=0,end angle=360,radius=-1];
  \end{knot}
\end{tikzpicture}}
  \end{minipage} 
    $n_{+} = 2$

  $n_{-} = 0$
  \end{center}
\end{wrapfigure}

\[
\mathcal{H}^0 = \Ker d_{01} = \langle v_- \otimes v_-, v_- \times v_+ - v_+ \times v_-\rangle \{2\} ~~ \Rightarrow ~~ \qdim\mathcal{H}^0 = 1 + q^2
\]

\[
\mathcal{H}^1 = \Ker d_{12}/\Im d_{01} = \frac{\langle v_{+(1)} + v_{+(2)}, v_{-(1)} + v_{-(2)}\rangle}{\langle v_{+(1)} + v_{+(2)}, v_{-(1)} + v_{-(2)}\rangle}\{3\} ~~ \Rightarrow ~~ \qdim\mathcal{H}^1 = 0
\]

\[
\mathcal{H}^2 = V/\Im d_{12} = (V/\langle v_- \otimes v_-, v_- \times v_+ + v_+ \times v_-\rangle)\{4\} ~~ \Rightarrow ~~ \qdim\mathcal{H}^1 =  q^4 + q^6
\]

\[\boxed{Kh(\begin{minipage}{0.06\linewidth}
      \resizebox{\linewidth}{!}{\begin{tikzpicture}
  \begin{knot}[
    clip width=5,
    flip crossing/.list={1}
  ]
    % первая окружность по дугам
    \strand[line width=3pt,->,black] 
      (0,0) arc[start angle=0,end angle=360,radius=1];
    
    % вторая окружность по дугам
    \strand[line width=3pt,<-,black] 
      (-1,0) arc[start angle=0,end angle=360,radius=-1];
  \end{knot}
\end{tikzpicture}}
  \end{minipage} ) = 1 + q^2 + t^2 q^4 + t^2 q^6}
\]

\begin{wrapfigure}{r}{0pt}
\end{wrapfigure}

\begin{figure}[h]
    \begin{center}
    \begin{tikzpicture}
    \node at (-0.5, 0) {0};
    \draw[->] (0,0) -- (1,0) node[midway, above] {$d_{00}$};

    \node[inner sep=0pt] at (2.5,0) {
    \resizebox{2cm}{!}{
        \begin{tikzpicture}
            \begin{knot}[clip width=5]
                \strand[line width=2pt, black]
                    (-0.2, 0.7) to[out=90,in=90,looseness=1]
                    (-1.5, 0) to[out=-90,in=-90,looseness=1]
                    (-0.2, -0.7) to[out=90,in=-90,looseness=1]
                    (-0.7, 0) to[out=90,in=-90,looseness=1] (-0.2, 0.7);

                \strand[line width=2pt, black]
                    (0.2, 0.7) to[out=90,in=90,looseness=1]
                    (1.5, 0) to[out=-90,in=-90,looseness=1]
                    (0.2, -0.7) to[out=90,in=-90,looseness=1]
                    (0.7, 0) to[out=90,in=-90,looseness=1] (0.2, 0.7);
            \end{knot}
        \end{tikzpicture}
    }
    };

    \draw[->] (4,0.5) -- (5,1.5) 
    node[pos=0.25, right = 3pt] {$d_{\bigstar 0}$}
    node[pos=0.5, left = 3pt] {$m$};
    \draw[->] (4,-0.5) -- (5,-1.5) 
    node[pos=0.25, right = 3pt] {$d_{0 \bigstar}$}
    node[pos=0.5, left = 3pt] {$m$};

    \node[inner sep=0pt] at (6.5, 1.5) {
    \resizebox{2cm}{!}{
        \begin{tikzpicture}
            \begin{knot}[clip width=5]
                \strand[line width=2pt, black]
                    (0, 0.8) to[out=180,in=0,looseness=1]
                    (-0.7, 1) to[out=180,in=180,looseness=1]
                    (-0.7, -1) to[out=0,in=-90,looseness=1]
                    (-0.2, -0.7) to[out=90,in=180,looseness=1]
                    (0, 0.4) to[out=0,in=90,looseness=1]
                    (0.2, -0.7) to[out=-90,in=180,looseness=1]
                    (0.7, -1) to[out=0,in=0,looseness=1]
                    (0.7, 1) to[out=180,in=0,looseness=1] (0, 0.8);
            \end{knot}
        \end{tikzpicture}
    }
    };

    \node[inner sep=0pt] at (6.5, -1.5) {
    \resizebox{2cm}{!}{
        \begin{tikzpicture}
            \begin{knot}[clip width=5]
                \strand[line width=2pt, black]
                    (0, -0.8) to[out=180,in=0,looseness=1]
                    (-0.7, -1) to[out=180,in=180,looseness=1]
                    (-0.7, 1) to[out=0,in=90,looseness=1]
                    (-0.2, 0.7) to[out=-90,in=180,looseness=1]
                    (0, -0.4) to[out=0,in=-90,looseness=1]
                    (0.2, 0.7) to[out=90,in=180,looseness=1]
                    (0.7, 1) to[out=0,in=0,looseness=1]
                    (0.7, -1) to[out=180,in=0,looseness=1] (0, -0.8);
            \end{knot}
        \end{tikzpicture}
    }
    };

    \draw[->] (8,1.5) -- (9,0.5) 
    node[pos=0.75, left = 3pt] {$d_{1 \bigstar}$}
    node[pos=0.25, right = 3pt] {$-\Delta$};
    \draw[->] (8,-1.5) -- (9,-0.5) 
    node[pos=0.75, left = 3pt] {$d_{\bigstar 1}$}
    node[pos=0.25, right = 3pt] {$\Delta$};

    \node[inner sep=0pt] at (10.5, 0) {
    \resizebox{2cm}{!}{
        \begin{tikzpicture}
            \begin{knot}[clip width=5]
                \strand[line width=2pt, black]
                    (0, 0.8) to[out=180,in=0,looseness=1]
                    (-0.7, 1) to[out=180,in=180,looseness=1]
                    (-0.7, -1) to[out=0,in=180,looseness=1]
                    (0, -0.8) to[out=0,in=180,looseness=1]
                    (0.7, -1) to[out=0,in=0,looseness=1]
                    (0.7, 1) to[out=180,in=0,looseness=1](0, 0.8);
                \strand[line width=2pt, black] 
                    (0,0) circle[radius=0.5cm];
            \end{knot}
        \end{tikzpicture}
    }
    };

    \draw[->] (12,0) -- (13,0) node[midway, above] {$d_{11}$};
    \node at (13.5, 0) {0};


    \draw[blue] (3.7,2) rectangle (5.3,-2);
    \draw[->, blue] (4,-2.7) -- (5,-2.7) node[midway, above, blue]{$d_{01}$};

    \draw[blue] (7.7,2) rectangle (9.3,-2);
    \draw[->, blue] (8,-2.7) -- (9,-2.7) node[midway, above, blue]{$d_{12}$};

    \node at (2.5, -2.7) {\scriptsize $V_0 = V^{\otimes 2} \{0\} \textcolor{red}{\{2\}}$};
    \node at (6.5, -2.7) {\scriptsize $V_1 = V^{\oplus 2} \{1\} \textcolor{red}{\{2\}}$};
    \node at (10.5, -2.7) {\scriptsize $V_2 = V^{\otimes 2} \{2\} \textcolor{red}{\{2\}}$};

\end{tikzpicture}
    \caption{Positive Hopf link Khovanov complex. $\textcolor{red}{\{n_{+}-2n_{-}\} = \{2\}}$ --- common degree shift}
    \label{fig:hopf-diagram}
    \end{center}
\end{figure}

\subsubsection{R1-invariance}
The following argument applies to both the left and right R1 moves.
\begin{tcolorbox}
\begin{proposition}
    Let $C$ be a bicomplex concentrated in two rows:
    \begin{center}
    \begin{minipage}{0.3\linewidth}
      \resizebox{\linewidth}{!}{\begin{tikzpicture}
    \node at (0, 0) {0};
    \node at (0, -2) {0};

    \draw[->] (0.5,0) -- (1.5,0);
    \draw[->] (0.5,-2) -- (1.5,-2);

    \node at (2,2) {0};
    \draw[->] (2,1.5) -- (2,0.5);
    \node at (2, 0) {$C_{0, 0}$};
    \draw[->] (2,-0.5) -- (2,-1.5) node[midway, right]{$d^v_{0, 0}$};
    \node at (2, -2) {$C_{0, 1}$};
    \draw[->] (2,-2.5) -- (2,-3.5);
    \node at (2, -4) {0};

    \draw[->] (2.5,0) -- (3.5,0) node[midway, above]{$d^h_{0, 0}$};
    \draw[->] (2.5,-2) -- (3.5,-2) node[midway, above]{$d^h_{0, 1}$};

    \node at (4,2) {0};
    \draw[->] (4,1.5) -- (4,0.5);
    \node at (4, 0) {$C_{1, 0}$};
    \draw[->] (4,-0.5) -- (4,-1.5) node[midway, right]{$d^v_{1, 0}$};
    \node at (4, -2) {$C_{1, 1}$};
    \draw[->] (4,-2.5) -- (4,-3.5);
    \node at (4, -4) {0};

    \draw[->] (4.5,0) -- (5.5,0) node[midway, above]{$d^h_{1, 0}$};
    \draw[->] (4.5,-2) -- (5.5,-2) node[midway, above]{$d^h_{1, 1}$};

    \node at (6,2) {0};
    \draw[->] (6,1.5) -- (6,0.5);
    \node at (6, 0) {$C_{2, 0}$};
    \draw[->] (6,-0.5) -- (6,-1.5) node[midway, right]{$d^v_{2, 0}$};
    \node at (6, -2) {$C_{2, 1}$};
    \draw[->] (6,-2.5) -- (6,-3.5);
    \node at (6, -4) {0};

    \draw[->] (6.5,0) -- (7.5,0) node[midway, above]{$d^h_{2, 0}$};
    \draw[->] (6.5,-2) -- (7.5,-2) node[midway, above]{$d^h_{2, 1}$};

    \node at (8, 0) {$\cdots$};
    \node at (8, -2) {$\cdots$};
\end{tikzpicture}}
  \end{minipage}
\end{center}
    If each vertical complex $0 \rightarrow C_{n, 0}\xrightarrow{d^v_{n, 0}} C_{n, 1} \rightarrow 0$ is acyclic, then the total complex $\mathrm{Tot}(C)$ ($0 \rightarrow Tot(C)_0 = C_{0, 0} \xrightarrow{d_0} Tot(C)_1 = C_{1, 0} \oplus C_{0, 1} \xrightarrow{d_1}\cdots$) is also acyclic.
\end{proposition}
\end{tcolorbox}
$\triangleleft$
From the acyclicity of the vertical complexes we obtain that $C_{n, 0} \overset{d^v_{n, 0}}{\cong} C_{n, 1}$. Hence, in view of the bicomplex anticommutativity $d^v_{n+1, m} \circ d^h_{n, m} + d^h_{n, m+1} \circ d^v_{n, m}=0$, we conclude that the bicomplex can be redrawn in the following way:

\begin{center}
\begin{minipage}{0.3\linewidth}
      \resizebox{\linewidth}{!}{\begin{tikzpicture}
    \node at (0, 0) {0};
    \node at (0, -2) {0};

    \draw[->] (0.5,0) -- (1.5,0);
    \draw[->] (0.5,-2) -- (1.5,-2);

    \node at (2,2) {0};
    \draw[->] (2,1.5) -- (2,0.5);
    \node at (2, 0) {$C_0$};
    \node[rotate=90] at (2,-1) {$\cong$};
    \node at (2.5, -1){$d^v_0$};
    \node at (2, -2) {$C_0$};
    \draw[->] (2,-2.5) -- (2,-3.5);
    \node at (2, -4) {0};

    \draw[->] (2.5,0) -- (3.5,0) node[midway, above]{$d^h_0$};
    \draw[->] (2.5,-2) -- (3.5,-2) node[midway, above]{$-d^h_0$};

    \node at (4,2) {0};
    \draw[->] (4,1.5) -- (4,0.5);
    \node at (4, 0) {$C_1$};
    \node[rotate=90] at (4,-1) {$\cong$};
    \node at (4.5, -1){$d^v_1$};
    \node at (4, -2) {$C_1$};
    \draw[->] (4,-2.5) -- (4,-3.5);
    \node at (4, -4) {0};

    \draw[->] (4.5,0) -- (5.5,0) node[midway, above]{$d^h_1$};
    \draw[->] (4.5,-2) -- (5.5,-2) node[midway, above]{$-d^h_1$};

    \node at (6,2) {0};
    \draw[->] (6,1.5) -- (6,0.5);
    \node at (6, 0) {$C_2$};
    \node[rotate=90] at (6,-1) {$\cong$};
    \node at (6.5, -1){$d^v_2$};
    \node at (6, -2) {$C_2$};
    \draw[->] (6,-2.5) -- (6,-3.5);
    \node at (6, -4) {0};

    \draw[->] (6.5,0) -- (7.5,0) node[midway, above]{$d^h_2$};
    \draw[->] (6.5,-2) -- (7.5,-2) node[midway, above]{$-d^h_2$};

    \node at (8, 0) {$\cdots$};
    \node at (8, -2) {$\cdots$};
\end{tikzpicture}}
  \end{minipage}
\end{center}

From this we have $\Ker(d_n) = \Im(d_{n-1}) = \{(c, d^h_{n - 1}(c)) | c \in C_{n - 1}\} ~\Rightarrow ~ H_n(Tot(C)) = 0$ $\triangleright$
\vspace{1cm}

The above proposition justifies the total complex factorizations used in \textcite{bar-natan}.


\section{$\mathcal{R}$-matrix construction of Jones and HOMFLY polynomials}

Along with taking the statistical sum, there is another approach to constructing knot invariants. It relies on the fact that any knot can be represented as a closed braid (\ref{th:braid}). This suggests searching for an invariant as a linear braid-group representation or, if we want to get a number or polynomial rather than a matrix, as the trace of such a representation.

In this section, we provide an off-the-shelf construction for the desired invariant. It is built on the basis of the quantized universal enveloping algebra $U_q$ for a simple Lie algebra $\mathfrak{g}$. The universal $\mathcal{R}$-matrix -- the operator in $U_q \otimes U_q$ -- then provides the braid group representation, while the quantum trace, taken via the $\mathcal{Q}$-operator, gives the knot invariant up to a compensating factor. Further intuition behind this comprehensive structure is set out in \parencite{anokhina}.

\subsection{Quantized universal enveloping algebra $U_q(\mathfrak{g})$}



\subsection{Implicit formulas for the crossing operator $\mathcal{R}$ and the turning operator $\mathcal{Q}$}


\subsection{Gauge-invariants by knot in braid representation}
\begin{wrapfigure}{r}{0.15\textwidth}
    \centering
  \includegraphics[width = 0.9\linewidth]{../img/braid-closure.png}
  \caption{Eight-knot as a braid closure \small{\parencite{ohtsuki}}}
  \label{fig:braid-closure}
\end{wrapfigure}

The closure of a braid is the link obtained from the braid by connecting upper ends and lower ends respectively as shown in Fig. \ref{fig:braid-closure}. The following theorem
assures us that such an expression always exists.
\begin{tcolorbox}
\begin{theorem}\label{thm:Alexander}
   (Alexander). Any (oriented) link is isotopic to the closure of some
braid (with downward orientation).  
\end{theorem}
\end{tcolorbox} 

However, as our intuition suggests, knots and braids have similar but not identical natures, so a single knot can be represented by a set of different braids. Nevertheless, this equivalence relation can be reformulated in terms of braid algebra operations, as stated by the following theorem.

\begin{wrapfigure}{r}{0pt}
\end{wrapfigure}

\begin{tcolorbox}
\begin{theorem}\label{thm:Alexander}
   (Markov).  Let $b$ and $b'$ be two braids, and $L$ and $L'$ their closures.
Then, $L$ is isotopic to $L'$ as oriented links if and only if $b$ is related to $b'$ by a
sequence of the following MI and MII moves.

\begin{center}
\begin{tabular}{c c}
\textbf{MI:} & \textbf{MII:} \\
$ab \longleftrightarrow ba$ & $b \sigma_n \longleftrightarrow b \longleftrightarrow b \sigma_n^{-1}$ \\[2mm]  
\includegraphics[height=3cm]{../img/MI.png} &
\includegraphics[height=3cm]{../img/MII.png}
\end{tabular}
\end{center}

\end{theorem}

\end{tcolorbox} 


As 
\subsection{$\mathfrak{sl}_2$ -- Jones polynomial}
\subsubsection{Implicit formulas for operators}
\subsubsection{Example: positive trefoil}
\subsubsection{Skein-relations}

\subsection{$\mathfrak{sl}_N$ -- HOMFLY polynomial}
\subsubsection{Implicit formulas for operators}
\subsubsection{Example: positive trefoil}
\subsubsection{Skein-relations}


Remark: given approach appears naturaly 


после этих слов старый текст
\subsection{HOMFLY definition by $\mathcal{R}$-matrices braid group representation}



\subsubsection{$\mathcal{R}$-matrices}

\begin{tcolorbox}
\begin{definition}
    Let $\mathfrak{g}$ be a simple finite-dimensional Lie algebra over $\mathbb{C}$. 
The \emph{quantized universal enveloping algebra} $U_q(\mathfrak{g})$ is an associative algebra over $\mathbb{C}(q)$ generated by elements
\[
\{ E_i, F_i, K_i^{\pm 1} \mid i = 1, \dots, \mathrm{rank}(\mathfrak{g}) \},\] 
where $E_i$ and $F_i$ --- simple positive and negative roots, respectively; $K_i = q^{d_i H_i}$ --- quantization of the Cartan element $H_i$ with symmetrizing factor $d_i = \frac{K(h_i, h_j)}{2}$. $A = (a_{ij})_{ij}$ --- Cartan matrix.

Commutation relations:

\end{definition}
\end{tcolorbox}



\begin{equation}
    \mathcal{R}:V_1 \otimes V_2 \rightarrow V_2 \otimes V_1, \ \ \mathbb{P} -
    \text{обычный оператор перестановки}
\end{equation}

Условие перестановочности (инвариантности):
\begin{equation}
    \forall g \in \mathfrak{g} \ \ \ \ \Delta g \mathcal{R} = \mathcal{R}
    \mathbb{P} \Delta g \mathbb{P}^{-1}
\end{equation}

Уравнения Янга-Бакстера:
\begin{equation}
    (\mathcal{R} \otimes \mathbb{I})(\mathbb{I} \otimes \mathcal{R})
    (\mathcal{R} \otimes \mathbb{I}) = (\mathbb{I} \otimes \mathcal{R})
    (\mathcal{R} \otimes \mathbb{I}) (\mathbb{I} \otimes \mathcal{R})
    \label{eq:YB}
\end{equation}


\section{Cobordism-based approach to Khovanov homology (\parencite{bar-natan-cob})}
эмммммм (скип??? я это не понимаю)

\section{Reduction of Khovanov complexes for antiparallel lock and linear representations of cobordisms}

Motivation: R-matrix reduction (ленина статья)
\subsection{Tangle Khovanov complex reduction for 4 types of antiparallel locks}
\subsection{Examples}
1) Hopf-link
2) see bar-natan (ссылка)
\subsection{Unsuccessfull attempt to translate cobordism-approach on linear operators}

\section{Conclusion}

\printbibliography




\end{document}