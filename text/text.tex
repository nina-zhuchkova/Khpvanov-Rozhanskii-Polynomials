\documentclass[12pt,a4paper]{article}
%%% Работа с русским языком
\usepackage{cmap}					% поиск в PDF
\usepackage{mathtext} 				% русские буквы в формулах
\usepackage[T2A]{fontenc}			% кодировка
\usepackage[utf8]{inputenc}			% кодировка исходного текста
\usepackage[english,russian]{babel}	% локализация и переносы

%%% Дополнительная работа с математикой
\usepackage{amsmath,amsfonts,amssymb,amsthm,mathtools} % AMS
\usepackage{icomma}
\usepackage{physics}
\usepackage{multicol}
\usepackage{bm}
\usepackage{mathrsfs}
\usepackage{verbatim}

%%% Номера формул
%\mathtoolsset{showonlyrefs=true} 
%\usepackage{leqno} 

%%% Свои команды
\DeclareMathOperator{\sgn}{sgn}

\usepackage{csquotes} 
\usepackage[backend=biber,style=authoryear,language=auto]{biblatex}


%%% Работа с графикой
\usepackage{graphicx}
\graphicspath{{images/}}  
\setlength\fboxsep{3pt} 
\setlength\fboxrule{1pt} 
\usepackage{wrapfig} 
\usepackage{tikz}
\usepackage{pgfplots}
\usepackage{pgfplotstable}
\usepgfplotslibrary{polar}
\pgfplotsset{compat=1.18} 

%%% Работа с таблицами
\usepackage{array,tabularx,tabulary,booktabs}
\usepackage{longtable}  
\usepackage{multirow} 
\usepackage{caption2}[2008/03/29]
\usepackage{soul} 

%%% Теоремы
\theoremstyle{plain} 
\newtheorem{theorem}{Th}[section]
\newtheorem{proposition}[theorem]{Proposition}
 
\theoremstyle{definition} 
\newtheorem{corollary}{Corollary}[theorem]
\newtheorem{problem}{Problem}[section]
\newtheorem{definition}{Def}[section]
 
\theoremstyle{remark} 
\newtheorem*{nonum}{Solution}

%%% Программирование
\usepackage{etoolbox} 

%%% Гиперссылки
\usepackage{hyperref}

\usetikzlibrary{knots}
\usepackage{tcolorbox}

%%% Страница
\usepackage{geometry} 
	\geometry{top=20mm, bottom=20mm, left=15mm, right=20mm}

\usepackage{setspace} 

\usepackage{lastpage} 
\usepackage{amssymb}
\usepackage{xcolor}
\DeclareMathOperator{\Ker}{Ker}
\DeclareMathOperator{\qdim}{qdim}

\begin{document}

\begin{center}
    \Large \textbf{Полиномы Хованова-Рожанского} \\[1em]
    \small
    Московский Физико-Технический Институт \\[0.5em]
    Лаборатория Математической и Теоретической Физики
\end{center}

\small
\begin{flushright}
\begin{tabular}{c c}
\textbf{Авторы:} & \textbf{Научные руководители:} \\[0.5em]
Артем Новохатний & Елена Ланина \\
Екатерина Цыганкова & Радомир Степанов \\
Эльдар Мифтахов & \\
\end{tabular}

\vspace{1em}

\today
\end{flushright}
\normalsize
\vspace{2em}


\tableofcontents
\vspace{2em}

\section{Полиномы Джонса}

\subsection{Вопросы}
\begin{itemize}
    \item  Можно ли из скобки Кауффмана получить полином Кауффмана?
    \item  Какие допаксиомы нужны, чтобы определить полином Джонса через скейн-соотношение на перекрестки?
    \item  Добавить другие степени в левой части скейн-соотношения. Будет ли это ещё инвариантом?
    \item Джонс нормируется на аннот. Как нормировать Ховановых?
\end{itemize}

\section{$\mathcal{R}$-матрицы}

\begin{equation}
    \mathcal{R}:V_1 \otimes V_2 \rightarrow V_2 \otimes V_1, \ \ \mathbb{P} -
    \text{обычный оператор перестановки}
\end{equation}

Условие перестановочности (инвариантности):
\begin{equation}
    \forall g \in \mathfrak{g} \ \ \ \ \Delta g \mathcal{R} = \mathcal{R}
    \mathbb{P} \Delta g \mathbb{P}^{-1}
\end{equation}

Уравнения Янга-Бакстера:
\begin{equation}
    (\mathcal{R} \otimes \mathbb{I})(\mathbb{I} \otimes \mathcal{R})
    (\mathcal{R} \otimes \mathbb{I}) = (\mathbb{I} \otimes \mathcal{R})
    (\mathcal{R} \otimes \mathbb{I}) (\mathbb{I} \otimes \mathcal{R})
    \label{eq:YB}
\end{equation}

\begin{theorem}
    Если выполняется условие перестановочности, то выполняются и
    уравнения Янга-Бакстера
\end{theorem}

Доказательство:

Разложим $\Delta g$:

\begin{equation}
    \Delta g = \sum g_{(1)} \otimes g_{(2)} 
    \label{eq:gexp}
\end{equation}

Тензорный куб определяется следующим образом (тут сразу показано также
условие коассоциативности):

\begin{equation}
    \Delta^{\circ 2} g = (\Delta \otimes \mathbb{I})\Delta g =
    (\mathbb{I} \otimes \Delta) \Delta g
    \label{eq:coprod-cube}
\end{equation}

Используем \eqref{eq:gexp} и введём следующее обозначение:
\begin{equation}
    \Delta^{\circ 2} g = (\Delta \otimes \mathbb{I})\Delta g = \sum \Delta (g_{(1)}) 
    \otimes g_{(2)} = \sum g_{(1,1)} \otimes g_{(1,2)} \otimes g_{(2)}
\end{equation}

Теперь посмотрим на следующее:
\begin{equation}
    ((\Delta \otimes \mathbb{I})\Delta g) \circ (\mathcal{R} \otimes \mathbb{I}) = 
    \sum \Delta(g_{(1)}) \mathcal{R} \otimes g_{(2)} =
    (\mathcal{R} \otimes \mathbb{I}) \sum \mathbb{P} \Delta(g_{(1)})
    \mathbb{P}^{-1} \otimes g_{(2)}
    \label{eq:trnsp-start}
\end{equation}

Посмотрим внимательнее на $\mathbb{P} \Delta(g_{(1)})\mathbb{P}^{-1}$:
\begin{equation}
    \mathbb{P} \Delta(g_{(1)})\mathbb{P}^{-1} = \mathbb{P} (g_{(1,1)} \otimes g_{(1,2)})
    \mathbb{P}^{-1} = g_{(1,2)} \otimes g_{(1,1)}
\end{equation}

Возвращаясь к \eqref{eq:trnsp-start}, получаем:
\begin{equation}
    ((\Delta \otimes \mathbb{I})\Delta g) \circ (\mathcal{R} 
    \otimes \mathbb{I}) = (\mathcal{R} 
    \otimes \mathbb{I}) \sum g_{(1,2)} \otimes g_{(1,1)} \otimes g_{(2)}
\end{equation}

Проделывая аналогичные рассуждения получаем:
\begin{equation}
    ((\Delta \otimes \mathbb{I})\Delta g) \circ (\mathcal{R} 
    \otimes \mathbb{I})(\mathbb{I} \otimes \mathcal{R})
    (\mathcal{R} \otimes \mathbb{I}) = (\mathcal{R} 
    \otimes \mathbb{I})(\mathbb{I} \otimes \mathcal{R})
    (\mathcal{R} \otimes \mathbb{I}) \sum g_{(2)} \otimes g_{(1,2)} \otimes g_{(1,1)}
    \label{eq:YB-1}
\end{equation}

Аналогично получаем:
\begin{equation}
    ((\mathbb{I} \otimes \Delta)\Delta g) \circ (\mathbb{I} \otimes \mathcal{R})
    (\mathcal{R} \otimes \mathbb{I}) (\mathbb{I} \otimes \mathcal{R}) = 
    (\mathbb{I} \otimes \mathcal{R})
    (\mathcal{R} \otimes \mathbb{I}) (\mathbb{I} \otimes \mathcal{R}) 
    \sum g_{(2, 2)} \otimes g_{(2,1)} \otimes g_{(1)}
    \label{eq:YB-2}
\end{equation}

Введём обозначения:
\begin{equation}
    A = (\mathcal{R} 
    \otimes \mathbb{I})(\mathbb{I} \otimes \mathcal{R})
    (\mathcal{R} \otimes \mathbb{I})
\end{equation}
\begin{equation}
    B = (\mathbb{I} \otimes \mathcal{R})
    (\mathcal{R} \otimes \mathbb{I}) (\mathbb{I} \otimes \mathcal{R})
\end{equation}

В силу коассоциативности:
\begin{equation}
    \sum g_{(1, 1)} \otimes g_{(1,2)} \otimes g_{(2)} = 
    \sum g_{(1)} \otimes g_{(2,1)} \otimes g_{(2, 2)} =
    \Delta^{\circ 2} g
\end{equation}

Значит \eqref{eq:YB-1} и \eqref{eq:YB-2} переписываются так:
\begin{equation}
    \Delta^{\circ 2} g \circ A = A \circ \Delta^{\circ 2} g
\end{equation}
\begin{equation}
    \Delta^{\circ 2} g \circ B = B \circ \Delta^{\circ 2} g
\end{equation}

Получается операторы $A$ и $B$ коммутируют со всеми элементами
тензорного куба. Осталось разобрать случаи, когда тензорный куб
является неприводимым представлением, и когда - приводимым.

Если тензорный куб неприводим, то по лемме Шура операторы $A$ и $B$
могут отличатся только домножением на константу:
\begin{equation}
    A = \lambda B
    \label{eq:Shur-cons}
\end{equation}

Воспользуемся тем, что $\mathcal{R}^{ii}_{ii} = q$:
\begin{equation}
    A(\upsilon_+ \otimes \upsilon_+ \otimes \upsilon_+) = q^3 (\upsilon_+ \otimes \upsilon_+ \otimes \upsilon_+)
\end{equation}
\begin{equation}
    B(\upsilon_+ \otimes \upsilon_+ \otimes \upsilon_+) = q^3 (\upsilon_+ \otimes \upsilon_+ \otimes \upsilon_+)
\end{equation}

Значит $\lambda = 1$ и теорема доказана.

Если же тензорный куб приводим, то \eqref{eq:Shur-cons} запишется
для каждой неприводимой компоненты со своим $\lambda_i$. (надо бы
как-то показать, что эти лямбды - единички)


\printbibliography

\end{document}