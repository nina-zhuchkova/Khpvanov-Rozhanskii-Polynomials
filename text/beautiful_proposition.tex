\documentclass[12pt,a4paper]{article}
%%% Работа с русским языком
\usepackage{cmap}					% поиск в PDF
\usepackage{mathtext} 				% русские буквы в формулах
\usepackage[T2A]{fontenc}			% кодировка
\usepackage[utf8]{inputenc}			% кодировка исходного текста
\usepackage[main=english,russian]{babel}	% локализация и переносы
\selectlanguage{english}
\usepackage{caption}

%%% Дополнительная работа с математикой
\usepackage{amsmath,amsfonts,amssymb,amsthm,mathtools} % AMS
\usepackage{icomma}
\usepackage{physics}
\usepackage{multicol}
\usepackage{bm}
\usepackage{mathrsfs}
\usepackage{verbatim}

%%% Номера формул
%\mathtoolsset{showonlyrefs=true} 
%\usepackage{leqno} 

%%% Свои команды
\DeclareMathOperator{\sgn}{sgn}

\usepackage{csquotes} 
\usepackage[backend=biber,style=authoryear]{biblatex}


%%% Работа с графикой
\usepackage{graphicx}
\graphicspath{{images/}}  
\setlength\fboxsep{3pt} 
\setlength\fboxrule{1pt} 
\usepackage{wrapfig} 
\usepackage{tikz}
\usepackage{pgfplots}
\usepackage{pgfplotstable}
\usepgfplotslibrary{polar}
\pgfplotsset{compat=1.18} 

%%% Работа с таблицами
\usepackage{array,tabularx,tabulary,booktabs}
\usepackage{longtable}  
\usepackage{multirow} 
\usepackage{soul} 

%%% Теоремы
\theoremstyle{plain} 
\newtheorem{theorem}{Th}[section]
\newtheorem{proposition}[theorem]{Proposition}
\newtheorem{question}{Question}[section]
 
\theoremstyle{definition} 
\newtheorem{corollary}{Corollary}[theorem]
\newtheorem{problem}{Problem}[section]
\newtheorem{definition}{Def}[section]
 
\theoremstyle{remark} 
\newtheorem*{nonum}{Solution}

%%% Программирование
\usepackage{etoolbox} 

%%% Гиперссылки
\usepackage{hyperref}

\usetikzlibrary{knots}
\usepackage{tcolorbox}

%%% Страница
\usepackage{geometry} 
	\geometry{top=20mm, bottom=20mm, left=15mm, right=20mm}

\usepackage{setspace} 

\usepackage{lastpage} 
\usepackage{amssymb}
\usepackage{xcolor}
\DeclareMathOperator{\Ker}{Ker}
\DeclareMathOperator{\qdim}{qdim}
\usepackage[all]{xy}
\usepackage{ dsfont }

\addbibresource{../source.bib}
\begin{document}

\subsubsection{R1-invariance}
The following argument applies to both the left and right R1 moves.
\begin{tcolorbox}
\begin{proposition}
    Let $C$ be a bicomplex concentrated in two rows:
    \begin{center}
    \begin{minipage}{0.3\linewidth}
      \resizebox{\linewidth}{!}{\begin{tikzpicture}
    \node at (0, 0) {0};
    \node at (0, -2) {0};

    \draw[->] (0.5,0) -- (1.5,0);
    \draw[->] (0.5,-2) -- (1.5,-2);

    \node at (2,2) {0};
    \draw[->] (2,1.5) -- (2,0.5);
    \node at (2, 0) {$C_{0, 0}$};
    \draw[->] (2,-0.5) -- (2,-1.5) node[midway, right]{$d^v_{0, 0}$};
    \node at (2, -2) {$C_{0, 1}$};
    \draw[->] (2,-2.5) -- (2,-3.5);
    \node at (2, -4) {0};

    \draw[->] (2.5,0) -- (3.5,0) node[midway, above]{$d^h_{0, 0}$};
    \draw[->] (2.5,-2) -- (3.5,-2) node[midway, above]{$d^h_{0, 1}$};

    \node at (4,2) {0};
    \draw[->] (4,1.5) -- (4,0.5);
    \node at (4, 0) {$C_{1, 0}$};
    \draw[->] (4,-0.5) -- (4,-1.5) node[midway, right]{$d^v_{1, 0}$};
    \node at (4, -2) {$C_{1, 1}$};
    \draw[->] (4,-2.5) -- (4,-3.5);
    \node at (4, -4) {0};

    \draw[->] (4.5,0) -- (5.5,0) node[midway, above]{$d^h_{1, 0}$};
    \draw[->] (4.5,-2) -- (5.5,-2) node[midway, above]{$d^h_{1, 1}$};

    \node at (6,2) {0};
    \draw[->] (6,1.5) -- (6,0.5);
    \node at (6, 0) {$C_{2, 0}$};
    \draw[->] (6,-0.5) -- (6,-1.5) node[midway, right]{$d^v_{2, 0}$};
    \node at (6, -2) {$C_{2, 1}$};
    \draw[->] (6,-2.5) -- (6,-3.5);
    \node at (6, -4) {0};

    \draw[->] (6.5,0) -- (7.5,0) node[midway, above]{$d^h_{2, 0}$};
    \draw[->] (6.5,-2) -- (7.5,-2) node[midway, above]{$d^h_{2, 1}$};

    \node at (8, 0) {$\cdots$};
    \node at (8, -2) {$\cdots$};
\end{tikzpicture}}
  \end{minipage}
\end{center}
    If each vertical complex $0 \rightarrow C_{n, 0}\xrightarrow{d^v_{n, 0}} C_{n, 1} \rightarrow 0$ is acyclic, then the total complex $\mathrm{Tot}(C)$ ($0 \rightarrow Tot(C)_0 = C_{0, 0} \xrightarrow{d_0} Tot(C)_1 = C_{1, 0} \oplus C_{0, 1} \xrightarrow{d_1}\cdots$) is also acyclic.
\end{proposition}
\end{tcolorbox}
$\triangleleft$
From the acyclicity of the vertical complexes we obtain that $C_{n, 0} \overset{d^v_{n, 0}}{\cong} C_{n, 1}$. Hence, in view of the bicomplex anticommutativity $d^v_{n+1, m} \circ d^h_{n, m} + d^h_{n, m+1} \circ d^v_{n, m}=0$, we conclude that the bicomplex can be redrawn in the following way:

\begin{center}
\begin{minipage}{0.3\linewidth}
      \resizebox{\linewidth}{!}{\begin{tikzpicture}
    \node at (0, 0) {0};
    \node at (0, -2) {0};

    \draw[->] (0.5,0) -- (1.5,0);
    \draw[->] (0.5,-2) -- (1.5,-2);

    \node at (2,2) {0};
    \draw[->] (2,1.5) -- (2,0.5);
    \node at (2, 0) {$C_0$};
    \node[rotate=90] at (2,-1) {$\cong$};
    \node at (2.5, -1){$d^v_0$};
    \node at (2, -2) {$C_0$};
    \draw[->] (2,-2.5) -- (2,-3.5);
    \node at (2, -4) {0};

    \draw[->] (2.5,0) -- (3.5,0) node[midway, above]{$d^h_0$};
    \draw[->] (2.5,-2) -- (3.5,-2) node[midway, above]{$-d^h_0$};

    \node at (4,2) {0};
    \draw[->] (4,1.5) -- (4,0.5);
    \node at (4, 0) {$C_1$};
    \node[rotate=90] at (4,-1) {$\cong$};
    \node at (4.5, -1){$d^v_1$};
    \node at (4, -2) {$C_1$};
    \draw[->] (4,-2.5) -- (4,-3.5);
    \node at (4, -4) {0};

    \draw[->] (4.5,0) -- (5.5,0) node[midway, above]{$d^h_1$};
    \draw[->] (4.5,-2) -- (5.5,-2) node[midway, above]{$-d^h_1$};

    \node at (6,2) {0};
    \draw[->] (6,1.5) -- (6,0.5);
    \node at (6, 0) {$C_2$};
    \node[rotate=90] at (6,-1) {$\cong$};
    \node at (6.5, -1){$d^v_2$};
    \node at (6, -2) {$C_2$};
    \draw[->] (6,-2.5) -- (6,-3.5);
    \node at (6, -4) {0};

    \draw[->] (6.5,0) -- (7.5,0) node[midway, above]{$d^h_2$};
    \draw[->] (6.5,-2) -- (7.5,-2) node[midway, above]{$-d^h_2$};

    \node at (8, 0) {$\cdots$};
    \node at (8, -2) {$\cdots$};
\end{tikzpicture}}
  \end{minipage}
\end{center}

From this we have $\Ker(d_n) = \Im(d_{n-1}) = \{(c, d^h_{n - 1}(c)) | c \in C_{n - 1}\} ~\Rightarrow ~ H_n(Tot(C)) = 0$ $\triangleright$
\vspace{1cm}

The above proposition justifies the total complex factorizations used in \textcite{bar-natan}.
\end{document}