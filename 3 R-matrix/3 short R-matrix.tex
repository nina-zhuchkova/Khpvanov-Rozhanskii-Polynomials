\documentclass[a4paper,8pt]{extarticle}

%%% Работа с русским языком
\usepackage{cmap}					% поиск в PDF
\usepackage{mathtext} 				% русские буквы в формулах
\usepackage[T2A]{fontenc}			% кодировка
\usepackage[utf8]{inputenc}			% кодировка исходного текста
\usepackage[english,russian]{babel}	% локализация и переносы

%%% Дополнительная работа с математикой
\usepackage{amsmath,amsfonts,amssymb,amsthm,mathtools} % AMS
\usepackage{icomma}
\usepackage{physics}
\usepackage{multicol}
\usepackage{bm}
\usepackage{mathrsfs}
\usepackage{verbatim}

%%% Номера формул
%\mathtoolsset{showonlyrefs=true} 
%\usepackage{leqno} 

%%% Свои команды
\DeclareMathOperator{\sgn}{sgn}

\usepackage{csquotes} 
\usepackage[backend=biber,style=authoryear,language=auto]{biblatex}
\addbibresource{source.bib}


%%% Работа с графикой
\usepackage{graphicx}
\graphicspath{{images/}}  
\setlength\fboxsep{3pt} 
\setlength\fboxrule{1pt} 
\usepackage{wrapfig} 
\usepackage{tikz}
\usepackage{pgfplots}
\usepackage{pgfplotstable}
\usepgfplotslibrary{polar}
\pgfplotsset{compat=1.18} 

%%% Работа с таблицами
\usepackage{array,tabularx,tabulary,booktabs}
\usepackage{longtable}  
\usepackage{multirow} 
\usepackage{caption2}[2008/03/29]
\usepackage{soul} 

%%% Теоремы
\theoremstyle{plain} 
\newtheorem{theorem}{Th}[section]
\newtheorem{proposition}[theorem]{Proposition}
 
\theoremstyle{definition} 
\newtheorem{corollary}{Corollary}[theorem]
\newtheorem{problem}{Problem}[section]
\newtheorem{definition}{Def}[section]
 
\theoremstyle{remark} 
\newtheorem*{nonum}{Solution}

%%% Программирование
\usepackage{etoolbox} 

%%% Гиперссылки
\usepackage{hyperref}

\usetikzlibrary{knots}
\usepackage{tcolorbox}

%%% Страница
\usepackage{geometry} 
	\geometry{top=20mm, bottom=20mm, left=15mm, right=20mm}

\usepackage{fancyhdr} 
 	\pagestyle{fancy}
 	\renewcommand{\headrulewidth}{1pt}  
    \rhead{\today}
    \lhead{Новохатний Артем, Цыганкова Екатерина, Мифтахов Эльдар}

\usepackage{setspace} 

\usepackage{lastpage} 

\addbibresource{../source.bib}

\begin{document}


\begin{center}
    \Huge $\mathcal{R}$-матрицы

\end{center}

\begin{multicols}{2}
\subsection*{Глобальная задача:}
Найти инварианты узлов, которые бы позволили различать их между собой. Привести эффективный алгоритм их вычисления.

\columnbreak
\subsection*{Локальная задача:}
Понять конструкцию полиномов ХОМФЛИ, используя формализм 
$\mathcal{R}$-матриц, для дальнейшего изучения полинома
Хованова-Рожанского.
\end{multicols}

\hrule

\begin{multicols}{2}
    Данный доклад в основном опирается на обзорную статью \parencite{morozov-chern}. Все пояснительные иллюстрации взяты из неё.

	\section{Теория Черна-Саймонса}
	
	\begin{equation}
		S_{CS} = k\int A^a_x \dot{A}^a_y - A^a_y \dot{A}^a_x dtdxdy
	\end{equation}
	
	\begin{equation}
		W_{\rho}(C) = tr_{\rho} P exp \oint\limits_{\tilde{C}} A_x dx + A_y dy
	\end{equation}
	
	Этот интеграл меняет значение только при пересечениях или разворотах контура по выбранной оси. Поскольку значение является топологическим инвариантом, можно выбрать одну из проекций контура на плоскость и классифицировать все возможные перекрестки и развороты.
	
	Каждому из перекрестков можно поставить в соответствие оператор. Это будет тензор $\mathcal{R}$ с четыремя индексами, обозначающими концы контура, которые входят и выходят из перекрестка. Соответсвенно, для разворотов будем использовать тензор $\mathcal{M}$ с двумя индексами. На рис.\ref{fig:Crossing_classification} приведены все варианты с соответствующими тензорами.
	
	Тогда для замкнутого контура, можно разбить его на секции и обозначить индексами. Затем получить инвариант - свертку всех операторов (каждый индекс встретится 2 раза).
	
	\section{Соотношения на операторы}
	
	Произведя некоторые деформации контура, можно получить уравнения на различные $\mathcal{R}$ и $\mathcal{M}$. В итоге получим систему, которая зависит от четырех операторов: $\mathcal{R}, \overline{{\mathcal{R}}}, \mathcal{M}, \overline{\mathcal{M}}$.
	
	\begin{equation}
		\begin{split}
			\mathcal{M}_1 &= \mathcal{M}\\
			\mathcal{M}_4 &= \overline{M}\\
			\mathcal{M}_3 &= \mathcal{M}^{-1}\\
			\mathcal{M}_2 &= \overline{\mathcal{M}}^{-1}\\
			\mathcal{R}_1 &= \mathcal{R}\\
			\mathcal{R}_2 &= (1 \otimes \mathcal{M})\mathcal{R}(\mathcal{M}\otimes 1)^{-1}\\
			\mathcal{R}_3 &=(\overline{\mathcal{M}} \otimes 1)\mathcal{R}(1 \otimes \overline{\mathcal{M}})^{-1}\\
			\mathcal{R}_4 &= (\mathcal{M} \otimes \mathcal{M})\mathcal{R}(\mathcal{M} \otimes \mathcal{M})^{-1}\\
			\mathcal{R}_5 &= \overline{\mathcal{R}}\\
			\mathcal{R}_6 &= (1 \otimes \mathcal{M})\overline{\mathcal{R}}(\mathcal{M}\otimes 1)^{-1}\\
			\mathcal{R}_7 &=(\overline{\mathcal{M}} \otimes 1)\overline{\mathcal{R}}(1 \otimes \overline{\mathcal{M}})^{-1}\\
			\mathcal{R}_8 &= (\mathcal{M} \otimes \mathcal{M})\overline{\mathcal{R}}(\mathcal{M} \otimes \mathcal{M})^{-1}
		\end{split}
	\end{equation}
	
	Нужен(почти) инвариант узлов. Поэтому должны выполнятся(почти) соотношения рейдемейстера. То есть уравнения:
	
	\begin{equation}
		\begin{split}
			\overline{\mathcal{R}}^{ab}_{cd} \mathcal{R}^{cd}_{ef} = \mathcal{R}^{ab}_{cd} \overline{\mathcal{R}^{cd}_{ef}} &= \delta^a_e\delta^b_f\\
			\mathcal{R}^{ae}_{bc}\mathcal{M}^c_d\overline{\mathcal{M}}^d_e = q^{\rho(\Omega_2)}\delta^a_b; & \quad \overline{\mathcal{R}}^{ae}_{bc}\mathcal{M}^c_d\overline{\mathcal{M}}^d_e = q^{-\rho(\Omega_2)}\delta^a_b\\
			\mathcal{R}^{ab}_{de}\mathcal{R}^{ec}_{fg}\mathcal{R}^{df}_{km} &= \mathcal{R}^{bc}_{de}\mathcal{R}^{ad}_{km}\mathcal{R}^{ge}_{mg}
		\end{split}
	\end{equation}
	
    \includegraphics[width=0.8\linewidth]{../img/morozov-reid.png}

    Или в более компактной форме:

    \begin{tcolorbox}
        \begin{equation}
        \tr_{2}(\mathcal{R}^{\pm 1} \ 1 \otimes \mathcal{Q})=
        q^{\pm \rho(\Omega_2)}, \ \ \mathcal{Q} = 
        \mathcal{M}\overline{\mathcal{M}}
        \label{eq:markov}
    \end{equation}

    \begin{equation}
        \text{Yang-Baxter: }
        \mathcal{R}_{12}\mathcal{R}_{23}\mathcal{R}_{12} = 
        \mathcal{R}_{23}\mathcal{R}_{12}\mathcal{R}_{23}
        \label{eq:YB}
    \end{equation}
    \end{tcolorbox}
    
    Здесь $\Omega_2$ - квадратичный оператор Казимира, $\mathcal{R}_{12} = 
    \mathcal{R} \otimes 1$, $\mathcal{R}_{23} = 
    1 \otimes \mathcal{R}$.
	
	Также контур можно изменить так, чтобы он принял форму косы с соединенными верними и нижними концами. В таком виде нужно рассматривать только две матрицы пересечений $\mathcal{R}$ и $\mathcal{R}^{-1}$. Третье соотношение в терминах кос будет иметь вид $\sigma_i \sigma_{i+1} \sigma_{i} = \sigma_{i+1}\sigma_{i}\sigma_{i+1}$.
	
    \section{$\mathcal{R}$-матрица}
    
    В качестве тензоров рассматриваются элементы квантовой обертывающей алгебры для алгебры Ли. Если таким образом выбрать алгебру $sl_2$, то можно получить полином Джлонса. Для $sl_N$ - полином Хомфли.

    Прежде чем записать ответ введём определения:

    \begin{equation}
        \Phi^+ \text{ - множество положительных корней c нормальным порядком}
    \end{equation}
    \begin{equation}
        \Delta \text{ - множество \textbf{простых} положительных корней }
    \end{equation}
    \begin{equation}
        \text{Сопряженный корень: }(\alpha_i^\vee,\alpha_j)=\delta_{ij}
    \end{equation}
    \begin{equation}
        h_\gamma=\sum\limits_{\alpha\in\Delta}h_\alpha (\alpha^\vee,
        \gamma)
    \end{equation}

    \textbf{$\mathcal{R}$- и $\mathcal{Q}$-матрицы следующего вида удовлетворяют соотношениям eqref[eq:markov] и eqref[eq:YB]:}
  \begin{tcolorbox}
\begin{equation}
	\begin{split}
		\mathcal{R} = \hat{P} q^{\sum\limits_{\alpha \in \Delta}h_\alpha \otimes h_\alpha \vee} &\prod\limits_{\beta \in \Phi^+}^{\rightarrow} exp_{q_\beta}((q_\beta - q_\beta^{-1})E_\beta \otimes F_\beta)\\
		\mathcal{M} &= \overline{\mathcal{M}} = q^{\frac{1}{2}h_\rho}
	\end{split}
	\label{eq:R-matrix-gen}
\end{equation}

\begin{equation}
        \mathcal{Q} = q^{h_\rho}, \ \
        \rho = \frac{1}{2} \sum\limits_{\alpha\in\Phi^+}\alpha, \ \
        h_\rho = \sum\limits_{\alpha\in\Phi^+} h_\alpha
    \end{equation}
  \end{tcolorbox}  
    Здесь:

    \begin{equation}
        \exp_q(A)=\sum\limits_{m=0}^{\infty}\frac{A^m}{[m]_q!}q^{\frac{m(m-1)}{2}}
    \end{equation}
    
    

    \section{Инвариант узла}

    \textbf{Здесь и далее, несмотря на предыдущие картинки, мы сопоставляем $\mathcal{R}$-матрицу положительному перекрестку, а $\mathcal{R}^{-1}$ -- отрицательному}
    \begin{equation}
  +: \ \
  \begin{minipage}{0.06\linewidth}
    \vspace{0pt}
    \resizebox{\linewidth}{!}{\input{../tikz knots/cross1+.tex}}
    \end{minipage} \ , \ \ -: \ \
    \begin{minipage}{0.06\linewidth}
    \vspace{0pt}
    \resizebox{\linewidth}{!}{\input{../tikz knots/cross2-.tex}}
    \end{minipage}
\end{equation}

    Исходя из того, что любой узел $L$ можнно представить в виде
    замыкания $\hat{b}$ косы $b\in B_n$, можем записать следующий полином:

    \begin{equation}
        I(L, \rho) = q^{-w(\hat{b})\rho(\Omega_2)}\tr(\mathcal{Q}_\rho^
        {\otimes n} b_\rho)
    \end{equation}

    \section{Случай $\mathfrak{sl}_2$. Полином Джонса}

    Для фундаменатального представления $\mathfrak{sl}_2$ формула
    \eqref{eq:R-matrix-gen}:

    \begin{equation}
    	\begin{split}
	        \mathcal{R}_{\mathfrak{sl}_2, \square} &= 
	        \hat{P}q^{\frac{H \otimes H}{2}} (1 + (q-q^{-1})E\otimes F)\\
	        \mathcal{M}_{\mathfrak{sl}_2, \square} &= q^{\frac{1}{2}H}
        \end{split}
    \end{equation}

    Что в матричной форме:

    \begin{equation}
    	\begin{split}
        \mathcal{R}_{\mathfrak{sl}_2, \square} &= 
	        \begin{pmatrix}
	        q^{\frac{1}{2}} & 0 & 0 & 0 \\
	        0 & 0 & q^{-\frac{1}{2}} & 0 \\
	        0 & q^{-\frac{1}{2}} & q^{\frac{1}{2}}-q^{-\frac{3}{2}} & 0 \\
	        0 & 0 & 0 & q^{\frac{1}{2}}
	        \end{pmatrix}\\
        \mathcal{M}_{\mathfrak{sl}_2, \square} &= 
	        \begin{pmatrix}
	        	q^{1/2} & 0\\
	        	0 & q^{-1/2}
	        \end{pmatrix}
        \end{split}
    \end{equation}
    
    Напомним, что скейн-соотношение для полинома Джонса выглядит
    следующим образом:

    \begin{equation}
    q^{-2} J\left (
    \begin{minipage}{0.06\linewidth}
        \resizebox{\linewidth}{!}{\input{../tikz knots/cross1+.tex}}
        \end{minipage}
    \right ) - q^2 J\left (
    \begin{minipage}{0.06\linewidth}
        \resizebox{\linewidth}{!}{\input{../tikz knots/cross2-.tex}}
        \end{minipage}
    \right ) = (q^{-1}-q) J\left (
    \begin{minipage}{0.06\linewidth}
        \resizebox{\linewidth}{!}{\begin{tikzpicture}
\begin{knot}[
consider self intersections,
clip width=5,
]
\strand[line width=3pt, ->, black]
(-1, -1) to[out=60,in=-90,looseness=1]
(-0.5, 0) to[out=90,in=-60,looseness=1] (-1, 1);
\strand[line width=3pt, ->, black]
(1, -1) to[out=120,in=-90,looseness=1]
(0.5, 0) to[out=90,in=-120,looseness=1] (1, 1);
\end{knot}
\end{tikzpicture}}
        \end{minipage}
    \right )
    \label{eq:skein}
    \end{equation}

    Будем искать схожее соотношение на $I$ из предыдущего раздела:

    \begin{equation}
    \alpha_+ \cdot I \left (
    \begin{minipage}{0.06\linewidth}
        \resizebox{\linewidth}{!}{\begin{tikzpicture}[rotate=180]
\begin{knot}[
consider self intersections,
clip width=5,
]
\strand[line width=3pt, ->, black]
(-1, -1) to[out=45,in=-135,looseness=1] (1, 1);
\strand[line width=3pt, <-, black]
(-1, 1) to[out=-45,in=135,looseness=1] (1, -1);
\end{knot}
\end{tikzpicture}}
        \end{minipage}
    \right ) + \alpha_- \cdot I\left (
    \begin{minipage}{0.06\linewidth}
        \resizebox{\linewidth}{!}{\input{../tikz knots/cross2-_rotate.tex}}
        \end{minipage}
    \right ) = \alpha_0 \cdot I\left (
    \begin{minipage}{0.06\linewidth}
        \resizebox{\linewidth}{!}{\input{../tikz knots/R2_2_down_down.tex}}
        \end{minipage}
    \right )
    \label{eq:skein-r}
    \end{equation}

    То есть нужно найти такие $\alpha_+,\alpha_-,\alpha_0$ что:
    
    \begin{equation}
        \alpha_+ q^{-\Omega_2} \mathcal{R} + 
        \alpha_- q^{\Omega_2} \mathcal{R}^{-1} = 
        \alpha_0 \ 1 \otimes 1, \ \Omega_2 = \frac{3}{2}
    \end{equation}

    В итоге мы получаем следующие свойства для $I$:

    \begin{tcolorbox}
        \begin{equation}
            q^2 I \left (
        \begin{minipage}{0.06\linewidth}
            \resizebox{\linewidth}{!}{\begin{tikzpicture}[rotate=180]
\begin{knot}[
consider self intersections,
clip width=5,
]
\strand[line width=3pt, ->, black]
(-1, -1) to[out=45,in=-135,looseness=1] (1, 1);
\strand[line width=3pt, <-, black]
(-1, 1) to[out=-45,in=135,looseness=1] (1, -1);
\end{knot}
\end{tikzpicture}}
            \end{minipage}
        \right ) - q^{-2} I\left (
        \begin{minipage}{0.06\linewidth}
            \resizebox{\linewidth}{!}{\input{../tikz knots/cross2-_rotate.tex}}
            \end{minipage}
        \right ) = (q-q^{-1}) I\left (
        \begin{minipage}{0.06\linewidth}
            \resizebox{\linewidth}{!}{\input{../tikz knots/R2_2_down_down.tex}}
            \end{minipage}
        \right )
        \end{equation}

        \begin{equation}
            I(
                \begin{minipage}{0.04\linewidth}
                \resizebox{\linewidth}{!}{\begin{tikzpicture}
\begin{knot}
\strand[line width=3pt, black] 
(0,0) circle[radius=1cm];
\end{knot}
\end{tikzpicture}}
                \end{minipage}
            ) = q+q^{-1}
        \end{equation}

        \begin{equation}
            I(L_1 \cup L_2) = I(L_1) \cdot I(L_2)
        \end{equation}
    \end{tcolorbox}

    Что в точности совпадает с определением полинома Джонса из
    первого доклада.

    Для трилистника:

    \begin{equation}
        \begin{minipage}{0.1\linewidth}
            \resizebox{\linewidth}{!}{\input{../tikz knots/3_1-brade.tex}}
            \end{minipage} \ \ 
            \leftrightarrow
        \begin{minipage}{0.13\linewidth}
            \vspace{7pt}
            \resizebox{\linewidth}{!}{\begin{tikzpicture}
\begin{knot}[
consider self intersections,
flip crossing/.list={1, 3}, %переворот пересечений
clip width=5,
]
\strand[line width=3pt, black]
(0, 1) to[out=180,in=-120,looseness=2]
%декартовы координаты
(-30:1) to[out=60,in=120,looseness=2]
%полярные координаты
(210:1) to[out=-60,in=0,looseness=2] (90:1);
\end{knot}
\end{tikzpicture}}
            \end{minipage} \ \ \
        I(
            \begin{minipage}{0.06\linewidth}
            \vspace{2pt}
            \resizebox{\linewidth}{!}{\begin{tikzpicture}
\begin{knot}[
consider self intersections,
flip crossing/.list={1, 3}, %переворот пересечений
clip width=5,
]
\strand[line width=3pt, black]
(0, 1) to[out=180,in=-120,looseness=2]
%декартовы координаты
(-30:1) to[out=60,in=120,looseness=2]
%полярные координаты
(210:1) to[out=-60,in=0,looseness=2] (90:1);
\end{knot}
\end{tikzpicture}}
            \end{minipage}
        ) = (q+q^{-1})\cdot (q^{-2}+q^{-6}-q^{-8})
    \end{equation}
    

        \section{Случай $\mathfrak{sl}_N$. Полином HOMFLY}

    Для фундаменатального представления $\mathfrak{sl}_N$ ненулевые элементы $\mathcal{R}$
    \eqref{eq:R-matrix-gen}:

    \begin{equation}
    	\mathcal{R}^{ii}_{ii} = q^{-\frac{1}{N}}\cdot q
    \end{equation}
    \begin{equation}
    	\mathcal{R}^{ij}_{ji} = q^{-\frac{1}{N}}~,~~~i \neq j
    \end{equation}
    \begin{equation}
    	\mathcal{R}^{ij}_{ij} = q^{-\frac{1}{N}}(q - q^{-1})~,~~~i>j 
    \end{equation}

    \begin{equation}
        \mathcal{Q} = \text{diag}(q^{N + 1 - 2i})~,~~~i=1, \cdots, N
    \end{equation}

    \begin{equation}
        \Omega_2 = N - \frac{1}{N}
    \end{equation}

    Переписывая, аналогично Джонсу, определение инварианта через skein-соотношения, получаем:

        \begin{tcolorbox}
        \begin{equation}
            q^N I \left (
        \begin{minipage}{0.06\linewidth}
            \resizebox{\linewidth}{!}{\begin{tikzpicture}[rotate=180]
\begin{knot}[
consider self intersections,
clip width=5,
]
\strand[line width=3pt, ->, black]
(-1, -1) to[out=45,in=-135,looseness=1] (1, 1);
\strand[line width=3pt, <-, black]
(-1, 1) to[out=-45,in=135,looseness=1] (1, -1);
\end{knot}
\end{tikzpicture}}
            \end{minipage}
        \right ) - q^{-N} I\left (
        \begin{minipage}{0.06\linewidth}
            \resizebox{\linewidth}{!}{\input{../tikz knots/cross2-_rotate.tex}}
            \end{minipage}
        \right ) = (q-q^{-1}) I\left (
        \begin{minipage}{0.06\linewidth}
            \resizebox{\linewidth}{!}{\input{../tikz knots/R2_2_down_down.tex}}
            \end{minipage}
        \right )
        \end{equation}

        \begin{equation}
            I(
                \begin{minipage}{0.04\linewidth}
                \resizebox{\linewidth}{!}{\begin{tikzpicture}
\begin{knot}
\strand[line width=3pt, black] 
(0,0) circle[radius=1cm];
\end{knot}
\end{tikzpicture}}
                \end{minipage}
            ) = \frac{q^N-q^{-N}}{q - q^{-1}}
        \end{equation}

        \begin{equation}
            I(L_1 \cup L_2) = I(L_1) \cdot I(L_2)
        \end{equation}
    \end{tcolorbox}

    Что, при замене $\alpha = q^N$ и $\beta = q - q^{-1}$ совпадает с определением Джима Хосте полинома HOMFLY.

    Для трилистника:

    \begin{equation}
        I(
            \begin{minipage}{0.06\linewidth}
            \vspace{2pt}
            \resizebox{\linewidth}{!}{\begin{tikzpicture}
\begin{knot}[
consider self intersections,
flip crossing/.list={1, 3}, %переворот пересечений
clip width=5,
]
\strand[line width=3pt, black]
(0, 1) to[out=180,in=-120,looseness=2]
%декартовы координаты
(-30:1) to[out=60,in=120,looseness=2]
%полярные координаты
(210:1) to[out=-60,in=0,looseness=2] (90:1);
\end{knot}
\end{tikzpicture}}
            \end{minipage}
        ) = ...
    \end{equation}

    \section{Вопросы}

    \begin{itemize}
        \item Почему из раза в раз испытываем трудности с первым преобразованием Рейдемейстера?
        \item (технический) про явный вид $\mathcal{R}$-матрицы в $\mathfrak{sl}_N$.
    \end{itemize}
\end{multicols}
\begin{figure}[h]
	\centering
	\includegraphics[width=0.8\linewidth]{../img/Crossing_classification}
	\caption{Классификация пересечений и разворотов \parencite{}}
	\label{fig:Crossing_classification}
\end{figure}

\printbibliography

\end{document}