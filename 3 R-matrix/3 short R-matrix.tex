\documentclass[a4paper,8pt]{extarticle}

%%% Работа с русским языком
\usepackage{cmap}					% поиск в PDF
\usepackage{mathtext} 				% русские буквы в формулах
\usepackage[T2A]{fontenc}			% кодировка
\usepackage[utf8]{inputenc}			% кодировка исходного текста
\usepackage[english,russian]{babel}	% локализация и переносы

%%% Дополнительная работа с математикой
\usepackage{amsmath,amsfonts,amssymb,amsthm,mathtools} % AMS
\usepackage{icomma}
\usepackage{physics}
\usepackage{multicol}
\usepackage{bm}
\usepackage{mathrsfs}
\usepackage{verbatim}

%%% Номера формул
%\mathtoolsset{showonlyrefs=true} 
%\usepackage{leqno} 

%%% Свои команды
\DeclareMathOperator{\sgn}{sgn}

\usepackage{csquotes} 
\usepackage[backend=biber,style=authoryear,language=auto]{biblatex}
\addbibresource{source.bib}


%%% Работа с графикой
\usepackage{graphicx}
\graphicspath{{images/}}  
\setlength\fboxsep{3pt} 
\setlength\fboxrule{1pt} 
\usepackage{wrapfig} 
\usepackage{tikz}
\usepackage{pgfplots}
\usepackage{pgfplotstable}
\usepgfplotslibrary{polar}
\pgfplotsset{compat=1.18} 

%%% Работа с таблицами
\usepackage{array,tabularx,tabulary,booktabs}
\usepackage{longtable}  
\usepackage{multirow} 
\usepackage{caption2}[2008/03/29]
\usepackage{soul} 

%%% Теоремы
\theoremstyle{plain} 
\newtheorem{theorem}{Th}[section]
\newtheorem{proposition}[theorem]{Proposition}
 
\theoremstyle{definition} 
\newtheorem{corollary}{Corollary}[theorem]
\newtheorem{problem}{Problem}[section]
\newtheorem{definition}{Def}[section]
 
\theoremstyle{remark} 
\newtheorem*{nonum}{Solution}

%%% Программирование
\usepackage{etoolbox} 

%%% Гиперссылки
\usepackage{hyperref}

\usetikzlibrary{knots}
\usepackage{tcolorbox}

%%% Страница
\usepackage{geometry} 
	\geometry{top=20mm, bottom=20mm, left=15mm, right=20mm}

\usepackage{fancyhdr} 
 	\pagestyle{fancy}
 	\renewcommand{\headrulewidth}{1pt}  
    \rhead{\today}
    \lhead{Новохатний Артем, Цыганкова Екатерина, Мифтахов Эльдар}

\usepackage{setspace} 

\usepackage{lastpage} 

\addbibresource{../source.bib}

\begin{document}

\begin{center}
    \Huge $\mathcal{R}$-матрицы
\end{center}

\begin{multicols}{2}
\subsection*{Глобальная задача:}
Найти инварианты узлов, которые бы позволили различать их между собой. Привести эффективный алгоритм их вычисления.

\columnbreak
\subsection*{Локальная задача:}
Понять конструкцию полиномов ХОМФЛИ, используя формализм 
$\mathcal{R}$-матриц, для дальнейшего изучения полинома
Хованова-Рожанского.
\end{multicols}

\hrule

\begin{multicols}{2}
    \section{Условия на операторы}

    Исходя из ходов Рейдемейстера \parencite{morozov-chern}:

    \includegraphics[width=0.8\linewidth]{../img/morozov-reid.png}
    
    Запишем условия на $\mathcal{R}$-матрицу:

    \begin{equation}
        \text{R1: } \
        \mathcal{R}^{ae}_{bc} \mathcal{M}^{c}_{d} 
        \overline{\mathcal{M}}^{d}_{e} = q^{\rho(\Omega_2)} \delta^a_b, \
        \overline{\mathcal{R}}^{ae}_{bc} \mathcal{M}^{c}_{d} 
        \overline{\mathcal{M}}^{d}_{e} = q^{-\rho(\Omega_2)} \delta^a_b
    \end{equation}

    \begin{equation}
        \text{R2: } \ \
        \overline{\mathcal{R}}^{ab}_{cd} \mathcal{R}^{cd}_{ef} = 
        \mathcal{R}^{ab}_{cd} \overline{\mathcal{R}}^{cd}_{ef} = 
        \delta^a_e \delta^b_f
    \end{equation}

    \begin{equation}
        \text{R3: } \ \
        \mathcal{R}^{ab}_{de} \mathcal{R}^{ec}_{fg}
        \mathcal{R}^{df}_{km} = \mathcal{R}^{bc}_{de}
        \mathcal{R}^{ad}_{km} \mathcal{R}^{ge}_{mg}
    \end{equation}

    Или в более компактной форме:

    \begin{tcolorbox}
        \begin{equation}
        \tr_{2}(\mathcal{R}^{\pm 1} \ 1 \otimes \mathcal{Q})=
        q^{\pm \rho(\Omega_2)}, \ \ \mathcal{Q} = 
        \mathcal{M}\overline{\mathcal{M}}
    \end{equation}

    \begin{equation}
        \text{Yang-Baxter: }
        \mathcal{R}_{12}\mathcal{R}_{23}\mathcal{R}_{12} = 
        \mathcal{R}_{23}\mathcal{R}_{12}\mathcal{R}_{23}
        \label{eq:YB}
    \end{equation}
    \end{tcolorbox}
    
    Здесь $\Omega_2$ - квадратичный оператор Казимира, $\mathcal{R}_{12} = 
    \mathcal{R} \otimes 1$, $\mathcal{R}_{23} = 
    1 \otimes \mathcal{R}$.

    \section{$\mathcal{R}$-матрица}

    Прежде чем записать ответ введём определения:

    \begin{equation}
        \Phi^+ \text{ - множество положительных корней}
    \end{equation}
    \begin{equation}
        \Delta \text{ - множество \textbf{простых} положительных корней }
    \end{equation}
    \begin{equation}
        \text{Сопряженный корень: }(\alpha_i^\vee,\alpha_j)=\delta_{ij}
    \end{equation}
    \begin{equation}
        h_\gamma=\sum\limits_{\alpha\in\Delta}h_\alpha (\alpha^\vee,
        \gamma)
    \end{equation}

    Итак, условия предыдущего раздела дают следующий ответ:

    \begin{equation}
        \mathcal{R}=\hat{P}q^{\sum\limits_{\alpha\in\Delta} 
        h_\alpha \otimes h_{\alpha^{\vee}}} \prod\limits_{\beta\in 
        \Phi^+}^{\rightarrow}\exp_{q_\beta}\left[
            (q_\beta - q_\beta^{-1}) E_\beta \otimes F_\beta
        \right]
        \label{eq:R-matrix-gen}
    \end{equation}

    Здесь:

    \begin{equation}
        \exp_q(A)=\sum\limits_{m=0}^{\infty}\frac{A^m}{[m]_q!}q^{\frac{m(m-1)}{2}}
    \end{equation}
    \begin{equation}
        \mathcal{Q} = q^{h_\rho}, \ \
        \rho = \frac{1}{2} \sum\limits_{\alpha\in\Phi^+}\alpha, \ \
        h_\rho = \sum\limits_{\alpha\in\Phi^+} h_\alpha
    \end{equation}

    \section{Инвариант узла}

    Исходя из того, что любой узел $L$ можнно представить в виде
    замыкания $\hat{b}$ косы $b\in B_n$, можем записать следующий полином:

    \begin{equation}
        I(L, \rho) = q^{-w(\hat{b})\rho(\Omega_2)}\tr(\mathcal{Q}_\rho^
        {\otimes n} b_\rho)
    \end{equation}

    \section{Случай $\mathfrak{sl}_2$. Полином Джонса}

    Для фундаменатального представления $\mathfrak{sl}_2$ формула
    \eqref{eq:R-matrix-gen}:

    \begin{equation}
        \mathcal{R}_{\mathfrak{sl}_2, \square} = 
        \hat{P}q^{\frac{H \otimes H}{2}} (1 + (q-q^{-1})E\otimes F)
    \end{equation}

    Что в матричной форме:

    \begin{equation}
        \mathcal{R}_{\mathfrak{sl}_2, \square} = 
        \begin{pmatrix}
        q^{\frac{1}{2}} & 0 & 0 & 0 \\
        0 & 0 & q^{-\frac{1}{2}} & 0 \\
        0 & q^{-\frac{1}{2}} & q^{\frac{1}{2}}-q^{-\frac{3}{2}} & 0 \\
        0 & 0 & 0 & q^{\frac{1}{2}}
        \end{pmatrix}
    \end{equation}
    
    Напомним, что скейн-соотношение для полинома Джонса выглядит
    следующим образом:

    \begin{equation}
    q^{-2} J\left (
    \begin{minipage}{0.06\linewidth}
        \resizebox{\linewidth}{!}{\input{../tikz knots/cross1+.tex}}
        \end{minipage}
    \right ) - q^2 J\left (
    \begin{minipage}{0.06\linewidth}
        \resizebox{\linewidth}{!}{\input{../tikz knots/cross2-.tex}}
        \end{minipage}
    \right ) = (q^{-1}-q) J\left (
    \begin{minipage}{0.06\linewidth}
        \resizebox{\linewidth}{!}{\begin{tikzpicture}
\begin{knot}[
consider self intersections,
clip width=5,
]
\strand[line width=3pt, ->, black]
(-1, -1) to[out=60,in=-90,looseness=1]
(-0.5, 0) to[out=90,in=-60,looseness=1] (-1, 1);
\strand[line width=3pt, ->, black]
(1, -1) to[out=120,in=-90,looseness=1]
(0.5, 0) to[out=90,in=-120,looseness=1] (1, 1);
\end{knot}
\end{tikzpicture}}
        \end{minipage}
    \right )
    \label{eq:skein}
    \end{equation}

    Будем искать схожее соотношение на $I$ из предыдущего раздела:

    \begin{equation}
    \alpha_+ \cdot I \left (
    \begin{minipage}{0.06\linewidth}
        \resizebox{\linewidth}{!}{\input{../tikz knots/cross2-_rotate.tex}}
        \end{minipage}
    \right ) + \alpha_- \cdot I\left (
    \begin{minipage}{0.06\linewidth}
        \resizebox{\linewidth}{!}{\begin{tikzpicture}[rotate=180]
\begin{knot}[
consider self intersections,
clip width=5,
]
\strand[line width=3pt, ->, black]
(-1, -1) to[out=45,in=-135,looseness=1] (1, 1);
\strand[line width=3pt, <-, black]
(-1, 1) to[out=-45,in=135,looseness=1] (1, -1);
\end{knot}
\end{tikzpicture}}
        \end{minipage}
    \right ) = \alpha_0 \cdot I\left (
    \begin{minipage}{0.06\linewidth}
        \resizebox{\linewidth}{!}{\input{../tikz knots/R2_2_down_down.tex}}
        \end{minipage}
    \right )
    \label{eq:skein-r}
    \end{equation}

    То есть нужно найти такие $\alpha_+,\alpha_-,\alpha_0$ что:
    
    \begin{equation}
        \alpha_+ q^{-\Omega_2} \mathcal{R} + 
        \alpha_- q^{\Omega_2} \mathcal{R}^{-1} = 
        \alpha_0 \ 1 \otimes 1, \ \Omega_2 = \frac{3}{2}
    \end{equation}

    В итоге мы получаем следующие свойства для $I$:

    \begin{tcolorbox}
        \begin{equation}
            q^2 I \left (
        \begin{minipage}{0.06\linewidth}
            \resizebox{\linewidth}{!}{\input{../tikz knots/cross2-_rotate.tex}}
            \end{minipage}
        \right ) - q^{-2} I\left (
        \begin{minipage}{0.06\linewidth}
            \resizebox{\linewidth}{!}{\begin{tikzpicture}[rotate=180]
\begin{knot}[
consider self intersections,
clip width=5,
]
\strand[line width=3pt, ->, black]
(-1, -1) to[out=45,in=-135,looseness=1] (1, 1);
\strand[line width=3pt, <-, black]
(-1, 1) to[out=-45,in=135,looseness=1] (1, -1);
\end{knot}
\end{tikzpicture}}
            \end{minipage}
        \right ) = (q-q^{-1}) I\left (
        \begin{minipage}{0.06\linewidth}
            \resizebox{\linewidth}{!}{\input{../tikz knots/R2_2_down_down.tex}}
            \end{minipage}
        \right )
        \end{equation}

        \begin{equation}
            I(
                \begin{minipage}{0.04\linewidth}
                \resizebox{\linewidth}{!}{\begin{tikzpicture}
\begin{knot}
\strand[line width=3pt, black] 
(0,0) circle[radius=1cm];
\end{knot}
\end{tikzpicture}}
                \end{minipage}
            ) = q+q^{-1}
        \end{equation}

        \begin{equation}
            I(L_1 \cup L_2) = I(L_1) \cdot I(L_2)
        \end{equation}
    \end{tcolorbox}

    Что в точности совпадает с определением полинома Джонса из
    первого доклада.

\end{multicols}

\printbibliography

\end{document}