\documentclass[12pt,a4paper]{article}
%%% Работа с русским языком
\usepackage{cmap}					% поиск в PDF
\usepackage{mathtext} 				% русские буквы в формулах
\usepackage[T2A]{fontenc}			% кодировка
\usepackage[utf8]{inputenc}			% кодировка исходного текста
\usepackage[main=english,russian]{babel}	% локализация и переносы
\selectlanguage{english}
\usepackage{caption}

%%% Дополнительная работа с математикой
\usepackage{amsmath,amsfonts,amssymb,amsthm,mathtools} % AMS
\usepackage{icomma}
\usepackage{physics}
\usepackage{multicol}
\usepackage{bm}
\usepackage{mathrsfs}
\usepackage{verbatim}

%%% Номера формул
%\mathtoolsset{showonlyrefs=true} 
%\usepackage{leqno} 

%%% Свои команды
\DeclareMathOperator{\sgn}{sgn}

\usepackage{csquotes} 
\usepackage[backend=biber,style=authoryear]{biblatex}


%%% Работа с графикой
\usepackage{graphicx}
\graphicspath{{images/}}  
\setlength\fboxsep{3pt} 
\setlength\fboxrule{1pt} 
\usepackage{wrapfig} 
\usepackage{tikz}
\usepackage{pgfplots}
\usepackage{pgfplotstable}
\usepgfplotslibrary{polar}
\pgfplotsset{compat=1.18} 

%%% Работа с таблицами
\usepackage{array,tabularx,tabulary,booktabs}
\usepackage{longtable}  
\usepackage{multirow} 
\usepackage{soul} 

%%% Теоремы
\theoremstyle{plain} 
\newtheorem{theorem}{Th}[section]
\newtheorem{proposition}[theorem]{Proposition}
\newtheorem{question}{Question}[section]
 
\theoremstyle{definition} 
\newtheorem{corollary}{Corollary}[theorem]
\newtheorem{problem}{Problem}[section]
\newtheorem{definition}{Def}[section]
 
\theoremstyle{remark} 
\newtheorem*{nonum}{Solution}

%%% Программирование
\usepackage{etoolbox} 

%%% Гиперссылки
\usepackage{hyperref}

\usetikzlibrary{knots}
\usepackage{tcolorbox}

%%% Страница
\usepackage{geometry} 
	\geometry{top=20mm, bottom=20mm, left=15mm, right=20mm}

\usepackage{setspace} 

\usepackage{lastpage} 
\usepackage{amssymb}
\usepackage{xcolor}
\DeclareMathOperator{\Ker}{Ker}
\DeclareMathOperator{\qdim}{qdim}
\usepackage[all]{xy}
\usepackage{ dsfont }

\begin{document}

\begin{center}
    \Large \textbf{Полиномы Хованова-Рожанского} \\[1em]
    \small
    Московский Физико-Технический Институт \\[0.5em]
    Лаборатория Математической и Теоретической Физики
\end{center}

\small
\begin{flushright}
\begin{tabular}{c c}
\textbf{Авторы:} & \textbf{Научные руководители:} \\[0.5em]
Артем Новохатний & Елена Ланина \\
Екатерина Цыганкова & Радомир Степанов \\
Эльдар Мифтахов & \\
\end{tabular}

\vspace{1em}

\today
\end{flushright}
\normalsize
\vspace{2em}


\tableofcontents
\vspace{2em}

\section{Полиномы Джонса}

\subsection{Вопросы}
\begin{itemize}
    \item  Можно ли из скобки Кауффмана получить полином Кауффмана?
    \item  Какие допаксиомы нужны, чтобы определить полином Джонса через скейн-соотношение на перекрестки?
    \item  Добавить другие степени в левой части скейн-соотношения. Будет ли это ещё инвариантом?
    \item Джонс нормируется на аннот. Как нормировать Ховановых?
\end{itemize}

\section{HOMFLY polynom}
\subsection{Skein-relation definition}

\subsection{$\mathcal{R}$-matrices definition}



\printbibliography

\end{document}