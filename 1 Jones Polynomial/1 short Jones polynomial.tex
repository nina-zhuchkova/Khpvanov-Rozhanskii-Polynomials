\documentclass[a4paper,8pt]{extarticle}

%%% Работа с русским языком
\usepackage{cmap}					% поиск в PDF
\usepackage{mathtext} 				% русские буквы в формулах
\usepackage[T2A]{fontenc}			% кодировка
\usepackage[utf8]{inputenc}			% кодировка исходного текста
\usepackage[english,russian]{babel}	% локализация и переносы

%%% Дополнительная работа с математикой
\usepackage{amsmath,amsfonts,amssymb,amsthm,mathtools} % AMS
\usepackage{icomma}
\usepackage{physics}
\usepackage{multicol}
\usepackage{bm}
\usepackage{mathrsfs}
\usepackage{verbatim}

%%% Номера формул
%\mathtoolsset{showonlyrefs=true} 
%\usepackage{leqno} 

%%% Свои команды
\DeclareMathOperator{\sgn}{sgn}

\usepackage{csquotes} 
\usepackage[backend=biber,style=authoryear,language=auto]{biblatex}
\addbibresource{source.bib}


%%% Работа с графикой
\usepackage{graphicx}
\graphicspath{{images/}}  
\setlength\fboxsep{3pt} 
\setlength\fboxrule{1pt} 
\usepackage{wrapfig} 
\usepackage{tikz}
\usepackage{pgfplots}
\usepackage{pgfplotstable}
\usepgfplotslibrary{polar}
\pgfplotsset{compat=1.18} 

%%% Работа с таблицами
\usepackage{array,tabularx,tabulary,booktabs}
\usepackage{longtable}  
\usepackage{multirow} 
\usepackage{caption2}[2008/03/29]
\usepackage{soul} 

%%% Теоремы
\theoremstyle{plain} 
\newtheorem{theorem}{Th}[section]
\newtheorem{proposition}[theorem]{Proposition}
 
\theoremstyle{definition} 
\newtheorem{corollary}{Corollary}[theorem]
\newtheorem{problem}{Problem}[section]
\newtheorem{definition}{Def}[section]
 
\theoremstyle{remark} 
\newtheorem*{nonum}{Solution}

%%% Программирование
\usepackage{etoolbox} 

%%% Гиперссылки
\usepackage{hyperref}

\usetikzlibrary{knots}
\usepackage{tcolorbox}

%%% Страница
\usepackage{geometry} 
	\geometry{top=20mm, bottom=20mm, left=15mm, right=20mm}

\usepackage{fancyhdr} 
 	\pagestyle{fancy}
 	\renewcommand{\headrulewidth}{1pt}  
    \rhead{\today}
    \lhead{Новохатний Артем, Цыганкова Екатерина, Мифтахов Эльдар}

\usepackage{setspace} 

\usepackage{lastpage} 


\begin{document}
\begin{center}
    \Huge Полиномы Джонса
\end{center}
\begin{multicols}{2}
\subsection*{Глобальная задача:}
Найти инварианты узлов, которые бы позволили различать их между собой. Привести эффективный алгоритм их вычисления.

\columnbreak
\subsection*{Локальная задача:}
Построить инвариантный полином Джонса через скобку Кауфмана, привести алгоритм вычисления, исследовать свойства и вычислить значение на примере простых узлов.
\end{multicols}

\section*{Мотивация}
...

\begin{multicols}{2}
    \section{Инварианты узлов, соотношения Рейдемейстера}
\begin{tcolorbox}
\begin{theorem}
(Reidemeister 1927) 

Два узла изотопны тогда и только тогда, когда могут быть получены друг из друга некоторой последовательностью локальных движений трех типов:
\begin{equation}
\Omega_1:  ~~~~~~ 
  \begin{minipage}{0.1\linewidth}
        \vspace{0pt}
      \resizebox{\linewidth}{!}{\begin{tikzpicture}
\begin{knot}[
consider self intersections,
clip width=5,
]
\strand[line width=3pt, black]
(-1, -1) to[out=45,in=0,looseness=1]
(0, 1) to[out=180,in=135,looseness=1] (1, -1);
\end{knot}
\end{tikzpicture}}
  \end{minipage} 
  \leftrightarrow
  \begin{minipage}{0.1\linewidth}
        \vspace{0pt}
      \resizebox{\linewidth}{!}{\begin{tikzpicture}
\begin{knot}[
consider self intersections,
clip width=5,
]
\strand[line width=3pt, black]
(-1, -1) to[out=60,in=180,looseness=1]
(0, 1) to[out=0,in=120,looseness=1] (1, -1);
\end{knot}
\end{tikzpicture}}
  \end{minipage} 
  \leftrightarrow
  \begin{minipage}{0.1\linewidth}
        \vspace{0pt}
      \resizebox{\linewidth}{!}{\begin{tikzpicture}
\begin{knot}[
consider self intersections,
flip crossing = 1,
clip width=5,
]
\strand[line width=3pt, black]
(-1, -1) to[out=45,in=0,looseness=1]
(0, 1) to[out=180,in=135,looseness=1] (1, -1);
\end{knot}
\end{tikzpicture}}
  \end{minipage} 
\end{equation}
\end{theorem}
\end{tcolorbox}
    \columnbreak
    asdfasdf
\end{multicols}

\end{document}